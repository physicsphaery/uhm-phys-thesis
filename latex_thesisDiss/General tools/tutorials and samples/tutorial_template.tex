%%%%%%%%%%%%%%%%%%%%%%%%%%%%%%%%%%%%%%%%%%%%%%%%%%%%%%%%%%%%
%
%  LaTeX template for Physics 274L lab reports
%  Created by Mike Hadmack on 10-26-2005
%
%  This template is intended for use with pdfLaTeX
%  if it does not compile try disabling some of \usepackage
%  statements below.
%%%%%%%%%%%%%%%%%%%%%%%%%%%%%%%%%%%%%%%%%%%%%%%%%%%%%%%%%%%%
\documentclass[10pt]{article}
\usepackage{geometry}
\geometry{letterpaper}
\usepackage{fullpage}
\usepackage{latexsym}
\usepackage{epstopdf}	% Allow use of eps images

\usepackage{setspace} 	% Set up double spacing 
\doublespacing			% just remove these lines to get single spaced text

\pdfoutput=1
\usepackage[pdftex]{graphicx}
%\usepackage{indentfirst}  % Enable to the first paragraph
\DeclareGraphicsExtensions{.pdf, .jpg, .tif, .png, .eps}

\def\half{{1 \over 2}}	% Macro for 1/2
\def\micron{{\rm \mu m}}  % Macro for micrometer

\title{My 274L Lab Report}
\author{Michael Hadmack}

\date{\today}
\begin{document}

\maketitle

%---------------------------------------------------------------%

\begin{abstract}
	This is where I will give a brief introduction to my experiment and summarize the results.
\end{abstract}

%---------------------------------------------------------------%
\section{Introduction} \label{sec:intro}
This is where you can write the introduction to your experiment.

To start a new paragraph just start on a new line like this and \TeX~will handle all of the details.  \LaTeX~is great at managing bibliographic references.  Say I wanted to cite this paragraph \cite{somesource}.  Now look in the bibliography

Here is a citation to the laser bible \cite{siegman1986l}.  You can add as many bibliographic references as you like.  For more advanced bibliography options you can check out an add-on called Bib\TeX.

It is best to edit your \LaTeX~files in a \TeX~friendly editor.  For windows I would recommend Crimson editor, and for Mac \TeX Shop.  Both of these are free to download just like the entire \TeX~typesetting system

%-----------------------------------------------------------------------------------%
\section{Procedure/Method} \label{sec:procedure}
If I wanted to use a numbered list in this sections I would do it like this:
\begin{enumerate}
	\item This is the first item
	\item An example of an inline is that of a line equation of a line is $y=m x + b$.
	\item Notice that inline equations do not have numbers
	\begin{enumerate}
		\item You can even make lists within lists like this
		\item And you can nest as many layers as you like
	\end{enumerate}
	\item Just make sure that if you use a "begin" tag you use a corresponding "end" tag as well
\end{enumerate}
%-----------------------------------------------------------------------------------%
\section{Theory}
There are three ways to include equations in a document.
\begin{enumerate}
	\item The first is inline equations as demonstrated above with single dollar signs
	\item Using double dollar signs puts equations on their own line like $$e^{i \theta} = \cos(\theta) + i \sin(\theta)$$
	\item This can also be accomplished with and equation block
\end{enumerate}

\begin{equation} \label{eqn:force}
	\vec{F} = {d \vec{p} \over d t} = m \vec{a}
\end{equation}
	

%-----------------------------------------------------------------------------------%
\section{Data \& Calculations}
The label tags that you see throughout this file can be used to refer to things later in your text such as Section \ref{sec:intro} or Equation \ref{eqn:force} and \LaTeX will take care of all of the numbering automatically.

It is simple to include data tables using this as a guide:

\begin{table}[h]
\caption{Beam characteristics}
\begin{center}
\begin{tabular}{|c|c|c|c|c|}
\hline
$\lambda \rm (\mu m)$ & $f_2 \rm(mm)$ & $\omega \rm(\mu m)$ & $\tau_{error} \rm(fs)$  \\ \hline \hline
1.0	&	61.38	&	146		&	59.1\\ \hline
1.5	&	62.60	&	150		&	59.3\\ \hline
2.0	&	63.00	&	152		&	59.8\\ \hline
2.5	&	63.18	&	154		&	60.3\\ \hline
3.0	&	63.28	&	156		&	61.1\\ \hline
3.5	&	63.34	&	158		&	62.0\\ \hline
4.0	&	63.38	&	161		&	63.0\\ \hline
4.5	&	63.41	&	164		&	64.0\\ \hline
\end{tabular}
\end{center}
\label{tbl:beamsize}
\end{table}

And Images are inserted in almost the same way.  \LaTeX~can take almost any kind of graphic files such as .jpg .png .gif .pdf .eps .tif
Uncomment the following code and insert the name of a graphics to use it in your document.  Replace the word filename with the actual image filename.  The file extension is not needed.  \LaTeX~may not always typeset images in exactly the location you specify but will try to fit it into your document to most effectively use page space.

% Warning if you do not have the graphic file in the same directory the LaTeX will not compile
% that is why this is commented
%\begin{figure}
%\begin{center}
%\includegraphics[width=1.0\textwidth]{filename}
%\caption{Autocorrelator Layout}  % caption displayed under image
%\label{fig:figurelabel}  % give it a label
%\end{center}
%\end{figure}

%-----------------------------------------------------------------------------------%
\section{Conclusion}

\subsection*{A subsection}
You can also have subsections within sections if you need them.  Notice that the star after the subsection tag suppresses the numbering of the subsection.  

\subsection*{Another subsection}
There are many more ways to organize text in \LaTeX.  The possibilities are almost endless.  Look around the internet for tutorials on how to use advanced features.

One good place to look is  {\tt http://en.wikibooks.org/wiki/LaTeX}

also my \LaTeX bookmarks {\tt http://del.icio.us/hadmack/latex}


\begin{thebibliography}{10} 
\bibitem{somesource} Some Author,
Some Book, (2005) pp.45-47
\bibitem{siegman1986l} A. Siegman. \emph{Lasers}. University Science Books, Sausalito, CA, 1986
\end{thebibliography}

\end{document}
