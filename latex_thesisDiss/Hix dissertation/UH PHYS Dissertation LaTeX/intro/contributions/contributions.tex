%----------------------------------------------------------------------------
%----------------------------------------------------------------------------
%----------------------------------------------------------------------------
%----------------------------------------------------------------------------
In the following chapters, significant and original contributions to the merger of quantum control and remote sensing LIDAR systems are highlighted. Chapters 2-4 describe analytic and numerical studies; the key results are an analysis of the average behavior of excited molecules when one takes into account the thermal properties (random orientation and velocity) of the target, the effects of different LIDAR geometries when the phase space constraints of the detector system are taken into consideration, the increased target/non-target discrimination obtained by using three color coherent population transfer, the advantages of the ``saturation'' effect of the STIRAP process and its line shape.

Chapter 5 presents a chronological description of the stages of laboratory development. The most significant data sets include three LED absorption spectra, a relatively large spectral scan of resonant dye laser induced fluorescence from a modestly prepared sample cell, a demonstration of transition discrimination in the molecular iodine energy manifold, and the temporal profile of a single resonant dye laser induced fluorescence line. An aromatic compound LIF experiment which exposes a short coming of the 1 m monochromator in the lab is included to underscore the developmental nature of these tests and to emphasize the fact that iodine is only a trial molecule: there are many possibilities for future demonstrations.

Chapter 6 encompasses an entire stage of laboratory development because of the importance and complexity of the stage: an extensive analysis of the mode content of the dye lasers. Detailed measurements of the dye laser RF beat spectrum and the dye laser transverse profiles are presented. The transverse mode profile is modeled using the Gauss-Laguerre resonator modes. 

Chapter 7 describes the current system design and discusses the remaining issues that need to be addressed before multi-color experiments can start. Beam line design including delay lines seems trivial until one considers the implementation of filtering systems to reduce the number of modes in the dye laser output. The results from an initial set of measurements of Pockels cell performance, a discussion of the efficiency of narrow band etalons, a discussion on the integration of pinhole filters into the beam line, and a test of a confocal etalon are presented.

%----------------------------------------------------------------------------
%----------------------------------------------------------------------------
