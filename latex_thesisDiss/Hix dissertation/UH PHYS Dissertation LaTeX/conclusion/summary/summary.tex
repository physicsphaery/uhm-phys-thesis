%----------------------------------------------------------------------------
%----------------------------------------------------------------------------
%----------------------------------------------------------------------------
%----------------------------------------------------------------------------
Application of quantum control ideas to the remote sensing LIDAR problem yeilds many interesting results. The numerical studies into the impact of the STIRAP process on the molecular dynamics of thermal targets inside a laser beam show that there are some non-obvious advantges to using coherent process. The ``baked potato'' data implies that the LIF LIDAR technique is inherently robust. The ability to selectively target specific transitions in molecular iodine demonstrates the viability of the multi-color method. The dye laser diagnositic shows that the current commercial lasers are in adiquate to produce quantum control of a relatively unprepared system. The confocal etalon test shows that it is possible to ``filter'' inadiquate laser soureces to produce control fields with the required parameters for control; however, the user will have to sacrafice power. In light of the geometric analysis of general LIDAR systems, sacraficing power means limiting the usefullness of the system.

%----------------------------------------------------------------------------
%----------------------------------------------------------------------------
