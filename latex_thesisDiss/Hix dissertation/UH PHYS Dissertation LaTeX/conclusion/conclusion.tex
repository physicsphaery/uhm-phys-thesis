%----------------------------------------------------------------------------
%----------------------------------------------------------------------------
%----------------------------------------------------------------------------
%----------------------------------------------------------------------------
This investigation into the application of quantum control techniques to LIDAR systems shows the theoretical potential, and details the experimental complexities involved in the merger. Atmospheric conditions are expected to force the interaction time to sub nanosecond time scales and the scale of molecular coupling constants demand a high fluence to induce the coherent processes under investigation. The only commercial broadly tunable laser source available with mJ pulse energies at the nanosecond time scale are dye lasers. As an option to undergoing the line of research outlined in \cite{Corless:1997a} passive filtering techniques are explored. 

Numerical studies of some coherent population transfer processes reveal some potential advantages these experimental techniques may have on the LIDAR problem. The ``robustness'' of the STIRAP with respect to the coupling yields near complete population inversion (even in a randomly polarized ensemble) and top hat spatial inversion profiles (even when using Gaussian beams). The ``ridge'' detuning feature may seem like a disadvantage at first; however, considering the Gaussian--like falloff when traversing the anti-ridge, this detuning feature may be exploited, for instance, to reduce the coupling to a particularly strong non-target transition to a degree not possible with Lorentzian--like detuning features. In the numerical studies mentioned above, the open nature of the problem is ignored. A simple model for the collision process combined with a parametric fit produced some intriguing initial results suggesting a mathematical form for the inclusion of relaxation processes into future numerical studies of open systems. 

In the investigation into the application of these processes on thermal molecular systems, iodine is used as the trial system since its energy structure is well understood. To make headway simulating molecular response to pulses laser output without a tractable dynamical model for molecular systems, a crude model is implemented and fluorescence responses are approximated. These numerical studies imply fluorescence extinction ratios around $10^{-8}$.

A set of fundamental measurements serve as a guide to develop the laboratory techniques, apparatus, and data acquisition systems appropriate for these types of synchronized pulsed laser experiments on thermal molecular systems. The 1 m monochromator available in the lab is fitted with a PMT and, for most fluorescence measurements, this system was adequate; however, in aromatic compound LIF experiments, the monochromator was shown to exhibit significant ``ghosting'', prompting a grating upgrade. The chart recorder has been replaced with a ADC/LabView system allowing long run times (some of the laser diagnostic scans were over two hours in length) and, because of its digital format, the data is immediately amenable to digital processing. The iodine broadband absorption tests generated a set of LED light sources; which could further be used to test the Hg pulser and a fast photodiode. The HeNe LIF experiments were used to try different beam line geometries: sideview and backscatter.

The dye laser performance tests provide insight into the complexities of these types of experiments. The wavelength ``scan'' feature of the dye lasers is not stable: it can not be used to determine the positions of absorptions features in molecular iodine. Upon the arrival of a CCD based laser beam profile imager, it is shown that the dye laser output contains more than a few spatial modes. RF analysis of the dye laser pulse intensity profiles revealed a axial mode structure with discrete features extending out to 1.8 GHz (at least) with 240 MHz spacing. Not only is the 240 MHz spacing inconsistent with the manufacture's specifications; but it exposes a property of the laser output that must be dealt with before coherent control experiments can succeed.

The sources of the multi--mode nature of the dye laser are fundamental to the operation of its cavity. The lasing medium has a broad gain bandwidth -- this provides it with broad tunability at the cost of supporting a family of modes in the cavity instead of one. To get into the mJ range in pulse energy, it is required to place the lasing modes very close to the walls of the dye flow cells. In this way the pumping efficiency is optimized, but at the cost of symmetric cavity modes. The dye cavity does not consist of two spherical mirrors; instead it has one output coupler (spherical) and two gratings. It is not clear what the dominant modes would be in such a resonator. The fact that the gain medium is suspended in a flowing solvent subject to periodic intense bursts of radiation makes the idea of a feedback system to stabilize the cavity modes to optical path length modulations daunting. In the literature, there is one example of a single mode dye laser system made specifically for coherent control experiments \cite{Corless:1997a}, but this system was severely limited in tunability making it useless for the exploratory demonstration experiments needed in the research proposed in this dissertation.

Tasks for future development must include a system to condition the dye laser output. Pinholes and confocal etalons are considered as spatial filters. The pinhole method is limited by the damage threshold of the pinhole material -- glass fibers may be the best option here. Near confocal etalons can be used to generate single mode output from multimode input and etalon mirrors can be made with high damage thresholds. This gives some hope to the possibility that a near confocal etalon may be able to solve both the transverse and axial mode problems simultaneously. Unfortunately, the multi--pass nature of the etalon places a fundamental limit on the transmitted power -- this limit decreases as resolution increases. The etalon tested here reduced 5 mW CW Green HeNe output such that the scattered radiation from a white target was not visible in room light.

The temporal shaper must also be developed since optimal Pockels cell performance has not been achieved. Current efforts are centered on a replacement for the cube beam splitters as the high power polarizer. A ``pile--of--plates'' ZnSe polarizer has been assembled but has not been tested with the Pockels cell. There may also be an issue with the Pockels cell itself. Even when used in the beam waist, the Pockels cell causes significant leakage between two crossed polarizers.

A data acquisition system must be developed to handle the stochastic nature of the experiment. Given the dye laser diagnostics presented here it is apparent that not only will the dye laser output be significantly reduced after passing through the temporal and modal filters, but its amplitude will not be able to be controlled. Discriminators might render the duty cycle of the three color experiment unworkable, so each run of the experiment will have to be recorded. This means at least four data channels: one for each of the three input laser pulses and one for the resulting fluorescence. This shot--to--shot requirement may eliminate the possibility of a sensitive CCD array in the data acquisition chain.

An immediate extension to this research is to include other molecular systems. This line of research has started with the cursory investigation of aromatics. In view of the difficulties described here with respect to multicolor coherent population transfer, this extension should await a successful demonstration in molecular iodine unless the iodine system proves untenable and an alternative molecular (or atomic) system has to be considered.

In addition to other targets, extensions could also include other coherent processes. The multi-color LIDAR idea presented here can be extended to more colors and/or separate packets. For example, the first multi-color packet could ``prepare'' the atmosphere in some way to help the secondary multi-color packets \cite{Mccall:1969a,Eberly:1998a,Scully:2001a,Oreg:1984a,Grobe:1994a}. Moreover, the processes selected for investigation here were simply plucked from the current literature. Other novel processes exist; for example, the ``quantum zeno'' effect \cite{Home:1997,Spreeuw:1997a} may allow selective control of the decay process in target molecules \cite{Itano:1990a}. Thus, not only could we exploit the large spectral shift of fluorescence energy, but also a possible \emph{temporal} shift of the fluorescence energy.

Finally, there may be implications into other areas of research; for example, these types of experiments (tunable laser targeting molecular systems) could be used to explore the field of quantum computing. \cite{Goswami:2002a} theoretically explores the possibility of ``ensemble quantum computing'' using multi-color optical spectroscopy of molecules. This remains one of the most exciting possibilities.
%----------------------------------------------------------------------------
%----------------------------------------------------------------------------
