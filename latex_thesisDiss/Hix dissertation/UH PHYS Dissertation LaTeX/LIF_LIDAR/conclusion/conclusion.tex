%----------------------------------------------------------------------------
%------------------------------Broad objectives------------------------------
%----------------------------------------------------------------------------
The equations of motion for population inversion of a thermal ensemble excited by a linearly polarized laser beam are presented; and the basic equations governing the collection efficiency of general LIDAR systems are developed.
%----------------------------------------------------------------------------
%----------------------------------So what?----------------------------------
%----------------------------------------------------------------------------
We have introduced the basic equations upon which other results in this dissertation are based. The calculated limits on collection efficiency will be referenced later to make claims on the ultimate sensitivity of LIDAR systems using coherent control.
%----------------------------------------------------------------------------
%---------------------------------Synthesize---------------------------------
%----------------------------------------------------------------------------

This research is the synthesis of two general fields: coherent quantum control and LIDAR. The most important (to this line of research) fundamental relationships from each field were derived in this chapter. The effect of the thermal velocities associated with gas samples, in fact, played an important role in both areas: in terms of coherent control, the velocity gives us a general equation of motion for laser excited gas molecules; additionally, this velocity distribution is shown to have a significant impact on the observed behavior of these molecules in bench top scale demonstrations.

The molecular orientation distribution is shown to have a ``damping'' effect on the dynamics. The best one can hope for (assuming the worst case scenario -- i.e. the classical case) is a little more than 60\% population inversion. For higher pulse energies (or equivalently longer pulse lengths) the ensemble quickly loses coherence and the population of the ground and excited states balance at 50\%/50\%. The velocity distribution is shown to lower the oscillation frequency. In addition, it reduces the maximum inversion and shifts the high energy/long pulse balance from 50\%/50\% toward the ground state. The velocity distribution also has a ``damping'' effect when one considers finite laser spot sizes and the finite acceptance space of the detection device. For our monochromator, it is found that spot sizes near 1 mm will avoid this effect.

The analysis of the backscattered LIDAR geometry led to encouraging results. First we see that the number of molecules to which the system is sensitive increases with range. This interesting result is tempered by the fact that the energy required increases at the same rate. At the moment, it seems unrealistic to expect more than 1 J per ns pulse. This limits the range of a reasonable LIDAR system using a 1 m aperture to about 1 km. The number of atmospheric molecules to which this system would be sensitive is around $10^{13}$. It is important to note that these calculations used our estimate for the matrix element, $M$, of the iodine molecule. The LIDAR ``efficiency'' ratio (Equation \ref{LIDAR eff}) depends on $M^2$ (the matrix element squared) and thus the range limits may change significantly for molecules with larger $M$'s.
%----------------------------------------------------------------------------
%----------------------------------------------------------------------------
%----------------------------------------------------------------------------
