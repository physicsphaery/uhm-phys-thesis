%----------------------------------------------------------------------------
\label{general dynamics}
%----------------------------------------------------------------------------
%----------------------------------------------------------------------------
%----------------------------------------------------------------------------
Consider an atom which interacts with a multi-mode radiation field. We model the interaction with the minimal-coupling Hamiltonian under the dipole approximation in the radiation gauge \cite{Scully:1997a}:
%----------------------------------------------------------------------------
\begin{equation}
H
=
H_{0}
+
H^{\prime}.
\end{equation}
%----------------------------------------------------------------------------
where $H_0$ is the Hamiltonian of the atomic system, namely
%----------------------------------------------------------------------------
\begin{equation}
H_0 \ket{a}
=
\hbar \omega_a \ket{a}
\end{equation}
%----------------------------------------------------------------------------
where $\ket{a}$ is some energy state of the atom, and $\hbar \omega_a$ is the energy of the state $\ket{a}$; the energy eigenbasis $\{\ket{x}\}$ is assumed complete for electrons bound to the nucleus of the atom (ionization and scattering states are not included here). $H^{\prime}$ is the interaction Hamiltonian, namely
%----------------------------------------------------------------------------
\begin{equation}
H^{\prime}
=
e \vec{E}_n \cdot \vec{r},
\end{equation}
%----------------------------------------------------------------------------
where $e$ is the charge of an electron, $\vec{E}_n$ is the local electric field at the atom due to the nth radiation mode (assumed uniform over the atom), and $\vec{r}$ is the coordinate of the electron (the atomic nucleus is the origin).

If the modes are linearly polarized in the same direction and coherent then the fields can be written as
%----------------------------------------------------------------------------
\begin{equation}
\vec{E}_n
=
\hat{\varepsilon} E_n
\frac
{e^{i (\nu_n t + \phi_n)} - e^{-i (\nu_n t + \phi_n)}}
{2 i}
\end{equation}
%----------------------------------------------------------------------------
where $\hat{\varepsilon}$ is the unit polarization vector for the modes, $E_n$ is the magnitude of the nth mode, $\nu_n$ is the frequency of the nth mode, and $\phi_n$ is the arbitrary phase of the nth mode. Now we can write
%----------------------------------------------------------------------------
\begin{equation}
H^{\prime}
=
\sum_{n}
\frac{1}{2 i}
E_n M
e^{i (\nu_n t + \phi_n)}
+
c.c.
\label{interaction}
\end{equation}
%----------------------------------------------------------------------------
where
%----------------------------------------------------------------------------
\begin{equation}
M\equiv e \hat{\varepsilon} \cdot \vec{r}
\end{equation}
%----------------------------------------------------------------------------
represents the amplitude of the coupling for the nth mode.
%----------------------------------------------------------------------------
%----------------------------------------------------------------------------
%----------------------------------------------------------------------------
%----------------------------------------------------------------------------
%----------------------------------------------------------------------------
%----------------------------------------------------------------------------
%----------------------------------------------------------------------------

The system obeys the Schr\"{o}dinger equation
%----------------------------------------------------------------------------
\begin{equation}
i \hbar \dot{\ket{\psi}}
=
H \ket{\psi}.
\label{raw se}
\end{equation}
%----------------------------------------------------------------------------
If we project the general solution $\ket{\psi}$ into $\{\ket{x}\}$, i.e.
%----------------------------------------------------------------------------
\begin{equation}
\ket{\psi}
=
\sum_a
c_a(t) e^{-i \omega_a t}
\ket{a}
\label{expansion}
\end{equation}
%----------------------------------------------------------------------------
where $c_a(t)$ is the probability amplitude associated with state $\ket{a}$ for $\ket{\psi}$, then Equation \ref{raw se} can be rewritten as \cite{Bransden:1989a}
%----------------------------------------------------------------------------
\begin{equation}
\dot{c_a}
=
\frac{1}{i \hbar}
\sum_b
H^{\prime}_{ab}
e^{(i \omega_{ab} t)}
c_b
\label{se}
\end{equation}
%----------------------------------------------------------------------------
where
%----------------------------------------------------------------------------
\begin{equation}
H^{\prime}_{ab}
=
\bra{a} H^{\prime} \ket{b},
\qquad
\ket{a},\ket{b}
\in
\{\ket{x}\}
\end{equation}
%----------------------------------------------------------------------------
and
%----------------------------------------------------------------------------
\begin{equation}
\omega_{ab}
=
\omega_a - \omega_b
\end{equation}
%----------------------------------------------------------------------------
is the transition energy. Now we combine Equations \ref{interaction} and \ref{se} to obtain the equation of motion for an atom interacting with a multi-mode radiation field:
%----------------------------------------------------------------------------
\begin{equation}
\boxed{
\dot{c_a}
=
\sum_{b,n}
\frac{E_n M_{ab}}{2 \hbar}
(
e^{-i \nu_n t}
-
e^{i \nu_n t}
)
e^{i \omega_{ab} t}
c_b,
\label{dim eom}
}
\end{equation}
%----------------------------------------------------------------------------
where $M_{ab}$ (called the dipole matrix element) is defined as
%----------------------------------------------------------------------------
\begin{equation}
M_{ab}
\equiv
\bra{a} M \ket{b},
\qquad
\ket{a},\ket{b}
\in
\{\ket{x}\}.
\label{matrix element}
\end{equation}
%----------------------------------------------------------------------------
Finally, we introduce
%----------------------------------------------------------------------------
\begin{equation}
\xi
=
\frac
{2 \hbar}
{e E_{o} a_o}
\end{equation}
%----------------------------------------------------------------------------
where $e$ is the elementary charge, $E_{o}$ is an arbitrary constant with units of the electric field, and $a_o$ is the Bohr radius, so that Equation \ref{dim eom} can be written in the dimensionless form (convenient for computer simulations)
%----------------------------------------------------------------------------
\begin{equation}
\frac
{dc_a}
{d\tau}
=
\sum_{b,n}
\Delta_{abn}
(
e^{-i \nu_n \xi \tau}
-
e^{i \nu_n \xi \tau}
)
e^{i \omega_{ab} \xi \tau}
c_b
\label{eom}
\end{equation}
%----------------------------------------------------------------------------
where
%----------------------------------------------------------------------------
\begin{equation}
\Delta_{abn}
=
\xi \frac{E_n M_{ab}}{2 \hbar}
\label{Delta}
\end{equation}
%----------------------------------------------------------------------------
and
%----------------------------------------------------------------------------
\begin{equation}
\tau
=
\frac{t}{\xi}.
\label{tau}
\end{equation}
%----------------------------------------------------------------------------
%----------------------------------------------------------------------------
%----------------------------------------------------------------------------
%----------------------------------------------------------------------------
%----------------------------------------------------------------------------
%----------------------------------------------------------------------------
%----------------------------------------------------------------------------
