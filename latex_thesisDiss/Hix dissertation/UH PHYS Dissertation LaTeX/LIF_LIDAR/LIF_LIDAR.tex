%----------------------------------------------------------------------------
%----------------------------------------------------------------------------
%----------------------------------------------------------------------------
%------------------------------Broad objectives------------------------------
%----------------------------------------------------------------------------
In this chapter, the basic equations of motion for multi-color population transfer are explored using a numerical integrator.
%----------------------------------------------------------------------------
%-------------------------concepts/results presented-------------------------
%----------------------------------------------------------------------------
Specific computer model parameters are calculated such that well known efficient multi-color pulse schemes can be simulated. To analyze robustness of various coherent processes with respect to amplitude fluctuations in the laser source, three different three-color pulse schemes are compared in a stochastic simulation. Because our target molecular system is relatively uncontrolled (trace molecular species in the atmosphere), we must consider the interaction between the target species and the environment (such systems are called ``open'' quantum systems \cite{Blum:1981a}). The de-coherence effect of collisions is studied using a simple model and the resulting motion is fit to a phenomenologically derived equation of motion using density matrix techniques in Liouville space.
%----------------------------------------------------------------------------
%------------------relevant concepts/results NOT presented-------------------
%----------------------------------------------------------------------------

The ``open'' nature of the problem is treated in an extremely simplified fashion. The target levels (four levels in the case of three-color excitation) are assumed to be the only states that will produce observable effects. It is known that in molecular systems, one typically must deal with a \emph{dense} energy level structure. To model an actual application more accurately, one must include some of the nearby energy levels in the observable system.

The only pulse schemes analyzed here are of the recently demonstrated STIRAP type (and, of course, the Gaussian $\pi$--pulse). With respect to the general detection problem at hand, there may be other pulse schemes with similar or better characteristics. For example, pulses with a temporal ``comb'' structure may be able to take advantage of the ``quantum zeno'' effect \cite{Itano:1990a}.
%----------------------------------------------------------------------------
%----------------------------------------------------------------------------
%----------------------------------------------------------------------------

See reference \cite{Vitanov:2001a} for a basic review of laser-induced adiabatic population transfer techniques. Reference \cite{Sola:1999a} reports the results of a numerical determination of the pulse parameters for STIRAP type population transfer using multi-color pulse trains. In this work, we re-calculate the results for two and three-color pulse trains. We extend the study of these pulse schemes by exploring the pulse amplitude solution space near the two-color optimum and comparing the different schemes when subject to randomized pulse amplitudes.

Reference \cite{Siegman:1986a} suggests a model for the quantum dynamical effects of collisions. This model is stochastically applied to a resonant three-color excitation of a four level system using a computer program. To model the resulting ``relaxation'' from the stochastic model, we use density matrix techniques \cite{Fano:1957a,Schirmer:2000a,Khaneja:2003a}. A relaxation matrix is written down, inspired by techniques reported in references \cite{Arimondo:1996a} and \cite{Kelley:1994a} but generated phenomenologically, and fit to the average behavior of the stochastic model.
%----------------------------------------------------------------------------
%----------------------------------------------------------------------------

%----------------------------------------------------------------------------
%----------------------------------------------------------------------------
\section{Interaction formalism}
The proposed detection scheme revolves around the laser excitation of a thermal target; i.e. the interaction of a highly prepared optical beam with a relatively unprepared system at equilibrium with ambient conditions. The general equation of motion is derived, the simple example of a two level system subject to a single mode field is examined, and the effects of random orientation and detuning effects are included.
%----------------------------------------------------------------------------
%----------------------------------------------------------------------------
\subsection{General equation of motion}
\input{LIF_LIDAR/general/general.tex}
%----------------------------------------------------------------------------
\subsection{Two level system}
\input{LIF_LIDAR/general/2_level.tex}
%----------------------------------------------------------------------------
\subsection{Semi-classical behavior of a randomly oriented ensemble}
%----------------------------------------------------------------------------
%----------------------------------------------------------------------------
\label{polarization damping}
%----------------------------------------------------------------------------
Suppose there exists an ensemble of molecules in which we can assign a classical dipole moment to each molecule. Let an incident beam of coherent linearly polarized light interact with each molecule through its dipole moment such that the coupling strength is proportional to $\hat{p} \cdot \hat{\varepsilon}$ where $\hat{p}$ is the direction of the dipole moment and $\hat{\varepsilon}$ the direction of the beam polarization. If the system is unprepared, i.e. we assume it is in thermodynamic equilibrium with its surroundings, the ``orientation polarization'' of the system will be completely incoherent.

Specifically, the normalized dipole vectors of the molecules will be uniformly distributed on the surface of the unit sphere and hence the projections of these vectors against the polarization axis of the incident light are also uniformly distributed. Thus, since the dipole interaction depends linearly on this projection, the distribution of coupling strengths in the ensemble will be uniformly distributed between zero and the maximum case (when the polarization and the dipole moment vectors point in the same direction).

%----------------------------------------------------------------------------
\input{figures/LIF_LIDAR/dynamics/dynamics.tex}
\input{figures/LIF_LIDAR/extremum/extremum.tex}
%----------------------------------------------------------------------------
Since the frequency of the Rabi oscillations depend linearly on the matrix element, the Rabi frequency is uniformly distributed in the ensemble. Now, for some molecule with a Rabi frequency reduced from the maximum by a factor $F\in[0,1]$, the probability of excitation at time $\tau$
%----------------------------------------------------------------------------
\begin{equation}
P_1(\tau)
=
\sin^2((F\Delta)\cdot\tau)
\end{equation}
%----------------------------------------------------------------------------
Thus, since the distribution is uniform, the average behavior of a (large) ensemble is
%----------------------------------------------------------------------------
\begin{equation}
P^{\prime}(\tau)
=
\int^{1}_{0}
\sin^2(F\Delta\tau)
dF
=
\frac{1}{2}
\left(
1-
\sinc{(2\Delta\tau)}
\right)
\label{initial eom}
\end{equation}
%----------------------------------------------------------------------------
or
%----------------------------------------------------------------------------
\begin{equation}
P^{\prime}(t)
=
\frac{1}{2}
\left(
1-
\sinc{(\Omega_R t)}
\right)
\label{initial eom normal}
\end{equation}
%----------------------------------------------------------------------------
where
%----------------------------------------------------------------------------
\begin{equation}
\sinc{(x)}
\equiv
\frac
{\sin{(x)}}
{x}.
\end{equation}
%----------------------------------------------------------------------------
See Figure \ref{dynamics} for a comparison between the ideal dynamics of a single molecule and the average behavior our thermal ensemble.

By examining the derivative of Equation \ref{initial eom} one finds that the extrema are located at the solutions to
%----------------------------------------------------------------------------
\begin{equation}
2 \Delta \tau
=
\tan{(2 \Delta \tau)}.
\end{equation}
%----------------------------------------------------------------------------
See Figure \ref{extremum} for a plot of the solution. The implications of these results are that more power will be required to induce the molecules to invert within a give time window (due to the forward shift of the maxima) and the resulting inversion probability will be reduced by a factor of $\sim0.6$. It can be shown that this is a worse case scenario (in terms of reduced modulation depth and forward shift of the maxima) when one compares this semi-classical behavior to the behavior resulting from a quantum mechanical treatment of the dipole approximation in the spherical harmonic basis set (personal communication, Pui K. Lam, June 2006).
%----------------------------------------------------------------------------
%----------------------------------------------------------------------------

%----------------------------------------------------------------------------
\subsection{Detuning effects}
\input{LIF_LIDAR/general/doppler.tex}
%----------------------------------------------------------------------------
%----------------------------------------------------------------------------
\section{LIDAR geometries}
For a geometry appropriate for a benchtop demonstration experiment (side view), a computer model is used to explore the effects of phase space constraints of thermal velocities on the photon capture probability. In the last subsection (Section \ref{backscatter section}), a simple geometric argument is applied to general LIDAR systems to place limits on the photon capture probability and quantify the required pulse energies for molecular control.
%----------------------------------------------------------------------------
%----------------------------------------------------------------------------
\subsection{Receiver phase space}
\input{LIF_LIDAR/geometries/phase-space.tex}
%----------------------------------------------------------------------------
\subsection{Monochromator resolution}
\input{LIF_LIDAR/geometries/mono_res.tex}
%----------------------------------------------------------------------------
\subsection{Velocity distribution}
\input{LIF_LIDAR/geometries/velocity.tex}
%----------------------------------------------------------------------------
\subsection{Side view geometry}
%----------------------------------------------------------------------------
%bb defines the bounding box for the pdf
%viewport defines the area of the pdf used
%in sidewaysfigure the last entry in bb moves the caption toward/away the pic
%in sidewaysfigure the second entry in bb moves the pic toward/away the caption
%----------------------------------------------------------------------------
\begin{figure}
\scalebox{0.7}[0.7]{
\includegraphics[bb=-60 360 489 725]
{side_view/side_view.pdf}
}
\caption[Side--view lens optimization surface]{Side--view lens optimization surface. For this calculation the beam waist diameter is 1 mm, the wavelength is 0.63 $\mu$m, the slitwidth is 300 $\mu$m, the slit height is 0.5 cm, and the monochromator acceptance angle is 52 mrad. The object lens is placed one focal length from the sample, while the image lens is placed one focal length from the monochromator input. The units for the contour in the plot are $10^{-12}$ m$^3$ (scaled volume). Thus if the target was air (near $10^{25}$ molecules per cubic meter) we would expect this system to be sensitive to $10^{13}$ atmospheric molecules (assuming all molecules decay optically in an isotropic fashion).}
\label{side_view}
\end{figure}
%----------------------------------------------------------------------------

%----------------------------------------------------------------------------
\subsection{Backscatter geometry}
\label{backscatter section}
%----------------------------------------------------------------------------
Consider an optical setup where a laser beam line is merged with receiver beam line. The merge could be facilitated through the use of a dichroic beam splitter: reflecting the wavelength associated with the laser while transmitting the broad spectrum associated with the fluorescence radiation. By allowing some of the laser focusing optics to double as the receiver optics gives rise to unique geometries where the receiver and laser source are localized and their beam lines are extended to some remote target region (a LIDAR application). See Figure \ref{backscatter_figure} for a beam line diagram.
%----------------------------------------------------------------------------
\input{figures/LIF_LIDAR/backscatter_figure/backscatter_figure.tex}
%----------------------------------------------------------------------------

The probability of photon detection is proportional to
%----------------------------------------------------------------------------
\begin{equation}
ABCD
\int\int
\rho(\vec{r})
P(\vec{r},t)
\Phi(\vec{r},\hat{c})
d\vec{r}
d\Omega
\label{total_prob}
\end{equation}
%----------------------------------------------------------------------------
where $A$ is the fraction of target molecules which couple to the laser radiation, $B$ represents atmospheric attenuation effects like Mie scattering, absorption, etc., $C$ represents the fraction of excited molecules that decay optically within the bandwidth of the detector, $D$ represents the fraction of optically decaying molecules which emit a photon before collisional effects force non-optical decay (see discussion at the end of Section \ref{side}), $\rho(\vec{r})$ is the density distribution of the target molecules, $P(\vec{r},t)$ is the molecular inversion probability (this is a function of the fluence at position $\vec{r}$ and at time $t$), $\Phi(\vec{r},\hat{c})$ is the phase space factor (a function of position as well as the direction of travel of the emitted photon $\hat{c}$). In the following analysis some approximations are made to estimate this integral. First we ignore $A$, $B$, $C$, and $D$, thus we simply calculate the photon capture probability assuming all molecules optically decay isotropically. The density is assumed uniform, $\Phi$ is simply taken to limit the integration volume to the focal volume and introduce a factor equal to the solid angle fraction available to the detector from the target region, and $P$ is assumed to be equal to unity.
%----------------------------------------------------------------------------
%----------------------------------------------------------------------------
%bb defines the bounding box for the pdf
%viewport defines the area of the pdf used
%in sidewaysfigure the last entry in bb moves the caption toward/away the pic
%in sidewaysfigure the second entry in bb moves the pic toward/away the caption
%----------------------------------------------------------------------------
\begin{figure}
\scalebox{0.8}[0.8]{
\includegraphics[bb=20 390 489 680]
{back_number/back_number.pdf}
}
\caption[Number of molecules emitting into reciever in a LIDAR application]{Number of molecules emitting into reciever acceptance area (ideal: $A=B=C=D\equiv1$ in Equation \ref{total_prob}). The ratio $R/D$ must be larger than unity to satisfy the approximations used to construct Equations \ref{energy_required} and \ref{v_eff}. Here we have used $\lambda=1$ $\mu$m, and a density of $10^{25}$ molecules per m$^3$. Thus, with an aperture of $D=1$ m and a focal length of 1 km, we expect the system to be sensitive to at most $10^{13}$ atmospheric molecules.}
\label{back_number}
\end{figure}
%----------------------------------------------------------------------------

%----------------------------------------------------------------------------

The output aperture, with diameter $D$, is taken as the origin and the target is at the focus of a Gaussian beam which is some distance, $R$, from the output aperture. The Gaussian beam emerges from the output aperture with a clear aperture compatible with a far-field ripple of less than 1\% \cite{Siegman:1986a}. The integration limits are taken as the boundaries of a rectangular volume centered at the focus: the extent of the region along the beam axis is taken as the Rayleigh range and the transverse dimensions are taken as the waist diameter. The solid angle fraction is taken as
%----------------------------------------------------------------------------
\begin{equation}
\frac{(2\theta)^2}{4\pi}
\end{equation}
%----------------------------------------------------------------------------
where $\theta$ is the far-field divergence \emph{half} angle of the Gaussian beam \cite{Siegman:1986a}. Using these ideas, one can arrive at the following expression for the scaled volume (the product of the volume and the solid angle fraction)
%----------------------------------------------------------------------------
\begin{equation}
\boxed{
V_{eff}
=
\frac{4.6^2}{2 \pi^2}
\lambda^3
\left(\frac{R}{D}\right)^2.
\label{v_eff}
}
\end{equation}
%----------------------------------------------------------------------------
The required pulse energy depends on the fluence (Equation \ref{required fluence}) and the focal spot diameter. The relationship is
%----------------------------------------------------------------------------
\begin{equation}
\boxed{
E
=
\frac{42.32}{\pi}
c \hbar^2 \epsilon_o
\frac{(\Delta\tau)^2}{M^2 t}
\lambda^2
\left(\frac{R}{D}\right)^2.
\label{energy_required}
}
\end{equation}
%----------------------------------------------------------------------------
See Figures \ref{back_number} and \ref{back_energy} for plots of these functions for various aperture sizes.

%----------------------------------------------------------------------------
%----------------------------------------------------------------------------
%bb defines the bounding box for the pdf
%viewport defines the area of the pdf used
%in sidewaysfigure the last entry in bb moves the caption toward/away the pic
%in sidewaysfigure the second entry in bb moves the pic toward/away the caption
%----------------------------------------------------------------------------
\begin{figure}
\scalebox{0.8}[0.8]{
\includegraphics[bb=20 390 489 700]
{back_energy/back_energy.pdf}
}
\caption[Required pulse energy to invert target molecules]{Required pulse energy to invert target molecules. Here we have used the dipole matrix element approximated in Section \ref{iodine} ($M=3.6\cross10^{-32}$ Cm), tophat pulse duration $t=1$ ns, $\lambda=1$ $\mu$m, and $\Delta \tau=\pi/2$ in Equation \ref{energy_required}.}
\label{back_energy}
\end{figure}
%----------------------------------------------------------------------------

%----------------------------------------------------------------------------

The key features of these equations illustrate the potential sensitivity and limitations of LIDAR geometries when one is concerned with LIF type experiments. The effective sensitive volume, Equation \ref{v_eff}, \emph{increases} as the square range, $R$. The obvious drawback is seen in Equation \ref{energy_required} where the required energy also increases with the square of the range. Indeed the ratio
%----------------------------------------------------------------------------
\begin{equation}
\frac{V}{E}
=
\frac{1}{4 \pi c \hbar^2 \epsilon_o}
\frac{M^2 t}{(\Delta\tau)^2}
\lambda
\label{LIDAR eff}
\end{equation}
%----------------------------------------------------------------------------
should be maximized for the most efficient application. $\lambda$ has a natural upper limit set by atmospheric transmission: optical wavelengths greater than 10 um are heavily attenuated. Collisional damping at STP forces $t$ to be less than 1 ns. $\Delta\tau$ is $\pi/2$ for Rabi oscillation population inversion; however, some coherent process require the quantity to increase by an order of magnitude or more (see Chapter \ref{computer chapter}). Most molecules have matrix elements, $M$, around one D ($\mbox{D}=10^{-18}\mbox{ esu}$ and $1\mbox{ Cm}=2.99792458\cross10^{9}\mbox{ esu}$).
%----------------------------------------------------------------------------
%----------------------------------------------------------------------------

%----------------------------------------------------------------------------
%----------------------------------------------------------------------------
\section{Conclusion}
%%%%%%%%%%%%%%%%%%%%%%%%%%%%%% -*- Mode: Latex -*- %%%%%%%%%%%%%%%%%%%%%%%%%%%%
%% >>conclusion/conclusion.tex<<
%% Author          : R. Jeffrey Kowalski
%% Created On      : Fri Mar 27 13:13:03 2007
%% Last Modified On: Thu Aug  2 10:54:56 HST 2007
%%%%%%%%%%%%%%%%%%%%%%%%%%%%%%%%%%%%%%%%%%%%%%%%%%%%%%%%%%%%%%%%%%%%%%%%%%%%%%%
SLAC T486 provided the confirmation of the Askaryan effect in ice:  coherent radio Cherenkov emission from high-energy particle cascades in dense media is detectable.  Such a validation has already been performed for two other dielectrics (salt, silica sand) which have similar radio properties as ice for observing the Askaryan effect.

\par In this analysis, I have demonstrated the quadratic scaling (1.84533 $\pm$ 0.07013) of Cherenkov pulse power with shower energy which indicates that the radiation is coherent over 2.6-3.95 GHz.  Within rms uncertainties, the electric field strength of radio impulses received at the standard gain horn were in good agreement with simulations incorporating electrodynamics and shower properties in ice.  This result illustrates the roll-off of the frequency spectrum for coherent radio Cherenkov emission.

\par These results provide a promising outlook for existing ultra-high energy neutrino detectors using ice as their interaction medium.  The ANITA experiment has just completed its first flight observing the vast majority of the Antarctica ice for neutrino induced electromagnetic showers.  Having accelerator-based experimental data confirming the electric field frequency spectrum that follows our current theory will improve efforts and sensitivity in detecting UHE neutrinos.  Furthermore, SLAC T486 has illuminated what effect we can expect to see from radiation patterns penetrating the ice surface from electromagnetic cascades in the  $10^{18}\rightarrow10^{22}$ eV energy regime.
%----------------------------------------------------------------------------
%----------------------------------------------------------------------------
