%----------------------------------------------------------------------------
%----------------------------------------------------------------------------
Consider the linear systems of ray optics where rays are traced through a series of optical components \cite{Saleh:1991a}. The rays are assigned two parameters at each position $z$ along the beam line: $x$ (or $y$) which represents the distance from the optic axis and $\alpha$ which represents the angle the ray makes with the optic axis. The rays form vectors in a \emph{phase space} in which their motion down the beam line is modeled; $x$ (or $y$) corresponds to the \emph{position} coordinate while $\alpha$ (which is related to the transverse velocity of the ray) corresponds to the \emph{momentum} coordinate. Optical components such as thin lenses or drift regions (free space regions between lenses) are represented by matrices which can be used to transform any ray to some other position along the beam line. In this study, ``bundles'' of rays are tracked along various beam lines to determine the collection efficiency of two beam line geometries appropriate for LIF investigation.

In terms of the ray optics phase space described above, any receiver in an optical system can be assigned a finite region of phase space to which it is sensitive. This region can be transformed to the target region where it can be overlapped with the phase space of the fluorescence in the target region. The Liouville Theorem \cite{Hassani:1999a} states that the bundle of rays will flow through phase space as an incompressible fluid. This places a fundamental limit to the coupling between the receiver and target region. Suppose the receiver has an acceptance ``half'' angle of 0.07 radians (this was an initial estimate for the monochromator used in the experiments described later in this dissertation). If we assume the spatial extent of the fluorescence is not a limiting factor, then with respect to the angular limits, the Liouville Theorem states that the receiver is sensitive to at most
%----------------------------------------------------------------------------
\begin{equation}
\frac{0.14^2}{4\pi}
=
0.00156
\label{angular overlap}
\end{equation}
%----------------------------------------------------------------------------
or a little over one thousandth of the emitted energy.

Moreover, this phase space overlap, in combination with a molecular density and inversion probability, can be interpreted as a ``photon capture probability'' for the detection system. For example, suppose there were $2\cross10^5$ molecules in the sensitive region and 50\% of the molecules were inverted and decayed optically; then the angular overlap from Equation \ref{angular overlap} implies we should capture 156 photons in our detection system.
%----------------------------------------------------------------------------
%----------------------------------------------------------------------------
%----------------------------------------------------------------------------
