%----------------------------------------------------------------------------
%------------------------------Broad objectives------------------------------
%----------------------------------------------------------------------------
The iodine molecule is introduced, the iodine energy level system is described, two basic three color schemes are proposed, the perceived advantages are described (though computer simulation), and the rationale behind each scheme design is explained.
%----------------------------------------------------------------------------
%----------------------------------So what?----------------------------------
%----------------------------------------------------------------------------

The population transfer methods described in Chapter \ref{computer chapter} are applied to a ``real'' molecular system, the $X$--$B$ system of iodine. The coherent process called STIRAP is shown to have some advantages: it reduces the inefficiency caused by random polarization and can distort the inversion cross section across the transverse profile of a Gaussian beam to a desirable top hat shape. The very difficult problem of dynamically modeling the iodine when subject to three color pulse sequences is made tractable by some sweeping assumptions. The ``robustness'' feature of the STIRAP process is placed into the context of discrimination to determine if there is an adverse affect.
%----------------------------------------------------------------------------
%---------------------------------Synthesize---------------------------------
%----------------------------------------------------------------------------

A major advantage of the STIRAP line shape remains in the steep Gaussian like fall off when traversing the anti-ridge in Figure \ref{ridge}. In the dense competing energy structure of isotopic iodine, the STIRAP ridge picked up many non-target transitions; however, in some other application there may be a single localized transition or group of transitions, which produce significant noise in usual LIF schemes, but may be selectively suppressed by designing a STIRAP excitation scheme that places the non-target transition(s) far down the anti-ridge. In this way we can exploit the distinctly non-Lorentzian characteristic of this particular coherent process to develop a high SNR detection scheme.
%----------------------------------------------------------------------------
%----------------------------------------------------------------------------
%----------------------------------------------------------------------------
