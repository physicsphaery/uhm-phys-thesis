% 2_color_ladder.tex
% by Troy Hix, March 2005
%----------------------------------------------------------------------------
%----------------------------------------------------------------------------
\begin{figure}
\setlength{\unitlength}{2cm}
\begin{center}
\begin{picture}(3.3,2.2)
\linethickness{1mm}
\put(0,2.0){\line(1,0){3}}
\put(0,1.2){\line(1,0){3}}
\put(0,0.0){\line(1,0){3}}
\put(3.1,2.0){$\ket{2}$}
\put(3.1,1.2){$\ket{1}$}
\put(3.1,0.0){$\ket{0}$}
\thinlines
\put(1.0,1.2){\vector(0,1){0.78}}
\put(1.0,0.0){\vector(0,1){1.18}}
\put(1.2,1.6){$\beta$}
\put(1.2,0.6){$\alpha$}
\end{picture}
\end{center}
\caption[Three level, two field diagram]{Three level, two field diagram. Ground state $\ket{0}$, first excited state $\ket{1}$, and second excited state $\ket{2}$ are coupled by fields $\alpha$ and $\beta$ so that population transfers from the ground state to the second excited state}
\label{2 color ladder}
\end{figure}
%----------------------------------------------------------------------------
%----------------------------------------------------------------------------
