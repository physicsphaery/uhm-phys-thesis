%----------------------------------------------------------------------------
%----------------------------------------------------------------------------
The Tektronix 7L14 and the 7L12 Spectrum Analyzers are both used here to process the signal from the photodiode. The key characteristics are their operating ranges: 1 kHz to 2.5 GHz for the 7L14 and 0.1 MHz to 1.8 GHz for the 7L12. For these data the largest resolution bandwidth, 3 MHz, is used. The output is designed to preserve Parseval's Theorem for RF electronic signals: mathematically, the output is the input's Fourier transform squared; i.e. the RF receiver's output voltage is proportional to the \emph{square} of the input voltage. Thus, since the current from the photodiode is proportional to the incident intensity of the optical signal, the receiver's output is proportional to the \emph{square} of the optical intensity. If a beam with a Gaussian intensity profile FWHM of $\sigma_t$ is incident the photodiode, it can be shown that 
%----------------------------------------------------------------------------
\begin{equation}
\delta_{\nu}
=
\frac
{\ln(16)}
{\sqrt{2}\pi\sigma_t}
\label{receiver width}
\end{equation}
%----------------------------------------------------------------------------
where $\delta_{\nu}$ is the FWHM of the resulting spectral profile generated by the RF receiver. Compare this relationship to Equation \ref{FWHM power}.

To calibrate the scans, an accurate reading of the RF receiver center frequency must be acquired. The local oscillator output from the receiver is connected to a Tektronix TR 501 tracking generator. The generator produces a signal with a frequency identical to that of the receiver local oscillator. This signal is analyzed by a Hewlett Packard 53132A universal counter, thus the center frequency of the RF receiver can be directly observed on the counter readout. The tracking generator has a smaller operating range than the 7L14 receiver, however, and can only report accurately on frequencies less than 1.8 GHz. The calibration was linearly extrapolated for frequencies above 1.8 GHz for data taken with the 7L14.

%----------------------------------------------------------------------------
%----------------------------------------------------------------------------
%----------------------------------------------------------------------------
%----------------------------------------------------------------------------
%----------------------------------------------------------------------------
