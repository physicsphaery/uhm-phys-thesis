%%%%%%%%%%%%%%%%%%%%%%%%%%%%%% -*- Mode: Latex -*- %%%%%%%%%%%%%%%%%%%%%%%%%%%%
%% >>science/intro.tex<<
%% Author          : R. Jeffrey Kowalski
%% Created On      : Thu Apr 12 23:37:21 HST 2007
%% Last Modified On: Tue Jun 26 10:35:21 HST 2007
%%%%%%%%%%%%%%%%%%%%%%%%%%%%%%%%%%%%%%%%%%%%%%%%%%%%%%%%%%%%%%%%%%%%%%%%%%%%%%%
\begin{center}
\textit{Neutrinos, they are small. \\
They have no charge and no mass \\
And do not interact at all. \\
The earth is just a silly ball \\
To them, through which they simply pass, \\
Like dustmaids down a drafty hall \\
Or photons through a sheet of glass ...\\}
\end{center}
\begin{flushright}
by John Updike~\cite{Updike.1960}\hspace{-10pt}
\end{flushright}

Today, we know that this Updike's 1960 poem has a fundamental flaw - neutrinos do have mass albeit tiny!  How massive are they?  Do they oscillate?  Are neutrinos the key to unfolding the mystery of dark matter and missing energy in the universe?

\par In 1930, Wolfgang Pauli hypothesized that a neutral particle (zero charge) could carry the missing energy in nuclear $\beta$-decays~\cite{Fermi.1934}.  Three years later, Enrico Fermi named this particle the \emph{neutrino} ($\nu$).  It wasn't until 1956 where Frederick Reines and Clyde Cowon found the first experimental evidence for the electron (\textit{anti-}) neutrino using a large 400 L water and cadmium chloride (CdCl$_{2}$) detector 12 m underground~\cite{Reines.1960}.  In this experiment, Reines and Cowon looked at the inverse $\beta$-decay ($\nu + p \rightarrow n + e^{+}$) and measured the gamma ray photons produced from positrons annihilating with electrons using photomultiplier tubes.  However, due to the nature of the weak interaction and the fact that neutrinos carry no electric charge, the probability of an interaction involving neutrinos is extremely small making detection very difficult.

\par In the 1960's, the Homestake experiment began operation by Ray Davies to hunt for solar neutrinos produced in the Sun's core from thermonuclear reactions that power the Sun.  The flux of electron-neutrinos observed was a factor of one third less than the standard solar developed by John Bahcall and others and ''solar neutrino problem" arose.

\par We now know from the Z-boson (Z$^0$) decay width, that there must be 3 light neutrinos in the Standard Model - one for each fermion family.  The neutrinos produced are detected in weak interactions as flavours:  electron-neutrino ($\nu_e$), muon-neutrino ($\nu_{\mu}$), and the tau-neutrino ($\nu_{\tau}$).  It seems that the most convincing explanation for the missing solar neutrino flux stems from the $\nu_e$ oscillating into the other neutrino flavours.  In order for oscillation to occur, neutrinos must have mass and neutrino flavours must mix~\cite{Coughlan.2006}.  The implications have developed an entirely new field of \emph{neutrino astrophysics} which will continue to unfold fundamental neutrino properties and peer into the physics of the early Universe.

\par Today, we know that neutrinos are the most penetrating form of radiation in the universe, and can pass through magnetic fields or intense radiation fields without difficulty at \emph{all energies}.  Direct measurements of high-energy neutrinos would provide confirmation of cosmic acceleration of hadrons (baryons and mesons) from higher energy particles such as protons, electrons, helium nuclei, and heavier nuclei (\emph{cosmic rays}) and would illuminate the nature of their source since neutrinos are essentially unaffected during transit.

% \par With the early success of underground neutrino experiments like that of Reines \& Cowon to a solar neutrino observatory designed by Ray Davies in the 1960's came more difficulty in detection due to very low event rates and the background signal contamination predominately coming from high energy particles from our local galaxy and from deep space.
% The latter constitutes .  All of which we call \emph{cosmic rays} with a distiction given to those with extremely high energies - ultra-high energy cosmic rays (UHECR's).

