%%%%%%%%%%%%%%%%%%%%%%%%%%%%%% -*- Mode: Latex -*- %%%%%%%%%%%%%%%%%%%%%%%%%%%%
%% >>science/gzk/gzk.tex<<
%% Author          : R. Jeffrey Kowalski
%% Created On      : Wed Apr 25 23:45:13 HST 2007
%% Last Modified On: Thu Aug  2 10:54:56 HST 2007
%%%%%%%%%%%%%%%%%%%%%%%%%%%%%%%%%%%%%%%%%%%%%%%%%%%%%%%%%%%%%%%%%%%%%%%%%%%%%%%
\section{\textbf{GZK} Limit}
\label{s:gzk}
\par The main motivation to search for cosmological high-energy neutrino sources is derived from the theory that the cosmic ray energy spectrum extends to energies greater than 10$^{20}$ eV.  If cosmic rays follow an energy spectrum given by:

\begin{equation}
\frac{dN}{dE_p} \propto E_{p}^{-2},
\label{eq:pwrLaw}
\end{equation}

\noindent we should see a steep cutoff occurring near 10$^{20}$eV due to the rapid decrease in particle flux.  The cosmic ray energy spectrum does not follow a perfect power law given by equation~\ref{eq:pwrLaw}.  The true spectrum has intrinsic properties shown in figure~\ref{fig:cosmicRayFlux} where the power law varies considerably.

\begin{figure}[htbp]
\centering
\epsfxsize=4.0in\epsfbox{figures/science/cosmicRayFlux.eps}
\caption{Cosmic Ray Energy Spectrum~\cite{cosmicRayGNU}:  The flux for lower energy cosmic rays (yellow) are associated with solar cosmic ray sources, intermediate energies (blue) with sources from our local galaxy, and higher energies (purple) with sources of extra-galactic origin.}
\label{fig:cosmicRayFlux}
\end{figure}

\par Above 10$^9$ eV, the CR energy spectrum can be described by a series of power laws, with the flux falling about 3 orders of magnitude for each decade increase in energy.  At the ''knee" ($\sim$10$^{15.5}$ eV), the spectrum steepens from E$^{-2.7}$ to E$^{-3.0}$ which is still not consistently explained after being discovered over 40 years ago~\cite{Kulikov.1959}.  Above $\sim$10$^{17.7}$ eV (''the dip"), the spectrum steepens further to E$^{-3.3}$ and then flattens out to E$^{-2.7}$ near the ''ankle" at $\sim$10$^{18.5}$ eV.  Sources above $\sim 3 \times 10^{18.5}$ eV are most likely extra-Galactic protons~\cite{Waxman.1998} produced from naturally occurring particle accelerators (i.e. AGN's, BL Lac's, GRB's, etc.).

\par In 1966, Greisen, Zatsepin, and Kuzmin~\cite{Greisen.1966,Zatsepin.1966} predicted that the flux of these UHE protons would be reduced via the process:

\begin{equation}
p + \gamma_{2.7K} \rightarrow \Delta^{*} \rightarrow n + \pi^{\pm}
\label{eq:gzk}
\end{equation}

\noindent where the decay chain of the $\pi^{\pm}$ leads to the flux of high energy neutrinos (''GZK $\nu$ flux")\footnote{$\pi^{+} \rightarrow \mu^{+} + \nu_{\mu}$; $\pi^{-} \rightarrow \mu^{-} + \bar{\nu}_{\mu}$; $\pi^{+} \rightarrow e^{+} + \nu_{e}$, $\pi^{-} \rightarrow e^{-} + \bar{\nu}_{e}$} which are observable to the visible edge of the universe, whereas CR's are limited to our local supercluster ($\leq$40 Mpc)\footnote{1 Mpc $=$ 3.0857 $\times$ 10$^{22}$ m}~\cite{ANITA.2006}.