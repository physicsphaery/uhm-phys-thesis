%%%%%%%%%%%%%%%%%%%%%%%%%%%%%% -*- Mode: Latex -*- %%%%%%%%%%%%%%%%%%%%%%%%%%%%
%% >>science/askaryan/askaryan.tex<<
%% Author          : R. Jeffrey Kowalski
%% Created On      : Fri Mar 27 13:16:40 2007
%% Last Modified On: Thu Aug  2 10:54:56 HST 2007
%%%%%%%%%%%%%%%%%%%%%%%%%%%%%%%%%%%%%%%%%%%%%%%%%%%%%%%%%%%%%%%%%%%%%%%%%%%%%%%
\section{The Askaryan Effect}
\label{s:askaryan}

\par During the development of a high-energy electromagnetic particle cascade in normal matter, photons will knock electrons from the material into the shower.  This process is known as Compton scattering ($\gamma + e_{atom}^{-} \rightarrow \gamma + e^{-}$) and dominates the likelihood for asymmetry of electrons over positrons.  Some positrons will annihilate in flight ($e^{+} + e_{atom}^{-} \rightarrow \gamma\gamma$) terminating positron trajectories while other high-energy process such as Bhabha scattering ($e^{+} + e_{atom}^{-} \rightarrow e^{+} + e^{-}$) and Moller scattering ($e^{-} + e_{atom}^{-} \rightarrow e^{-} + e^{-}$) add more electrons to the shower but do not change the total energy electron bunch propagating through the media.  The cumulative effect develops a negative charge asymmetry of 20-30 \%.  The charge excess~\cite{Zas.1992} is approximated by:

\begin{equation}
\label{eq:chargeEx}
Q = \frac{N_{e^{-}} - N_{e^{+}}}{N_{e^{-}} + N_{e^{+}}}
\end{equation}

\noindent which induces an electric current.  Since the dimensions of the clump of charged particles are small compared to the wavelength of radio waves, the electromagnetic shower radiates coherent radio Cherenkov radiation whose power is proportional to the square of the net charge in the shower.  The net charge in the shower is proportional to the primary energy so the radiated power scales quadratically with the shower energy ($P_{RF}\propto E^{2}$).  This effect was first hypothesized by Gurgen Askaryan in 1962~\cite{Askaryan.1962,Askaryan.1965} and is depicted in figure~\ref{fig:nuInteractionInIce} where ice is the dielectric medium.

\begin{figure}[htbp]
\centering
\epsfxsize=4.0in\epsfbox{figures/science/nuInteractionInIce.eps}
\caption{UHE Neutrino Interaction in Ice:  Graphical representation of a UHE $\nu$ collision with an atom in ice (not to scale).  The neutrino, travelling faster than \emph{c}, collides with an atom or molecule in the ice producing an electromagnetic shower.  At $\theta_{C} \simeq 56^{\circ}$ ($\cos\theta_{C} \simeq 1/n_{ice}$), the RF emission propagates to the surface where it refracts and is received by the ANITA payload.}
\label{fig:nuInteractionInIce}
\end{figure}

\par In the early 1990's, Zas, Halzen, and Stanev~\cite{Zas.1992} (\textit{ZHS}) provided the first detailed simulations of the electromagnetic pulses generated by shower electrons in ice.  Here, an empirical formula for the electric field pulse spectrum, \textbf{E}($\nu$), was derived giving the result:

\begin{equation}
\label{eq:EfieldinIce}
R|\bf{E}(\nu)| = 1.1 \times 10^{-7}\frac{\mathcal{E}_{shower}}{1 TeV}\frac{\nu}{\nu_0}\frac{1}{1 + 0.4(\nu/\nu_0)^2} \times e^{-(\theta - \theta_{C})^2/2\Delta\theta} \hspace{4mm} V \hspace{2mm}MHz^{-1},
\end{equation}

\noindent where R is the distance to the observation point in ice, $\mathcal{E}_{shower}$ is the shower energy, $\nu$ is the electromagnetic wave frequency, $\nu_0$ = 500 MHz, $\theta_{C}$ is the Cherenkov angle in ice, and $\Delta\theta$ = 2.4$^\circ \nu/\nu_0$.  A more recent result from Alvarez-Mu\~{n}iz, V\'{a}zquez, and Zas~\cite{Alvarez.2000} (\textit{AVZ}) gives a new parametrization for the frequency spectrum in the Cherenkov direction with finer resolution than ZHS and is valid up to $\sim$5 GHz.  This is given by:

\begin{equation}
\label{eq:zhs}
R|\bf{E}(\nu,R,\theta_C)| \simeq 2.53 \times 10^{-7}\frac{\mathcal{E}_{shower}}{1 TeV}\frac{\nu}{\nu_0}\frac{1}{1 + (\nu/\nu_0)^{1.44}} \hspace{4mm} V \hspace{2mm}MHz^{-1},
\end{equation}

\noindent where $\nu_0$ is now 1.15 GHz for ice.

\par Zas \textit{et al.} also discuss the dependence of radio frequency with the width of the Cherenkov cone and show that the two are inversely proportional~\cite{Zas.1992} which is consistent with slit diffraction if one considers the dimensions of the antenna's feed-point to be much smaller than the wavelength of electromagnetic radiation.  Since the width of the Cherenkov cone broadens at low frequencies, equation~\ref{eq:EfieldinIce} becomes inaccurate and it an analytic estimate is necessary.  Lehtinen \textit{et al.}~\cite{FORTE.2004} show that by assuming the shower is a point charge moving at the speed of light in the $\hat{z}$ direction with current\footnote{$J_x = J_y = 0$} $J_{z}(\bf{r},t) = cq(z)\delta(\bf{r}-ct\hat{z})$, the electromagnetic cascade can be approximated by a Gaussian shower profile $q(z) = Qe^{-z^{2}/2L^{2}}$ where L is the length of the electromagnetic shower and Q is the maximum attained charge excess given by equation~\ref{eq:chargeEx}.  The estimate for the electric field pulse spectrum becomes:

\begin{equation}
\label{eq:pulseSpectrum}
R|\bf{E}(\nu)| = \sqrt{2\pi}\mu\mu_{0}QL\nu sin\theta e^{-(kL)^{2}(cos\theta - 1/n)^{2}/2}
\end{equation}

\noindent where $\mu = 1$ for typical dielectrics, $\mu_0$ is the permeability of free space, and $\theta$ is the polar angle around the shower axis.  It should be mentioned here, that according to the Landau-Pomeranchuck-Migdal (LPM) effect, equations~\ref{eq:EfieldinIce} and ~\ref{eq:pulseSpectrum} are not valid when electromagnetic showers are started by particles with high energies ($\mathcal{E} > \mathcal{E}_{LPM}$, where $\mathcal{E}_{LPM} = 2.4 \times 10^{15}$ eV).  At these high energies, a small fraction of the electromagnetic shower displays characteristic elongation which reduces the width of the Cherenkov cone.  Extensive simulations have shown this effect for EeV hadronic showers in ice~\cite{Alvarez.1998}.  However, in this analysis, we need not consider LPM effects since the measured Cherenkov emission was induced by 28.5 GeV electrons discussed in section~\ref{ss:beamCurrent}.

\subsection{Efforts in UHE $\nu$ Detection}
\label{ss:askaryanExp}
Detection of UHE $\nu$ events is difficult due to their extreme rarity.  For neutrinos with fluxes comparable to the ultra-high energy cosmic ray flux at 10$^{21}$ eV shown in figure~\ref{fig:cosmicRayFlux}, detection volumes of order 10$^6$ km$^3$ sr are necessary to achieve useful sensitivity~\cite{Berezinsky.1969}.  Large scale optical Cherenkov detectors such as AMANDA~\cite{AMANDA.2006} and IceCube~\cite{IceCube.2006} have successfully demonstrated the means of detecting Cherenkov radiation from neutrino interactions $>$ 10$^{12}$ eV with Antarctic ice as a target medium but utilized $\sim$km$^3$ sr scale volumetric apertures.  Evolving the detection regime to energies from 10$^{17}$-10$^{21}$ eV involves a new method which utilizes the detection of radio Cherenkov radiation.

\par Nearly 40 years after Askaryan's prediction, the first laboratory tests of the Ask\-ar\-yan effect were performed using silica sand~\cite{Gorham.2000} and later with rock salt~\cite{Gorham.2005}.  These results provided firm confirmation of coherence of RF emission within a dielectric medium.  Furthermore, other analysis has shown that most of the energy from an Askaryan pulse arrives with $\sim$0.1 ns~\cite{Miocinovic.2006}.  Prior to these measurements, in the late 1990's the Radio Ice Cherenkov Experiment (RICE)~\cite{RICE.2003} began operation at the South Pole to test the validity of the Askaryan effect.  Since mid-2000, the Goldstone Lunar Ultra-high energy neutrino Experiment (GLUE)~\cite{GLUE.2004} became an ongoing experiment to probe 10$^{21}$ eV energy neutrinos using the regolith in the lunar surface as a dielectric medium.  The ability to detect coherent pulsed radio emission from lunar distances provoked the interest to develop satellite and balloon-payload experiments to observe large volumes of ice existing on Earth.  The Fast On-orbit Recorder of Transient Events (FORTE) using dual-polarization antennas with a frequency range of 30 to 300 MHz observed the Greenland ice sheet and set the first experimental limit on neutrino fluxes in the 10$^{22}$-10$^{25}$ eV energy regime~\cite{FORTE.2004}.  In 2003, the balloon-borne experiment, ANITA-lite, set the best limits on neutrino fluxes above 10$^{19.5}$ eV~\cite{ANITA.2006} after flying a primarily engineering test over the Antarctic ice sheet with the TIGER payload.

\par The success of the ANITA-lite mission has provided the support to develop a full-scale ANITA detector to improve the sensitivity of constraining neutrino fluxes in the UHE region and make the first discovery of a UHE neutrino of extragalactic origin.