%%%%%%%%%%%%%%%%%%%%%%%%%%%%%% -*- Mode: Latex -*- %%%%%%%%%%%%%%%%%%%%%%%%%%%%
%% >>slacT486/analysis/dataAcq.tex<<
%% Author          : R. Jeffrey Kowalski
%% Created On      : Mon Jul  2 15:35:51 HST 2007
%% Last Modified On: Thu Aug  2 10:54:56 HST 2007
%%%%%%%%%%%%%%%%%%%%%%%%%%%%%%%%%%%%%%%%%%%%%%%%%%%%%%%%%%%%%%%%%%%%%%%%%%%%%%%
\section{Data Acquisition}
\label{s:acq}
All measurements of pulsed RF emission were made in the time domain with the ANITA instrument (described in section~\ref{s:DACsystem}); a Tektronix 694C real-time digital sampling oscilloscope with 3 GHz of bandwidth and 10 GSa/s, eight-bit linear digitization of four channels; and with a Hewlett-Packard 54121A Digital Mainframe Oscilloscope with 20 GHz of bandwidth.  The signals received at the ANITA antennas were attenuated by 60 dB to bring signals into an acceptable range for 50 $\ohm$ inputs while 20 dB of attenuation was used on received signals from the dipole antenna.  Table~\ref{tab:extAntenna} summarizes the mapping of antennas to the TDS694C and HP54121A with a switch occurring on the morning of June 20, 2006. \vspace{-15mm}

\begin{center}
\begin{table}
\caption{A summary of the additional antenna used in T486 with their mapping to oscilloscope.  The top half of the table shows the initial setup for data acquisition and the bottom half denotes the setup after 06/20/06 $\sim$2am PDT.}
\begin{tabular}{| l | c | c | c | c | c |} \hline
Antenna & Polarization & Channel & Device & Distance from Ice Target \\
\hline \hline
PCB LPDA & H & 1 & TDS694C & 5.2 m \\
PCB LPDA & V & 2 & TDS694C & 5.2 m \\
Gain Horn & V & 3 & TDS694C & 4.5 m\\
Dipole & & 4 & TDS694C & N/A \\
LPDA & H & 1 & HP54121A & 5.2 m \\
LPDA & V & 2 & HP54121A & 5.2 m \\
Monocone & H & 3 & HP54121A & 5.2 m \\
Monocone & V & 4 & HP54121A & 5.2 m \\
\hline
Monocone & H & 1 & TDS694C & 5.2 m \\
Monocone & V & 2 & TDS694C & 5.2 m \\
Gain Horn & V & 3 & TDS694C & 4.5 m\\
Dipole & & 4 & TDS694C & N/A \\
LPDA & H & 1 & HP54121A & 5.2 m \\
LPDA & V & 2 & HP54121A & 5.2 m \\
PCB LPDA & H & 3 & HP54121A & 5.2 m \\
PCB LPDA & V & 4 & HP54121A & 5.2 m \\
\hline
\end{tabular}
\label{tab:extAntenna}
\end{table}
\end{center}

\par A typical run during T486 captured $\sim$1000 events with the TDS694C while data acquisition with the HP54121A was a ''work in progress" and will not be presented in this analysis.  Figure~\ref{fig:wvfmsRun126} shows a snapshot from event 888 in run 126 with an approximate shower energy of  2.3 $\times$ 10$^{19}$ eV.

\begin{figure}[htbp]
\centering
\epsfxsize=6.2in\epsfbox{figures/slacT486/run126AllWvfms_2.eps}
% \epsfxsize=4.0in\epsfbox{figures/slacT486/run126AllWvfms_3.eps}
\caption{Waveforms with the respective power spectrum from run 126, event 888.}
\label{fig:wvfmsRun126}
\end{figure}