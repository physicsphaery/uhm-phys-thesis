%%%%%%%%%%%%%%%%%%%%%%%%%%%%%% -*- Mode: Latex -*- %%%%%%%%%%%%%%%%%%%%%%%%%%%%
%% >>slacT486/analysis/finalResults.tex<<
%% Author          : R. Jeffrey Kowalski
%% Created On      : Mon Jul  2 15:35:51 HST 2007
%% Last Modified On: Thu Aug  2 10:54:56 HST 2007
%%%%%%%%%%%%%%%%%%%%%%%%%%%%%%%%%%%%%%%%%%%%%%%%%%%%%%%%%%%%%%%%%%%%%%%%%%%%%%%
\newpage
\section{Electric Field Reconstruction}
\label{ss:PulseRecon}
The electric field of the pulsed RF Cherenkov emission can be calculated using ~\ref{eq:heff}.  Each waveform has been zero-mean subtracted with a Gaussian window applied in the time-domain.  Figure~\ref{fig:hornWvfmPsd} shows a sample waveform from run 102 with the above corrections applied.

\begin{figure}[htbp]
\centering
\epsfxsize=5.2in\epsfbox{figures/slacT486/run102hornPulse.eps}
\caption{(Top) Waveform received by the horn antenna in run 102 with the corresponding power spectrum (bottom).}
\label{fig:hornWvfmPsd}
\end{figure}

\noindent Effects due to cable corrections were negligible for this analysis due to the fact that the amplitude change was insignificant.  Furthermore, the TDS694C only has 3 GHz of bandwidth so a correction for the frequency roll-off of the oscilloscope is necessary.  A picosecond pulse was injected into the scope and the measured frequency spectrum was recognized to have significant contributions in power above the quoted roll-off from the manufacturer.  This effect, however, was not used as a correction to the data since the change in field strength was collectively unnoticeable.  A time-domain representation of the effective height was not calculated, so the electric field was calculated in the Fourier-domain with 1 MHz bin resolution.  A careful FDTD representation of the effective height may improve the quality of these results but was not taken into account for this result.

\par The theory in equation~\ref{eq:zhs} was derived for a pulse detected at a large distance from a shower initiated by a single particle.  In this case, we can treat the shower as a single source emitter with radiation coherent over the full length of the shower.  Figure~\ref{fig:fieldVsFreq} shows the measured electric field strength for several different antennas used during T486.

\begin{figure}[htbp]
\centering
\epsfxsize=5.2in\epsfbox{figures/slacT486/spectrum.eps}
\caption{Electric field strength vs. frequency of radio Cherenkov radiation from the T486 experiment for several different antennas used, including the expected theoretical curve~\cite{Zas.1992, Alvarez.2000}.  The lower frequency data was borrowed from Gor\-ham \textit{et al.}~\cite{ANITA.2007} measured with the ANITA instrument.  The high frequency data came from this work with the Standard Gain Horn which illustrates the roll-off of the Cherenkov radiation from electromagnetic showers in ice with frequency.}
\label{fig:fieldVsFreq}
\end{figure}

\noindent The low-frequency data was extracted from Gorham \textit{et al.}~\cite{ANITA.2007} measured with upper and lower Seavey horns, bicones, and discone antennas.  The errors for the ANITA data are statistical, and are $\pm$40\% in field strength ($\pm$3 dB).  The high-frequency data was calculated from run 102 with the standard gain horn.  The standard Fourier transform\footnote{$E(\omega) = \int_{-\infty}^{+\infty} E(t)e^{i \omega t} dt$} of the measured electric field was multiplied by a factor of 2 in order to agree with equation~\ref{eq:zhs}.  The uncertainty in these data is the rms (root-mean-squared) of the field strength for the entire run which yielded approximately $\pm$50\% error.  The theoretical curve was generated with $\mathcal{E}_{shower}$ = 2.86 $\times$ 10$^{19}$ eV and $\nu_0$ = 1.15 GHz.

% \subsection{Discussion of Analysis Code (\textit{Optional})}
% If I have time to talk about Python.