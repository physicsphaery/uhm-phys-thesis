%%%%%%%%%%%%%%%%%%%%%%%%%%%%%% -*- Mode: Latex -*- %%%%%%%%%%%%%%%%%%%%%%%%%%%%
%% >>introduction/thesis/thesis.tex<<
%% Author          : R. Jeffrey Kowalski
%% Created On      : Mon Jul 23 09:55:07 HST 2007
%% Last Modified On: Wed Jul 25 20:57:54 HST 2007
%%%%%%%%%%%%%%%%%%%%%%%%%%%%%%%%%%%%%%%%%%%%%%%%%%%%%%%%%%%%%%%%%%%%%%%%%%%%%%%
% Briefly overview the entire document making specific call reference to chapters and appendices.
\par Chapter~\ref{c:nuAstro} discusses the developments in the field of neutrino astrophysics with an emphasis on the GZK cutoff (~\ref{s:gzk}, ~\ref{s:uhecrDetection}) and the Askaryan Effect (~\ref{s:askaryan}, ~\ref{ss:askaryanExp}) which the ANITA experiment relies on.

\par In chapter~\ref{c:anita}, I provide an overview of the ANITA experiment.  Here I outline the gondola design (~\ref{s:design}) with a brief overview of the payload (~\ref{s:payload}).  Special attention is directed toward the ANITA data acquisition system (~\ref{s:DACsystem}) since the vast majority of my work with the ANITA collaboration was spent on the development of the RF and data acquisition electronics.

\par The first measurement of the Askaryan effect in ice, SLAC T486, is discussed in chapter~\ref{c:slac} outlining the experiment~\ref{s:slacSetup}, data acquisition~\ref{s:acq}, and analyses performed (~\ref{ss:dipoleCal}, ~\ref{ss:gainHorn}, ~\ref{ss:PulseRecon}).  I finish with a discussion of the results from SLAC T486 in chapter~\ref{c:concl}.