%%%%%%%%%%%%%%%%%%%%%%%%%%%%%% -*- Mode: Latex -*- %%%%%%%%%%%%%%%%%%%%%%%%%%%%
%% >>introduction/radio/radio.tex<<
%% Author          : R. Jeffrey Kowalski
%% Created On      : Mon Jul 23 09:55:07 HST 2007
%% Last Modified On: Thu Aug  2 10:54:56 HST 2007
%%%%%%%%%%%%%%%%%%%%%%%%%%%%%%%%%%%%%%%%%%%%%%%%%%%%%%%%%%%%%%%%%%%%%%%%%%%%%%%
\par The progress in capabilities of using coherent radio Cherenkov emission to detect high energy particles has provided a possible alternative for discovering ultra-high energy (UHE) neutrinos.  Since Askaryan's early prediction that electromagnetic showers in dense media would produce an excess charge that generates coherent Cherenkov pulses in radio frequencies~\cite{Askaryan.1962,Askaryan.1965}, many new detectors have arisen to exploit the effect of neutrino induced showers with the Moon~\cite{GLUE.2004}, Greenland ice~\cite{FORTE.2004}, and Antarctic ice~\cite{RICE.2003,AMANDA.2006,ANITA.2006} as the target medium.  The validity of this method was first emphasized in a detailed study combining electrodynamics with electromagnetic shower simulations in ice~\cite{Zas.1992}.  Furthermore, a careful treatment of the detection of radio pulses from high energy showers has been examined under the classic descriptions of Fraunhofer~\cite{Alvarez.2000} and Fresnel~\cite{Buniy.2001} zones.