%%%%%%%%%%%%%%%%%%%%%%%%%%%%%% -*- Mode: Latex -*- %%%%%%%%%%%%%%%%%%%%%%%%%%%%
%% >>instrument/concept/concept.tex<<
%% Author          : R. Jeffrey Kowalski
%% Created On      : Tue Apr 10 19:40:36 HST 2007
%% Last Modified On: Thu Aug  2 10:54:56 HST 2007
%%%%%%%%%%%%%%%%%%%%%%%%%%%%%%%%%%%%%%%%%%%%%%%%%%%%%%%%%%%%%%%%%%%%%%%%%%%%%%%
\section{Design Concept}
\label{s:design}

\begin{figure}[htbp]
\centering
\epsfxsize=4.0in\epsfbox{figures/science/anitaConceptView_hemisphere.eps}
\caption{Conceptual View of the ANITA Experiment:  At a float altitude of $\sim$37 km, the ANITA payload is tethered from a NASA LDB filled with He to a volume near 2 million m$^3$ and a diameter of $\sim$175 m.  Due to the low RF attenuation of Antarctic ice, the ANITA receiver system is able to detect EMP's emanating from UHE $\nu$'s interactions in the ice.}
\label{fig:anitaConcept}
\end{figure}

The ANITA experiment, depicted in figure~\ref{fig:anitaConcept}, is a radio telescope designed to utilize the Akaryan effect from neutrino cascades produced in Antarctic ice.  When a UHE neutrino interacts with an atom or molecule in ice, an electromagnetic particle shower develops that produces coherent radio emission at the Cherenkov angle.  The resulting RF refracts at the ice-air interface producing a detectable impulsive radio signal observable by the payload $\sim$37 km above the surface.  During the austral summer, a vortex develops in the upper stratosphere creating circumpolar winds which make ideal conditions for sending LDB-based experiments above the Antarctic continent.

\par Antarctica is an excellent site for utilizing the Askaryan effect due to low presence of RF background events otherwise detected in most parts of the world.  The combined thermal noise of the ice at T$_{eff}$ $\sim$ 250 K, with the receiver noise temperature allows triggering of impulsive events well into thermal noise levels which was recognized with the ANITA-lite experiment~\cite{ANITA.2006}.  Furthermore, since Antarctica has regions of deep ice (1-4 km range) that is radio transparent up to $\sim$1 GHz~\cite{Barwick.2005}, ANITA gains an effective telescope area of $\sim$10$^6$ km$^2$ and volumetric aperture of 1260 km$^3$ sr at $3 \times 10^{18}$ eV~\cite{ANITAprop.2003}.