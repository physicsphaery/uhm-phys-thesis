%----------------------------------------------------------------------------
%----------------------------------------------------------------------------
The main noise issue in LIDAR applications is the intervening atmosphere, not just between the target region and the receiver but also at the target region itself. In the following argument, we ignore the effects of the intervening atmosphere between the target region and transmitter/receiver. This assumption is suspect for long range demonstrations since the intervening atmosphere may alter key characteristics of the laser beam on its way to the target region; however, for bench top demonstrations, this assumption is valid since the path length between the transmitter and target region is relativity small.

Once a pulse train is transported to the target region, with its key characteristics intact, one must still contend with the distinct possibility that the non-target molecules will emit/scatter radiation along with the target molecules. At concentration ratios near unity, careful selection (or tuning) of the excitation laser pulse wavelength is usually enough to ensure the target fluorescence dominates; however, when one considers the problem of detecting low concentration targets, i.e. smaller than a part per billion, simple wavelength tuning is not sufficient to bring the SNR above unity. This is mainly due to two factors. First, the molecular constituents of the atmosphere typically have a dense energy level structure making the tuning approach analogous to hitting an archery target through a thick grove of trees. Secondly, the fluorescence process is exponentially damped in time, giving it a Lorentzian response in frequency. This gives rise to the famous 6dB/octave roll off, which allows the weak fluorescence response from an abundant non-target molecule to dominate the strong fluorescence response from a dilute target molecule.

Inspired by the recent successes in the application of quantum control to molecular systems \cite{Weinacht:2001a,Jeanneret:1998a,Schiemann:1993a} we propose that application of the concepts and methods of multi-color coherent molecular control to the LIDAR remote sensing problem will lead to systems with increased sensitivity and selectivity. Through the use of computer simulations, it is shown that multi--color techniques might allow signal (target molecule) to noise (non--target atmospheric constituents) ratios around $10^8$, implying remote detection of trace molecules in the atmosphere with concentrations as low as $10^{11}$ molecules/cm$^3$ (unity SNR). Motivated by these theoretical insights, we execute a battery of basic laboratory tests to develop the apparatus, methods, and knowledge associated with multi-color spectroscopy. This work took place in six main stages using molecular iodine as a trial target: 1) broadband absorption spectroscopy using ``light emitting diode'' (LED) output, 2) red and green HeNe laser induced fluorescence (LIF), 3) narrowband absorption spectroscopy using dye laser output, 4) resonant LIF using dye laser output, 5) resonant LIF decay (single line), and 6) dye laser output mode spectrum diagnostic. In stage (6) it is discovered that some of the inherent characteristics of the dye laser sources used here are incompatible with molecular coherent control. Currently, research efforts are centered on dealing with the multi-mode nature of these laser sources.

%----------------------------------------------------------------------------
%----------------------------------------------------------------------------
%----------------------------------------------------------------------------
