%----------------------------------------------------------------------------
%------------------------------Broad objectives------------------------------
%----------------------------------------------------------------------------
A series of measurements haves shown the dye laser to be a multi--mode source. The transverse mode content is revealed using CCD images of the transverse profile of the dye laser output. A RF receiver has exposed multiple discrete spectral features in the beat spectrum of the intensity profile of the pulses which are interpreted as the result of interference between axial modes with 240 MHz spacing. Given the length of the cavity (about 30 cm), this spacing is consistent with the mode spectrum of a \emph{confocal} cavity \cite{Siegman:1986a}.
%----------------------------------------------------------------------------
%----------------------------------So what?----------------------------------
%----------------------------------------------------------------------------

The results in this chapter imply that before embarking on an exploratory program of coherent molecular control processes the dye laser output must be conditioned if we choose to leave the dye laser system intact. \cite{Corless:1997a} details an effort to generate dye laser output for use in a coherent control experiment; but at the cost of power and tuning range. There may be fundamental issues keeping the possibility of broad tunability, high peak power, and mode purity out of the question for systems based on fluorescent dye.
%----------------------------------------------------------------------------
%---------------------------------Synthesize---------------------------------
%----------------------------------------------------------------------------
%----------------------------------------------------------------------------
%----------------------------------------------------------------------------
%----------------------------------------------------------------------------
