%----------------------------------------------------------------------------
%----------------------------------------------------------------------------
The YAG--pumped dye laser system produces 8 ns pulses at 20 Hz - in the transform limit, these pulses should have a spectral width of 55 MHz. In fact, the manufacturer of the dye lasers claims the spectral width of each laser pulse to be 1 GHz with axial modes separated by 600 MHz. These specifications are assumed to not be base on measurement (they could not produce hard data), so we assume they arrived at these numbers through some calculation. For example, with a quick glance at the dye laser cavity, one can see that its approximate length is 30 cm. This corresponds to an axial mode spacing of 500 MHz which is consistent with the manufacture's claim.

If the dye laser output contains multiple longitudinal modes, this should show up as ``beating'' in the intensity profile. This can be seen by direct observation of the laser output on a square law detector using an oscilloscope; however, this method is usually limited by the bandwidth of the scope. For example, if the manufacture's claim is true and the dye laser output has 2 or 3 axial modes separated by 600 MHz, then we should see intensity beats at 600 MHz and 1200 MHz - this is beyond the bandwidth of typical oscilloscopes. However, when analyzed with a narrow band RF receiver, there should be 1 or 2 spectral features (in addition to the ``DC'' feature) at 600 MHz and 1200 MHz - well within the bandwidth of the RF receivers we have in the lab. See the discussion in Chapter 19 in reference \cite{Siegman:1986a}.
%----------------------------------------------------------------------------
%----------------------------------------------------------------------------
