%----------------------------------------------------------------------------
%----------------------------------------------------------------------------
A rough estimate of $M^2$ for the beam from dye laser \#22 is obtained from data acquired using the Spiricon CCD array system. Many shots were acquired at various positions downstream from a +2 m lens (more than 1000 at each position). For each shot, the Spiricon system records the $\frac{1}{e^2}$ full widths (the result of a Gaussian fit) of both the horizontal and vertical projections of the beam cross section. These widths are averaged (first horizontal and vertical, then to simplify the analysis these two are averaged together) over each set to give a single width at each beam position.

A Gaussian beam profile is manually fit to these data. The two free parameters used in the fit were waist position and far field divergence. The equation for the $\frac{1}{e^2}$ envelope of a Gaussian beam is \cite{Saleh:1991a}
%----------------------------------------------------------------------------
\begin{equation}
w(z)
=
w_0
\sqrt{
1+
\frac
{z-z_0}
{z_R}^2
},
\end{equation}
%----------------------------------------------------------------------------
where
%----------------------------------------------------------------------------
\begin{equation}
w_0
=
\sqrt{
\frac
{\lambda z_R}
{\pi n}
},
\end{equation}
%----------------------------------------------------------------------------
and $z_0$ is the position of the waist, $z_R$ is the Rayleigh parameter, $\lambda$ is the wavelength, and $n$ is the index of refraction. The Gaussian beam is propagated by keeping track of the complex source point, $q$. The real part of $q$ is the distance the waist lags behind the origin and the imaginary part of $q$ is the Rayleigh parameter. In free space, $q$ propagates according to
%----------------------------------------------------------------------------
\begin{equation}
q^\prime
=
q
+L
\label{free space}
\end{equation}
%----------------------------------------------------------------------------
where $L$ is the distance propagated. When transmitting through a thin lens, the $q$ parameter obeys \cite{Saleh:1991a}
%----------------------------------------------------------------------------
\begin{equation}
q^\prime
=
\frac{q}
{\frac{-q}{f}+1}
\label{thin lens}
\end{equation}
%----------------------------------------------------------------------------
where $f$ is the focal length of the lens. The resulting hand fit implies the dye laser beam waist is 53 inches behind the output surface with $w_0=1.4$ mm when $\lambda=628$ nm ($q=1.346+9.808i$ in SI units). The Spiricon data imply the raw beam has a waist radius of $\omega_o=439$ $\mu$m, with a far field divergence of 747 $\mu$rad (half the total divergence of the beam). Since \cite{Saleh:1991a}
%----------------------------------------------------------------------------
%----------------------------------------------------------------------------
\begin{equation}
M^2
=
\frac
{\theta\pi D}
{\lambda},
\end{equation}
%----------------------------------------------------------------------------
where $\theta$ is the far field divergence (half total divergence) and $D$ is the waist (half width) of the raw laser beam; our estimate of $M^2$ is 1.6. See figure \ref{M_squared}. One typical interpretation of $M^2$ is that it represents the number of modes present in the beam(technically in the x and y directions separately, but we have averaged the two together to simplify the results) so 1.6 may seem to imply near single mode operation with Gaussian transverse profiles; however, as we will see in Section \ref{HG section} this is not the case.
%----------------------------------------------------------------------------
%----------------------------------------------------------------------------
%----------------------------------------------------------------------------
A rough estimate of $M^2$ for the beam from dye laser \#22 is obtained from data acquired using the Spiricon CCD array system. Many shots were acquired at various positions downstream from a +2 m lens (more than 1000 at each position). For each shot, the Spiricon system records the $\frac{1}{e^2}$ full widths (the result of a Gaussian fit) of both the horizontal and vertical projections of the beam cross section. These widths are averaged (first horizontal and vertical, then to simplify the analysis these two are averaged together) over each set to give a single width at each beam position.

A Gaussian beam profile is manually fit to these data. The two free parameters used in the fit were waist position and far field divergence. The equation for the $\frac{1}{e^2}$ envelope of a Gaussian beam is \cite{Saleh:1991a}
%----------------------------------------------------------------------------
\begin{equation}
w(z)
=
w_0
\sqrt{
1+
\frac
{z-z_0}
{z_R}^2
},
\end{equation}
%----------------------------------------------------------------------------
where
%----------------------------------------------------------------------------
\begin{equation}
w_0
=
\sqrt{
\frac
{\lambda z_R}
{\pi n}
},
\end{equation}
%----------------------------------------------------------------------------
and $z_0$ is the position of the waist, $z_R$ is the Rayleigh parameter, $\lambda$ is the wavelength, and $n$ is the index of refraction. The Gaussian beam is propagated by keeping track of the complex source point, $q$. The real part of $q$ is the distance the waist lags behind the origin and the imaginary part of $q$ is the Rayleigh parameter. In free space, $q$ propagates according to
%----------------------------------------------------------------------------
\begin{equation}
q^\prime
=
q
+L
\label{free space}
\end{equation}
%----------------------------------------------------------------------------
where $L$ is the distance propagated. When transmitting through a thin lens, the $q$ parameter obeys \cite{Saleh:1991a}
%----------------------------------------------------------------------------
\begin{equation}
q^\prime
=
\frac{q}
{\frac{-q}{f}+1}
\label{thin lens}
\end{equation}
%----------------------------------------------------------------------------
where $f$ is the focal length of the lens. The resulting hand fit implies the dye laser beam waist is 53 inches behind the output surface with $w_0=1.4$ mm when $\lambda=628$ nm ($q=1.346+9.808i$ in SI units). The Spiricon data imply the raw beam has a waist radius of $\omega_o=439$ $\mu$m, with a far field divergence of 747 $\mu$rad (half the total divergence of the beam). Since \cite{Saleh:1991a}
%----------------------------------------------------------------------------
%----------------------------------------------------------------------------
\begin{equation}
M^2
=
\frac
{\theta\pi D}
{\lambda},
\end{equation}
%----------------------------------------------------------------------------
where $\theta$ is the far field divergence (half total divergence) and $D$ is the waist (half width) of the raw laser beam; our estimate of $M^2$ is 1.6. See figure \ref{M_squared}. One typical interpretation of $M^2$ is that it represents the number of modes present in the beam(technically in the x and y directions separately, but we have averaged the two together to simplify the results) so 1.6 may seem to imply near single mode operation with Gaussian transverse profiles; however, as we will see in Section \ref{HG section} this is not the case.
%----------------------------------------------------------------------------
%----------------------------------------------------------------------------
%----------------------------------------------------------------------------
A rough estimate of $M^2$ for the beam from dye laser \#22 is obtained from data acquired using the Spiricon CCD array system. Many shots were acquired at various positions downstream from a +2 m lens (more than 1000 at each position). For each shot, the Spiricon system records the $\frac{1}{e^2}$ full widths (the result of a Gaussian fit) of both the horizontal and vertical projections of the beam cross section. These widths are averaged (first horizontal and vertical, then to simplify the analysis these two are averaged together) over each set to give a single width at each beam position.

A Gaussian beam profile is manually fit to these data. The two free parameters used in the fit were waist position and far field divergence. The equation for the $\frac{1}{e^2}$ envelope of a Gaussian beam is \cite{Saleh:1991a}
%----------------------------------------------------------------------------
\begin{equation}
w(z)
=
w_0
\sqrt{
1+
\frac
{z-z_0}
{z_R}^2
},
\end{equation}
%----------------------------------------------------------------------------
where
%----------------------------------------------------------------------------
\begin{equation}
w_0
=
\sqrt{
\frac
{\lambda z_R}
{\pi n}
},
\end{equation}
%----------------------------------------------------------------------------
and $z_0$ is the position of the waist, $z_R$ is the Rayleigh parameter, $\lambda$ is the wavelength, and $n$ is the index of refraction. The Gaussian beam is propagated by keeping track of the complex source point, $q$. The real part of $q$ is the distance the waist lags behind the origin and the imaginary part of $q$ is the Rayleigh parameter. In free space, $q$ propagates according to
%----------------------------------------------------------------------------
\begin{equation}
q^\prime
=
q
+L
\label{free space}
\end{equation}
%----------------------------------------------------------------------------
where $L$ is the distance propagated. When transmitting through a thin lens, the $q$ parameter obeys \cite{Saleh:1991a}
%----------------------------------------------------------------------------
\begin{equation}
q^\prime
=
\frac{q}
{\frac{-q}{f}+1}
\label{thin lens}
\end{equation}
%----------------------------------------------------------------------------
where $f$ is the focal length of the lens. The resulting hand fit implies the dye laser beam waist is 53 inches behind the output surface with $w_0=1.4$ mm when $\lambda=628$ nm ($q=1.346+9.808i$ in SI units). The Spiricon data imply the raw beam has a waist radius of $\omega_o=439$ $\mu$m, with a far field divergence of 747 $\mu$rad (half the total divergence of the beam). Since \cite{Saleh:1991a}
%----------------------------------------------------------------------------
%----------------------------------------------------------------------------
\begin{equation}
M^2
=
\frac
{\theta\pi D}
{\lambda},
\end{equation}
%----------------------------------------------------------------------------
where $\theta$ is the far field divergence (half total divergence) and $D$ is the waist (half width) of the raw laser beam; our estimate of $M^2$ is 1.6. See figure \ref{M_squared}. One typical interpretation of $M^2$ is that it represents the number of modes present in the beam(technically in the x and y directions separately, but we have averaged the two together to simplify the results) so 1.6 may seem to imply near single mode operation with Gaussian transverse profiles; however, as we will see in Section \ref{HG section} this is not the case.
%----------------------------------------------------------------------------
%----------------------------------------------------------------------------
%----------------------------------------------------------------------------
A rough estimate of $M^2$ for the beam from dye laser \#22 is obtained from data acquired using the Spiricon CCD array system. Many shots were acquired at various positions downstream from a +2 m lens (more than 1000 at each position). For each shot, the Spiricon system records the $\frac{1}{e^2}$ full widths (the result of a Gaussian fit) of both the horizontal and vertical projections of the beam cross section. These widths are averaged (first horizontal and vertical, then to simplify the analysis these two are averaged together) over each set to give a single width at each beam position.

A Gaussian beam profile is manually fit to these data. The two free parameters used in the fit were waist position and far field divergence. The equation for the $\frac{1}{e^2}$ envelope of a Gaussian beam is \cite{Saleh:1991a}
%----------------------------------------------------------------------------
\begin{equation}
w(z)
=
w_0
\sqrt{
1+
\frac
{z-z_0}
{z_R}^2
},
\end{equation}
%----------------------------------------------------------------------------
where
%----------------------------------------------------------------------------
\begin{equation}
w_0
=
\sqrt{
\frac
{\lambda z_R}
{\pi n}
},
\end{equation}
%----------------------------------------------------------------------------
and $z_0$ is the position of the waist, $z_R$ is the Rayleigh parameter, $\lambda$ is the wavelength, and $n$ is the index of refraction. The Gaussian beam is propagated by keeping track of the complex source point, $q$. The real part of $q$ is the distance the waist lags behind the origin and the imaginary part of $q$ is the Rayleigh parameter. In free space, $q$ propagates according to
%----------------------------------------------------------------------------
\begin{equation}
q^\prime
=
q
+L
\label{free space}
\end{equation}
%----------------------------------------------------------------------------
where $L$ is the distance propagated. When transmitting through a thin lens, the $q$ parameter obeys \cite{Saleh:1991a}
%----------------------------------------------------------------------------
\begin{equation}
q^\prime
=
\frac{q}
{\frac{-q}{f}+1}
\label{thin lens}
\end{equation}
%----------------------------------------------------------------------------
where $f$ is the focal length of the lens. The resulting hand fit implies the dye laser beam waist is 53 inches behind the output surface with $w_0=1.4$ mm when $\lambda=628$ nm ($q=1.346+9.808i$ in SI units). The Spiricon data imply the raw beam has a waist radius of $\omega_o=439$ $\mu$m, with a far field divergence of 747 $\mu$rad (half the total divergence of the beam). Since \cite{Saleh:1991a}
%----------------------------------------------------------------------------
%----------------------------------------------------------------------------
\begin{equation}
M^2
=
\frac
{\theta\pi D}
{\lambda},
\end{equation}
%----------------------------------------------------------------------------
where $\theta$ is the far field divergence (half total divergence) and $D$ is the waist (half width) of the raw laser beam; our estimate of $M^2$ is 1.6. See figure \ref{M_squared}. One typical interpretation of $M^2$ is that it represents the number of modes present in the beam(technically in the x and y directions separately, but we have averaged the two together to simplify the results) so 1.6 may seem to imply near single mode operation with Gaussian transverse profiles; however, as we will see in Section \ref{HG section} this is not the case.
%----------------------------------------------------------------------------
\input{figures/diagnostics/M_squared/M_squared.tex}
%----------------------------------------------------------------------------
%----------------------------------------------------------------------------
%----------------------------------------------------------------------------
%----------------------------------------------------------------------------
%----------------------------------------------------------------------------

%----------------------------------------------------------------------------
%----------------------------------------------------------------------------
%----------------------------------------------------------------------------
%----------------------------------------------------------------------------
%----------------------------------------------------------------------------

%----------------------------------------------------------------------------
%----------------------------------------------------------------------------
%----------------------------------------------------------------------------
%----------------------------------------------------------------------------
%----------------------------------------------------------------------------

%----------------------------------------------------------------------------
%----------------------------------------------------------------------------
%----------------------------------------------------------------------------
%----------------------------------------------------------------------------
%----------------------------------------------------------------------------
