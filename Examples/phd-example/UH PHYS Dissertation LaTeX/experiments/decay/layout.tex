%------------------------------------------------------------------------------
%------------------------------------------------------------------------------
The beam line used here is similar to the one described in Section \ref{sample layout} except the attenuators are not used and in its place we use a pair of crossed polarizers (polarizing cube beam splitters) and a Pockels cell. The system has an extinction ratio of about 200:1 (see UH notebook UH-018 page 65) and produces a pulse with a FWHM of about 4 ns. Before the Pockels cell we have about 5.4 mJ, thus after the Pockels cell (heading toward the +750 mm lens) is about 27 uJ. The dye laser is tuned to 535.893 nm (a targeted absorption line) and the monochromator is tuned to capture a single LIF line at 632.9 nm. Again the signal from the PMT at the output of the monochromator is scanned (temporal gate scan) and averaged with a boxcar averager.

The measurement has facilitated the development of the data acquisition system required for lifetime measurements. The boxcar averager works at the 1000 ps gate width setting (it has settings down to 100 ps, but these have not been tried) and the LabView program records the data as a convenient text file. We have also discovered that the intensity fluctuations of the dye laser output introduce unwanted jitter in the position of the trigger relative to the peak -- accurate lifetime measurements will require a way to eliminate the jitter or, better yet, reduce the intensity fluctuations in the input beam. Contamination of the cell may have introduced additional channels through which the iodine can decay. A clean loading method must be developed in order to isolate and measure specific effects on the fluorescence decay.
%------------------------------------------------------------------------------
%------------------------------------------------------------------------------
%------------------------------------------------------------------------------
%------------------------------------------------------------------------------
%------------------------------------------------------------------------------
%------------------------------------------------------------------------------
