%----------------------------------------------------------------------------
We examine a STIRAP sequence followed by pi pulse. The first two pulses are a STIRAP sequence with $t_\beta=10$, $t_\alpha=t_\beta - \Delta_\alpha = 10 + 1.26997923288$ and $A=B=10.4616402919$. The last pulse is a $\pi$ pulse with $\sigma_\gamma=2$, $t_\gamma=25$, and $C=0.737832313319$. For each run the amplitudes are selected in a uniform random fashion on the interval $[-20\%,+20\%]$ using the ``runif'' random number generator in MathCAD. Then a fourth-order Runge-Kutta fixed-step method is used to find the solution at 5001 points in the interval, and hence the residue ($\Psi_{residue}$) for each run. See figure \ref{STIRAP plus pi pulse}.
%----------------------------------------------------------------------------
%----------------------------------------------------------------------------
%----------------------------------------------------------------------------
We examine a STIRAP sequence followed by pi pulse. The first two pulses are a STIRAP sequence with $t_\beta=10$, $t_\alpha=t_\beta - \Delta_\alpha = 10 + 1.26997923288$ and $A=B=10.4616402919$. The last pulse is a $\pi$ pulse with $\sigma_\gamma=2$, $t_\gamma=25$, and $C=0.737832313319$. For each run the amplitudes are selected in a uniform random fashion on the interval $[-20\%,+20\%]$ using the ``runif'' random number generator in MathCAD. Then a fourth-order Runge-Kutta fixed-step method is used to find the solution at 5001 points in the interval, and hence the residue ($\Psi_{residue}$) for each run. See figure \ref{STIRAP plus pi pulse}.
%----------------------------------------------------------------------------
%----------------------------------------------------------------------------
%----------------------------------------------------------------------------
We examine a STIRAP sequence followed by pi pulse. The first two pulses are a STIRAP sequence with $t_\beta=10$, $t_\alpha=t_\beta - \Delta_\alpha = 10 + 1.26997923288$ and $A=B=10.4616402919$. The last pulse is a $\pi$ pulse with $\sigma_\gamma=2$, $t_\gamma=25$, and $C=0.737832313319$. For each run the amplitudes are selected in a uniform random fashion on the interval $[-20\%,+20\%]$ using the ``runif'' random number generator in MathCAD. Then a fourth-order Runge-Kutta fixed-step method is used to find the solution at 5001 points in the interval, and hence the residue ($\Psi_{residue}$) for each run. See figure \ref{STIRAP plus pi pulse}.
%----------------------------------------------------------------------------
%----------------------------------------------------------------------------
%----------------------------------------------------------------------------
We examine a STIRAP sequence followed by pi pulse. The first two pulses are a STIRAP sequence with $t_\beta=10$, $t_\alpha=t_\beta - \Delta_\alpha = 10 + 1.26997923288$ and $A=B=10.4616402919$. The last pulse is a $\pi$ pulse with $\sigma_\gamma=2$, $t_\gamma=25$, and $C=0.737832313319$. For each run the amplitudes are selected in a uniform random fashion on the interval $[-20\%,+20\%]$ using the ``runif'' random number generator in MathCAD. Then a fourth-order Runge-Kutta fixed-step method is used to find the solution at 5001 points in the interval, and hence the residue ($\Psi_{residue}$) for each run. See figure \ref{STIRAP plus pi pulse}.
%----------------------------------------------------------------------------
%----------------------------------------------------------------------------
\input{figures/transfer/STIRAP_pi/STIRAP_pi.tex}
%----------------------------------------------------------------------------
%----------------------------------------------------------------------------
%----------------------------------------------------------------------------
%----------------------------------------------------------------------------
%----------------------------------------------------------------------------
%----------------------------------------------------------------------------

%----------------------------------------------------------------------------
%----------------------------------------------------------------------------
%----------------------------------------------------------------------------
%----------------------------------------------------------------------------
%----------------------------------------------------------------------------
%----------------------------------------------------------------------------

%----------------------------------------------------------------------------
%----------------------------------------------------------------------------
%----------------------------------------------------------------------------
%----------------------------------------------------------------------------
%----------------------------------------------------------------------------
%----------------------------------------------------------------------------

%----------------------------------------------------------------------------
%----------------------------------------------------------------------------
%----------------------------------------------------------------------------
%----------------------------------------------------------------------------
%----------------------------------------------------------------------------
%----------------------------------------------------------------------------
