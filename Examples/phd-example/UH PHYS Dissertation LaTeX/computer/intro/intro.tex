%----------------------------------------------------------------------------
%------------------------------Broad objectives------------------------------
%----------------------------------------------------------------------------
In this chapter, the basic equations of motion for multi-color population transfer are explored using a numerical integrator.
%----------------------------------------------------------------------------
%-------------------------concepts/results presented-------------------------
%----------------------------------------------------------------------------
Specific computer model parameters are calculated such that well known efficient multi-color pulse schemes can be simulated. To analyze robustness of various coherent processes with respect to amplitude fluctuations in the laser source, three different three-color pulse schemes are compared in a stochastic simulation. Because our target molecular system is relatively uncontrolled (trace molecular species in the atmosphere), we must consider the interaction between the target species and the environment (such systems are called ``open'' quantum systems \cite{Blum:1981a}). The de-coherence effect of collisions is studied using a simple model and the resulting motion is fit to a phenomenologically derived equation of motion using density matrix techniques in Liouville space.
%----------------------------------------------------------------------------
%------------------relevant concepts/results NOT presented-------------------
%----------------------------------------------------------------------------

The ``open'' nature of the problem is treated in an extremely simplified fashion. The target levels (four levels in the case of three-color excitation) are assumed to be the only states that will produce observable effects. It is known that in molecular systems, one typically must deal with a \emph{dense} energy level structure. To model an actual application more accurately, one must include some of the nearby energy levels in the observable system.

The only pulse schemes analyzed here are of the recently demonstrated STIRAP type (and, of course, the Gaussian $\pi$--pulse). With respect to the general detection problem at hand, there may be other pulse schemes with similar or better characteristics. For example, pulses with a temporal ``comb'' structure may be able to take advantage of the ``quantum zeno'' effect \cite{Itano:1990a}.
%----------------------------------------------------------------------------
%----------------------------------------------------------------------------
%----------------------------------------------------------------------------

See reference \cite{Vitanov:2001a} for a basic review of laser-induced adiabatic population transfer techniques. Reference \cite{Sola:1999a} reports the results of a numerical determination of the pulse parameters for STIRAP type population transfer using multi-color pulse trains. In this work, we re-calculate the results for two and three-color pulse trains. We extend the study of these pulse schemes by exploring the pulse amplitude solution space near the two-color optimum and comparing the different schemes when subject to randomized pulse amplitudes.

Reference \cite{Siegman:1986a} suggests a model for the quantum dynamical effects of collisions. This model is stochastically applied to a resonant three-color excitation of a four level system using a computer program. To model the resulting ``relaxation'' from the stochastic model, we use density matrix techniques \cite{Fano:1957a,Schirmer:2000a,Khaneja:2003a}. A relaxation matrix is written down, inspired by techniques reported in references \cite{Arimondo:1996a} and \cite{Kelley:1994a} but generated phenomenologically, and fit to the average behavior of the stochastic model.
%----------------------------------------------------------------------------
%----------------------------------------------------------------------------
