%----------------------------------------------------------------------------
\label{density formalism}
%----------------------------------------------------------------------------
The equation of motion for the density matrix is given by the Liouville-von Neumann equation:
\begin{equation}
i \hbar \frac{\partial \rho}{\partial t}
=
[H, \rho];
\label{Liouville equation}
\end{equation}
%----------------------------------------------------------------------------
where $H$ is the Hamiltonian and $\rho$ is the density matrix. If the system of interest is not totally isolated and the degrees of freedom due to its surroundings (the so called ``heat bath'') are unobserved, then equation (\ref{Liouville equation}) may not describe the time evolution of the system \cite{Blum:1981a}. In our application, we have a target molecule (system of interest) in a ``sea'' of non--target atmospheric molecules (the ``bath'').

We introduce a \emph{relaxation} operator $\hat{R}$ \cite{Band:1991a} to parameterize the effect of the surroundings:
%----------------------------------------------------------------------------
\begin{equation}
i \frac{\partial \rho}{\partial t}
=
[H, \rho]
+
i \hat{R} \vec{\rho}.
\label{open Liouville equation}
\end{equation}
%----------------------------------------------------------------------------
If we are working in $n$ dimensional Hilbert space, the left hand side and the first two terms on the right hand side are $n\cross n$ matrices. Since the last term is in Liouville space (i.e. $\vec{\rho} = (\rho_{00}, \cdots, \rho_{nn})^{T}$ and $\hat{R}$ is a $n^2\cross n^2$ matrix) there remains a one to one correspondence between the elements of each term in the equation of motion.

Let us consider the example of a many--level system (perhaps a target molecule with its many ro-vibrational levels) where we are only concerned with a small subset of these levels -- specifically two levels (perhaps the two levels resonantly coupled with a tuned laser field). The Hamiltonian from Equation \ref{two dynamics} is:
%----------------------------------------------------------------------------
\begin{equation}
H
=
i\left(
\begin{array}{cc}
0 & \alpha \\
-\alpha & 0
\end{array}
\right).
\label{Hilbert}
\end{equation}
%----------------------------------------------------------------------------
Next we give the relaxation matrix a specific form.
%----------------------------------------------------------------------------
\begin{equation}
\hat{R}
=
\hat{R}^{ext}
+
\hat{R}^{int}
+
\hat{R}^{phase};
\label{R}
\end{equation}
%----------------------------------------------------------------------------
where $\hat{R}^{ext}$ represents the tendency of the two level subset to relax into the other levels outside the subset (external relaxation), $\hat{R}^{int}$ represents the tendency of one level in the subset to relax into another level in the subset (internal relaxation), and $\hat{R}^{phase}$ represents de--phasing resulting from the relaxation process mentioned above and/or from other sources (perhaps collisions).
%----------------------------------------------------------------------------
\begin{equation}
\hat{R}^{ext}
=
\left(
\begin{array}{cccc}
-\Gamma_{00} & 0 & 0 & 0 \\
0 & 0 & 0 & 0 \\
0 & 0 & 0 & 0 \\
0 & 0 & 0 & -\Gamma_{11} 
\end{array}
\right),
\label{decay}
\end{equation}
%----------------------------------------------------------------------------
\begin{equation}
\hat{R}^{int}
=
\left(
\begin{array}{cccc}
-\Gamma^{0}_{1} & 0 & 0 & \Gamma^{0}_{1}  \\
0 & 0 & 0 & 0 \\
0 & 0 & 0 & 0 \\
\Gamma^{1}_{0} & 0 & 0 & -\Gamma^{1}_{0} 
\end{array}
\right), \quad \mbox{and}
\label{exchange}
\end{equation}
%----------------------------------------------------------------------------
\begin{equation}
\hat{R}^{phase}
=
\left(
\begin{array}{cccc}
0 & 0 & 0 & 0 \\
0 & \gamma_{01} & 0 & 0 \\
0 & 0 & \gamma_{01} & 0 \\
0 & 0 & 0 & 0 
\end{array}
\right);
\label{dephase}
\end{equation}
%----------------------------------------------------------------------------
where $\Gamma_{ii}$ is the relaxation rate of the state $i$ to some external state, $\Gamma^i_j$ is the relaxation rate of state $i$ into state $j$, and $\gamma_{ij}$ is the relaxation rate of $\rho_{ij}$ $(i \not= j)$. $\Gamma_{ii}$ is the rate at state $i$ relaxes to some other level \emph{outside} the subset under consideration; hence if $\Gamma_{ii}\not=0$ then probability will not be conserved in equation (\ref{open Liouville equation}). $\gamma_{ij}$ is applied to only the off diagonal terms in the density matrix; thus it is the rate at which the system loses coherence.
%----------------------------------------------------------------------------
%----------------------------------------------------------------------------
%----------------------------------------------------------------------------
%----------------------------------------------------------------------------
%----------------------------------------------------------------------------
%----------------------------------------------------------------------------
%----------------------------------------------------------------------------
%----------------------------------------------------------------------------
%----------------------------------------------------------------------------
%----------------------------------------------------------------------------

Now suppose our subsystem has three levels (see Figure \ref{2 color ladder}). The Hamiltonian from Equation \ref{three dynamics} is:
%----------------------------------------------------------------------------
\begin{equation}
H
=
i\left(
\begin{array}{ccc}
0 & \alpha & 0 \\
-\alpha & 0 & \beta \\
0 & -\beta & 0
\end{array}
\right).
\label{Hilbert3}
\end{equation}
%----------------------------------------------------------------------------
Using the same definitions as in the two level case, we can write the relaxation matrix terms as
%----------------------------------------------------------------------------
\begin{equation}
\hat{R}^{ext}
=
\left(
\begin{array}{ccccccccc}
-\Gamma_{00} & 0 & 0 & 0 & 0 & 0 & 0 & 0 & 0 \\
0 & 0 & 0 & 0 & 0 & 0 & 0 & 0 & 0 \\
0 & 0 & 0 & 0 & 0 & 0 & 0 & 0 & 0 \\
0 & 0 & 0 & 0 & 0 & 0 & 0 & 0 & 0 \\
0 & 0 & 0 & 0 & -\Gamma_{11} & 0 & 0 & 0 & 0 \\
0 & 0 & 0 & 0 & 0 & 0 & 0 & 0 & 0 \\
0 & 0 & 0 & 0 & 0 & 0 & 0 & 0 & 0 \\
0 & 0 & 0 & 0 & 0 & 0 & 0 & 0 & 0 \\
0 & 0 & 0 & 0 & 0 & 0 & 0 & 0 & -\Gamma_{22}
\end{array}
\right),
\label{decay3}
\end{equation}
%----------------------------------------------------------------------------
%----------------------------------------------------------------------------
\begin{equation}
\hat{R}^{int}
=
\left(
\begin{array}{ccccccccc}
-\Gamma^{0}_{1}-\Gamma^{0}_{2} & 0 & 0 & 0 & \Gamma^{0}_{1} & 0 & 0 & 0 & \Gamma^{0}_{2} \\
0 & 0 & 0 & 0 & 0 & 0 & 0 & 0 & 0 \\
0 & 0 & 0 & 0 & 0 & 0 & 0 & 0 & 0 \\
0 & 0 & 0 & 0 & 0 & 0 & 0 & 0 & 0 \\
\Gamma^{1}_{0} & 0 & 0 & 0 & -\Gamma^{1}_{0}-\Gamma^{1}_{2} & 0 & 0 & 0 & \Gamma^{1}_{2} \\
0 & 0 & 0 & 0 & 0 & 0 & 0 & 0 & 0 \\
0 & 0 & 0 & 0 & 0 & 0 & 0 & 0 & 0 \\
0 & 0 & 0 & 0 & 0 & 0 & 0 & 0 & 0 \\
\Gamma^{2}_{0} & 0 & 0 & 0 & \Gamma^{2}_{1} & 0 & 0 & 0 & -\Gamma^{2}_{0}-\Gamma^{2}_{1}
\end{array}
\right),
\label{exchange3}
\end{equation}
%----------------------------------------------------------------------------
and
%----------------------------------------------------------------------------
\begin{equation}
\hat{R}^{phase}
=
\left(
\begin{array}{ccccccccc}
0 & 0 & 0 & 0 & 0 & 0 & 0 & 0 & 0 \\
0 & \gamma_{01} & 0 & 0 & 0 & 0 & 0 & 0 & 0 \\
0 & 0 & \gamma_{02} & 0 & 0 & 0 & 0 & 0 & 0 \\
0 & 0 & 0 & \gamma_{01} & 0 & 0 & 0 & 0 & 0 \\
0 & 0 & 0 & 0 & 0 & 0 & 0 & 0 & 0 \\
0 & 0 & 0 & 0 & 0 & \gamma_{12} & 0 & 0 & 0 \\
0 & 0 & 0 & 0 & 0 & 0 & \gamma_{02} & 0 & 0 \\
0 & 0 & 0 & 0 & 0 & 0 & 0 & \gamma_{12} & 0 \\
0 & 0 & 0 & 0 & 0 & 0 & 0 & 0 & 0 
\end{array}
\right).
\label{dephase3}
\end{equation}
%----------------------------------------------------------------------------
%----------------------------------------------------------------------------
%----------------------------------------------------------------------------
%----------------------------------------------------------------------------
