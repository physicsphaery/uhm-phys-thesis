%----------------------------------------------------------------------------
\label{computer chapter}
%----------------------------------------------------------------------------
%----------------------------------------------------------------------------
%------------------------------Broad objectives------------------------------
%----------------------------------------------------------------------------
In this chapter, the basic equations of motion for multi-color population transfer are explored using a numerical integrator.
%----------------------------------------------------------------------------
%-------------------------concepts/results presented-------------------------
%----------------------------------------------------------------------------
Specific computer model parameters are calculated such that well known efficient multi-color pulse schemes can be simulated. To analyze robustness of various coherent processes with respect to amplitude fluctuations in the laser source, three different three-color pulse schemes are compared in a stochastic simulation. Because our target molecular system is relatively uncontrolled (trace molecular species in the atmosphere), we must consider the interaction between the target species and the environment (such systems are called ``open'' quantum systems \cite{Blum:1981a}). The de-coherence effect of collisions is studied using a simple model and the resulting motion is fit to a phenomenologically derived equation of motion using density matrix techniques in Liouville space.
%----------------------------------------------------------------------------
%------------------relevant concepts/results NOT presented-------------------
%----------------------------------------------------------------------------

The ``open'' nature of the problem is treated in an extremely simplified fashion. The target levels (four levels in the case of three-color excitation) are assumed to be the only states that will produce observable effects. It is known that in molecular systems, one typically must deal with a \emph{dense} energy level structure. To model an actual application more accurately, one must include some of the nearby energy levels in the observable system.

The only pulse schemes analyzed here are of the recently demonstrated STIRAP type (and, of course, the Gaussian $\pi$--pulse). With respect to the general detection problem at hand, there may be other pulse schemes with similar or better characteristics. For example, pulses with a temporal ``comb'' structure may be able to take advantage of the ``quantum zeno'' effect \cite{Itano:1990a}.
%----------------------------------------------------------------------------
%----------------------------------------------------------------------------
%----------------------------------------------------------------------------

See reference \cite{Vitanov:2001a} for a basic review of laser-induced adiabatic population transfer techniques. Reference \cite{Sola:1999a} reports the results of a numerical determination of the pulse parameters for STIRAP type population transfer using multi-color pulse trains. In this work, we re-calculate the results for two and three-color pulse trains. We extend the study of these pulse schemes by exploring the pulse amplitude solution space near the two-color optimum and comparing the different schemes when subject to randomized pulse amplitudes.

Reference \cite{Siegman:1986a} suggests a model for the quantum dynamical effects of collisions. This model is stochastically applied to a resonant three-color excitation of a four level system using a computer program. To model the resulting ``relaxation'' from the stochastic model, we use density matrix techniques \cite{Fano:1957a,Schirmer:2000a,Khaneja:2003a}. A relaxation matrix is written down, inspired by techniques reported in references \cite{Arimondo:1996a} and \cite{Kelley:1994a} but generated phenomenologically, and fit to the average behavior of the stochastic model.
%----------------------------------------------------------------------------
%----------------------------------------------------------------------------

%----------------------------------------------------------------------------
%----------------------------------------------------------------------------
\section{Transform limited pulse sequences}
%----------------------------------------------------------------------------
%----------------------------------------------------------------------------
\subsection{One color}
%----------------------------------------------------------------------------
%----------------------------------------------------------------------------
%----------------------------------------------------------------------------
Suppose the dynamics of some four level quantum system is given by
%----------------------------------------------------------------------------
%----------------------------------------------------------------------------
\begin{equation}
i\frac{\partial}{\partial t} \ket{\Psi}
=
i\left(
\begin{array}{cccc}
0 & \alpha & 0 & 0 \\
-\alpha & 0 & \beta & 0 \\
0 & -\beta & 0 & \gamma \\
0 & 0 & -\gamma & 0 
\end{array}
\right)
\ket{\Psi},
\label{three dynamics}
\end{equation}
%----------------------------------------------------------------------------
where the square of the nth element in $\Psi$ is the probability of finding the system in the nth state, $\alpha$ is the coupling field between the zeroth (ground) and first state, $\beta$ is the coupling field between the first and second state, and $\gamma$ is the coupling field between the second and third state. Using the same methods as in Section \ref{basic_two_level}, it can be shown that closed non-degenerate four level quantum systems with three completely resonant ``ladder'' coupling fields can be described this way in the energy basis (see figure \ref{three color ladder}).
%----------------------------------------------------------------------------
%----------------------------------------------------------------------------
\input{figures/transfer/3_color_ladder/3_color_ladder.tex}
%----------------------------------------------------------------------------
%----------------------------------------------------------------------------
%----------------------------------------------------------------------------
%----------------------------------------------------------------------------
Again we consider temporally localized coupling fields of the form (\ref{gaussian}). Let the positions of $\alpha$ and $\gamma$ be given in terms of $\beta$'s position; namely let $\Delta_\alpha$ be the temporal distance alpha \emph{leads} beta and $\Delta_\gamma$ be the temporal distance gamma \emph{lags} beta; specifically
%----------------------------------------------------------------------------
\begin{subequations}
\begin{eqnarray} 
t_\alpha & = & t_0 - \Delta_\alpha \\
t_\beta  & = & t_0 \\
t_\gamma  & = & t_0 + \Delta_\gamma
\end{eqnarray}
\label{xxx}
\end{subequations}
%----------------------------------------------------------------------------
where $t_0$ is the middle of the pulse sequence (and the position of $\beta$). Thus, for a sequential order as shown in Figure \ref{three pulses}, both $\Delta_\alpha$ and $\Delta_\gamma$ are positive.
%----------------------------------------------------------------------------
%----------------------------------------------------------------------------
% 3_pulses.tex
% by Troy Hix, March 2005
%----------------------------------------------------------------------------
\begin{figure}
\centering
\includegraphics[width=5.00in]
{3_pulses/3_pulses.png}\\
\caption[Three pulse sequence example]{Three pulse sequence example. In general the pulses $\alpha$, $\beta$, and $\gamma$ have arbitrary widths $\sigma_\alpha$, $\sigma_\beta$, and $\sigma_\gamma$ as well as arbitrary amplitudes $A$, $B$, and $C$.}
\label{three pulses}
\end{figure} 
%----------------------------------------------------------------------------

%----------------------------------------------------------------------------
%----------------------------------------------------------------------------
%----------------------------------------------------------------------------

%----------------------------------------------------------------------------
%----------------------------------------------------------------------------
%----------------------------------------------------------------------------
%----------------------------------------------------------------------------
%----------------------------------------------------------------------------
Given an arbitrary pulse width of $\sigma_{\alpha}=2$, we seek to find the pulse height $A$ that corresponds to a complete inversion (gound state complete depleted). Suppose the solution, $\ket{\Psi}$, to (\ref{one dynamics}) is known. Consider the cost function
%----------------------------------------------------------------------------
\begin{equation}
\Phi
=
P_0(t=t_{final})
\label{one cost}
\end{equation}
%----------------------------------------------------------------------------
where $P_0$ be the probability that the system is found in the ground state (i.e. $\braket{0}{0}$); thus, if $P_0=0$ then the two level system is completely inverted.
%----------------------------------------------------------------------------
%----------------------------------------------------------------------------
A MathCAD program is used to minimize (\ref{one cost}) as a function of $A$ for a system initially in the ground state over the time interval $[0,30]$ (thus Equation \ref{one cost} is evaluated at $t=30$). A fourth-order Runge-Kutta fixed-step method is used to find the solution at 5000 points in the time interval. The MathCAD function ``Minimize'' is used to find the optimal solution. The ``Minimize'' function tries the conjugate gradient, quasi-Newton, and Levenberg-Marquardt methods in succession until one converges.

Figure \ref{solution one} shows the pulse and resulting motion in $\braket{\Phi}{\Phi}$ at a local optimum. There are many local optima as one increases $A$. The nature of these repeated optima becomes obvious when one considers the analytic solution (they are sucessive Rabi oscillations), Equation \ref{2 level dynamics}; in lieu of such a discussion we present an example (see figure \ref{big solution one}).
%----------------------------------------------------------------------------
%----------------------------------------------------------------------------
% solution_1.tex
% by Troy Hix, April 2005
%----------------------------------------------------------------------------
\begin{figure}
\centering
\includegraphics[width=5.00in]
{solution_1/solution_1.png}\\
\caption[Single color optimal solution]{Single color optimal solution. With $\sigma_{\alpha}\equiv 2$, this local optimum occurs when $A=0.737832313319$. Many more local optima occur with increasing $A$. If the pulse area is $\xi$, then $\xi - \pi/2 \sim 10^{-12}$ (the precision of $A$). In the literature, this is called a $\pi$--pulse.}
\label{solution one}
\end{figure} 
%----------------------------------------------------------------------------

%----------------------------------------------------------------------------
%----------------------------------------------------------------------------
% big_solution_1.tex
% by Troy Hix, April 2005
%----------------------------------------------------------------------------
\begin{figure}
\centering
\includegraphics[width=5.00in]
{big_solution_1/big_solution_1.png}\\
\caption[Single color optimal solution - increased pulse amplitude]{Single color optimal solution - increased pulse amplitude. This local optimum occurs when $A=3\cross0.737832313319$ (three times the area of the $\pi$--pulse).}
\label{big solution one}
\end{figure} 
%----------------------------------------------------------------------------

%----------------------------------------------------------------------------
%----------------------------------------------------------------------------
%----------------------------------------------------------------------------
%----------------------------------------------------------------------------

%----------------------------------------------------------------------------
%----------------------------------------------------------------------------
\subsection{Two color}
%----------------------------------------------------------------------------
%----------------------------------------------------------------------------
%----------------------------------------------------------------------------
Suppose the dynamics of some four level quantum system is given by
%----------------------------------------------------------------------------
%----------------------------------------------------------------------------
\begin{equation}
i\frac{\partial}{\partial t} \ket{\Psi}
=
i\left(
\begin{array}{cccc}
0 & \alpha & 0 & 0 \\
-\alpha & 0 & \beta & 0 \\
0 & -\beta & 0 & \gamma \\
0 & 0 & -\gamma & 0 
\end{array}
\right)
\ket{\Psi},
\label{three dynamics}
\end{equation}
%----------------------------------------------------------------------------
where the square of the nth element in $\Psi$ is the probability of finding the system in the nth state, $\alpha$ is the coupling field between the zeroth (ground) and first state, $\beta$ is the coupling field between the first and second state, and $\gamma$ is the coupling field between the second and third state. Using the same methods as in Section \ref{basic_two_level}, it can be shown that closed non-degenerate four level quantum systems with three completely resonant ``ladder'' coupling fields can be described this way in the energy basis (see figure \ref{three color ladder}).
%----------------------------------------------------------------------------
%----------------------------------------------------------------------------
\input{figures/transfer/3_color_ladder/3_color_ladder.tex}
%----------------------------------------------------------------------------
%----------------------------------------------------------------------------
%----------------------------------------------------------------------------
%----------------------------------------------------------------------------
Again we consider temporally localized coupling fields of the form (\ref{gaussian}). Let the positions of $\alpha$ and $\gamma$ be given in terms of $\beta$'s position; namely let $\Delta_\alpha$ be the temporal distance alpha \emph{leads} beta and $\Delta_\gamma$ be the temporal distance gamma \emph{lags} beta; specifically
%----------------------------------------------------------------------------
\begin{subequations}
\begin{eqnarray} 
t_\alpha & = & t_0 - \Delta_\alpha \\
t_\beta  & = & t_0 \\
t_\gamma  & = & t_0 + \Delta_\gamma
\end{eqnarray}
\label{xxx}
\end{subequations}
%----------------------------------------------------------------------------
where $t_0$ is the middle of the pulse sequence (and the position of $\beta$). Thus, for a sequential order as shown in Figure \ref{three pulses}, both $\Delta_\alpha$ and $\Delta_\gamma$ are positive.
%----------------------------------------------------------------------------
%----------------------------------------------------------------------------
% 3_pulses.tex
% by Troy Hix, March 2005
%----------------------------------------------------------------------------
\begin{figure}
\centering
\includegraphics[width=5.00in]
{3_pulses/3_pulses.png}\\
\caption[Three pulse sequence example]{Three pulse sequence example. In general the pulses $\alpha$, $\beta$, and $\gamma$ have arbitrary widths $\sigma_\alpha$, $\sigma_\beta$, and $\sigma_\gamma$ as well as arbitrary amplitudes $A$, $B$, and $C$.}
\label{three pulses}
\end{figure} 
%----------------------------------------------------------------------------

%----------------------------------------------------------------------------
%----------------------------------------------------------------------------
%----------------------------------------------------------------------------

%----------------------------------------------------------------------------
%----------------------------------------------------------------------------
%----------------------------------------------------------------------------
%----------------------------------------------------------------------------
%----------------------------------------------------------------------------
Given an arbitrary pulse width of $\sigma_{\alpha}=2$, we seek to find the pulse height $A$ that corresponds to a complete inversion (gound state complete depleted). Suppose the solution, $\ket{\Psi}$, to (\ref{one dynamics}) is known. Consider the cost function
%----------------------------------------------------------------------------
\begin{equation}
\Phi
=
P_0(t=t_{final})
\label{one cost}
\end{equation}
%----------------------------------------------------------------------------
where $P_0$ be the probability that the system is found in the ground state (i.e. $\braket{0}{0}$); thus, if $P_0=0$ then the two level system is completely inverted.
%----------------------------------------------------------------------------
%----------------------------------------------------------------------------
A MathCAD program is used to minimize (\ref{one cost}) as a function of $A$ for a system initially in the ground state over the time interval $[0,30]$ (thus Equation \ref{one cost} is evaluated at $t=30$). A fourth-order Runge-Kutta fixed-step method is used to find the solution at 5000 points in the time interval. The MathCAD function ``Minimize'' is used to find the optimal solution. The ``Minimize'' function tries the conjugate gradient, quasi-Newton, and Levenberg-Marquardt methods in succession until one converges.

Figure \ref{solution one} shows the pulse and resulting motion in $\braket{\Phi}{\Phi}$ at a local optimum. There are many local optima as one increases $A$. The nature of these repeated optima becomes obvious when one considers the analytic solution (they are sucessive Rabi oscillations), Equation \ref{2 level dynamics}; in lieu of such a discussion we present an example (see figure \ref{big solution one}).
%----------------------------------------------------------------------------
%----------------------------------------------------------------------------
% solution_1.tex
% by Troy Hix, April 2005
%----------------------------------------------------------------------------
\begin{figure}
\centering
\includegraphics[width=5.00in]
{solution_1/solution_1.png}\\
\caption[Single color optimal solution]{Single color optimal solution. With $\sigma_{\alpha}\equiv 2$, this local optimum occurs when $A=0.737832313319$. Many more local optima occur with increasing $A$. If the pulse area is $\xi$, then $\xi - \pi/2 \sim 10^{-12}$ (the precision of $A$). In the literature, this is called a $\pi$--pulse.}
\label{solution one}
\end{figure} 
%----------------------------------------------------------------------------

%----------------------------------------------------------------------------
%----------------------------------------------------------------------------
% big_solution_1.tex
% by Troy Hix, April 2005
%----------------------------------------------------------------------------
\begin{figure}
\centering
\includegraphics[width=5.00in]
{big_solution_1/big_solution_1.png}\\
\caption[Single color optimal solution - increased pulse amplitude]{Single color optimal solution - increased pulse amplitude. This local optimum occurs when $A=3\cross0.737832313319$ (three times the area of the $\pi$--pulse).}
\label{big solution one}
\end{figure} 
%----------------------------------------------------------------------------

%----------------------------------------------------------------------------
%----------------------------------------------------------------------------
%----------------------------------------------------------------------------
%----------------------------------------------------------------------------

%----------------------------------------------------------------------------

For the optimal solution, $\Phi_{residue}=9\cross10^{-15}$, we map out the surface this variable defines in solution space using two coordinate systems.
%----------------------------------------------------------------------------
%----------------------------------------------------------------------------
The region in solution space near the optimal solution shown in Figure \ref{solution 2 pulses} was mapped out by allowing the magnitude of $\Delta_{\alpha}$ and $A$ to vary by $\pm 50\%$. Figure \ref{delay amp} shows the resulting $\log(\Psi_{residue})$ surface.
%----------------------------------------------------------------------------
%----------------------------------------------------------------------------
The region in solution space near the optimal solution shown in Figure \ref{solution 2 pulses} was mapped out by allowing the magnitude of $A$ and $B$ to vary by $\pm 50\%$. Figure \ref{amp amp} shows the resulting $\log(\Psi_{residue})$ surface. Figures \ref{delay amp} and \ref{amp amp} show the robustness of the STIRAP process with regards to population transfer. The since the residue is less than $10^{-2}$ for most of the points (i.e. there is very little population left over in states 0 and 1) population transfer in nearly complete for relatively large detuning of certain pulse parameters (amplitude and delay). This means that even if there are fluctuations in these parameters, the targeted two color pathway will still result in nearly complete population transfer; thus the probability that the final target state releases a fluorescence photon is nearly independent of these types of fluctuations.
%----------------------------------------------------------------------------
%----------------------------------------------------------------------------
% delay_amp.tex
% by Troy Hix, March 2005
%----------------------------------------------------------------------------
\begin{figure}
\centering
\includegraphics[width=4.00in]
{delay_amp/delay_amp.png}\\
\caption[$\log\left(\Phi_{residue}\right)$ dependence on scaling $\Delta_{\alpha}$ and $A$]{$\log\left(\Phi_{residue}\right)$ dependence on scaling $\Delta_{\alpha}$ and $A$. There are many local optima arranged in a repeating crescent shape. In general the local optima and the average height of the nearby surface decreases as $\Delta_{\alpha}$ and $A$ increase. There are $41^2$ evenly spaced points here.}
\label{delay amp}
\end{figure} 
%----------------------------------------------------------------------------

%----------------------------------------------------------------------------
%----------------------------------------------------------------------------
\input{figures/transfer/amp_amp/amp_amp.tex}
%----------------------------------------------------------------------------
%----------------------------------------------------------------------------

%----------------------------------------------------------------------------
%----------------------------------------------------------------------------
\subsection{Three color}
%----------------------------------------------------------------------------
%----------------------------------------------------------------------------
%----------------------------------------------------------------------------
Suppose the dynamics of some four level quantum system is given by
%----------------------------------------------------------------------------
%----------------------------------------------------------------------------
\begin{equation}
i\frac{\partial}{\partial t} \ket{\Psi}
=
i\left(
\begin{array}{cccc}
0 & \alpha & 0 & 0 \\
-\alpha & 0 & \beta & 0 \\
0 & -\beta & 0 & \gamma \\
0 & 0 & -\gamma & 0 
\end{array}
\right)
\ket{\Psi},
\label{three dynamics}
\end{equation}
%----------------------------------------------------------------------------
where the square of the nth element in $\Psi$ is the probability of finding the system in the nth state, $\alpha$ is the coupling field between the zeroth (ground) and first state, $\beta$ is the coupling field between the first and second state, and $\gamma$ is the coupling field between the second and third state. Using the same methods as in Section \ref{basic_two_level}, it can be shown that closed non-degenerate four level quantum systems with three completely resonant ``ladder'' coupling fields can be described this way in the energy basis (see figure \ref{three color ladder}).
%----------------------------------------------------------------------------
%----------------------------------------------------------------------------
\input{figures/transfer/3_color_ladder/3_color_ladder.tex}
%----------------------------------------------------------------------------
%----------------------------------------------------------------------------
%----------------------------------------------------------------------------
%----------------------------------------------------------------------------
Again we consider temporally localized coupling fields of the form (\ref{gaussian}). Let the positions of $\alpha$ and $\gamma$ be given in terms of $\beta$'s position; namely let $\Delta_\alpha$ be the temporal distance alpha \emph{leads} beta and $\Delta_\gamma$ be the temporal distance gamma \emph{lags} beta; specifically
%----------------------------------------------------------------------------
\begin{subequations}
\begin{eqnarray} 
t_\alpha & = & t_0 - \Delta_\alpha \\
t_\beta  & = & t_0 \\
t_\gamma  & = & t_0 + \Delta_\gamma
\end{eqnarray}
\label{xxx}
\end{subequations}
%----------------------------------------------------------------------------
where $t_0$ is the middle of the pulse sequence (and the position of $\beta$). Thus, for a sequential order as shown in Figure \ref{three pulses}, both $\Delta_\alpha$ and $\Delta_\gamma$ are positive.
%----------------------------------------------------------------------------
%----------------------------------------------------------------------------
% 3_pulses.tex
% by Troy Hix, March 2005
%----------------------------------------------------------------------------
\begin{figure}
\centering
\includegraphics[width=5.00in]
{3_pulses/3_pulses.png}\\
\caption[Three pulse sequence example]{Three pulse sequence example. In general the pulses $\alpha$, $\beta$, and $\gamma$ have arbitrary widths $\sigma_\alpha$, $\sigma_\beta$, and $\sigma_\gamma$ as well as arbitrary amplitudes $A$, $B$, and $C$.}
\label{three pulses}
\end{figure} 
%----------------------------------------------------------------------------

%----------------------------------------------------------------------------
%----------------------------------------------------------------------------
%----------------------------------------------------------------------------

%----------------------------------------------------------------------------
%----------------------------------------------------------------------------
%----------------------------------------------------------------------------
%----------------------------------------------------------------------------
%----------------------------------------------------------------------------
Given an arbitrary pulse width of $\sigma_{\alpha}=2$, we seek to find the pulse height $A$ that corresponds to a complete inversion (gound state complete depleted). Suppose the solution, $\ket{\Psi}$, to (\ref{one dynamics}) is known. Consider the cost function
%----------------------------------------------------------------------------
\begin{equation}
\Phi
=
P_0(t=t_{final})
\label{one cost}
\end{equation}
%----------------------------------------------------------------------------
where $P_0$ be the probability that the system is found in the ground state (i.e. $\braket{0}{0}$); thus, if $P_0=0$ then the two level system is completely inverted.
%----------------------------------------------------------------------------
%----------------------------------------------------------------------------
A MathCAD program is used to minimize (\ref{one cost}) as a function of $A$ for a system initially in the ground state over the time interval $[0,30]$ (thus Equation \ref{one cost} is evaluated at $t=30$). A fourth-order Runge-Kutta fixed-step method is used to find the solution at 5000 points in the time interval. The MathCAD function ``Minimize'' is used to find the optimal solution. The ``Minimize'' function tries the conjugate gradient, quasi-Newton, and Levenberg-Marquardt methods in succession until one converges.

Figure \ref{solution one} shows the pulse and resulting motion in $\braket{\Phi}{\Phi}$ at a local optimum. There are many local optima as one increases $A$. The nature of these repeated optima becomes obvious when one considers the analytic solution (they are sucessive Rabi oscillations), Equation \ref{2 level dynamics}; in lieu of such a discussion we present an example (see figure \ref{big solution one}).
%----------------------------------------------------------------------------
%----------------------------------------------------------------------------
% solution_1.tex
% by Troy Hix, April 2005
%----------------------------------------------------------------------------
\begin{figure}
\centering
\includegraphics[width=5.00in]
{solution_1/solution_1.png}\\
\caption[Single color optimal solution]{Single color optimal solution. With $\sigma_{\alpha}\equiv 2$, this local optimum occurs when $A=0.737832313319$. Many more local optima occur with increasing $A$. If the pulse area is $\xi$, then $\xi - \pi/2 \sim 10^{-12}$ (the precision of $A$). In the literature, this is called a $\pi$--pulse.}
\label{solution one}
\end{figure} 
%----------------------------------------------------------------------------

%----------------------------------------------------------------------------
%----------------------------------------------------------------------------
% big_solution_1.tex
% by Troy Hix, April 2005
%----------------------------------------------------------------------------
\begin{figure}
\centering
\includegraphics[width=5.00in]
{big_solution_1/big_solution_1.png}\\
\caption[Single color optimal solution - increased pulse amplitude]{Single color optimal solution - increased pulse amplitude. This local optimum occurs when $A=3\cross0.737832313319$ (three times the area of the $\pi$--pulse).}
\label{big solution one}
\end{figure} 
%----------------------------------------------------------------------------

%----------------------------------------------------------------------------
%----------------------------------------------------------------------------
%----------------------------------------------------------------------------
%----------------------------------------------------------------------------

%----------------------------------------------------------------------------
%----------------------------------------------------------------------------
\section{Random excitation amplitude simulations}
It was observed in the laboratory that the dye lasers, intended for use in a demonstration experiment of these processes, exhibit relatively large ($\sim20\%$) intensity fluctuations. Here we estimate the effect of these fluctuations on the population transfer efficiency of various three color transfer schemes.
%----------------------------------------------------------------------------
\label{random sims}
%----------------------------------------------------------------------------
\subsection{Three $\pi$ pulse sequence}
%----------------------------------------------------------------------------
We examine a three $\pi$ pulse sequence. Each pulse has the temporal profile (\ref{gaussian}) with $\sigma_\alpha=\sigma_\beta=\sigma_\gamma=2$ and $A=B=C=0.737832313319$. The centers of each pulse were placed symmetrically in the interval $[0,30]$: $t_\alpha=5$, $t_\beta=15$, and $t_\gamma=25$. For each run the amplitudes are selected in a uniform random fashion on the interval $[-20\%,+20\%]$ using the ``runif'' random number generator in MathCAD. Then a fourth-order Runge-Kutta fixed-step method is used to find the solution at 5001 points in the interval, and hence the residue ($\Psi_{residue}$) for each run. See figure \ref{three pi}.
%----------------------------------------------------------------------------
%----------------------------------------------------------------------------
%----------------------------------------------------------------------------
We examine a three $\pi$ pulse sequence. Each pulse has the temporal profile (\ref{gaussian}) with $\sigma_\alpha=\sigma_\beta=\sigma_\gamma=2$ and $A=B=C=0.737832313319$. The centers of each pulse were placed symmetrically in the interval $[0,30]$: $t_\alpha=5$, $t_\beta=15$, and $t_\gamma=25$. For each run the amplitudes are selected in a uniform random fashion on the interval $[-20\%,+20\%]$ using the ``runif'' random number generator in MathCAD. Then a fourth-order Runge-Kutta fixed-step method is used to find the solution at 5001 points in the interval, and hence the residue ($\Psi_{residue}$) for each run. See figure \ref{three pi}.
%----------------------------------------------------------------------------
%----------------------------------------------------------------------------
%----------------------------------------------------------------------------
We examine a three $\pi$ pulse sequence. Each pulse has the temporal profile (\ref{gaussian}) with $\sigma_\alpha=\sigma_\beta=\sigma_\gamma=2$ and $A=B=C=0.737832313319$. The centers of each pulse were placed symmetrically in the interval $[0,30]$: $t_\alpha=5$, $t_\beta=15$, and $t_\gamma=25$. For each run the amplitudes are selected in a uniform random fashion on the interval $[-20\%,+20\%]$ using the ``runif'' random number generator in MathCAD. Then a fourth-order Runge-Kutta fixed-step method is used to find the solution at 5001 points in the interval, and hence the residue ($\Psi_{residue}$) for each run. See figure \ref{three pi}.
%----------------------------------------------------------------------------
%----------------------------------------------------------------------------
\input{figures/transfer/three_pi/three_pi.tex}
%----------------------------------------------------------------------------
%----------------------------------------------------------------------------
%----------------------------------------------------------------------------
%----------------------------------------------------------------------------
%----------------------------------------------------------------------------
%----------------------------------------------------------------------------

%----------------------------------------------------------------------------
%----------------------------------------------------------------------------
%----------------------------------------------------------------------------
%----------------------------------------------------------------------------
%----------------------------------------------------------------------------
%----------------------------------------------------------------------------

%----------------------------------------------------------------------------
%----------------------------------------------------------------------------
%----------------------------------------------------------------------------
%----------------------------------------------------------------------------
%----------------------------------------------------------------------------
%----------------------------------------------------------------------------

%----------------------------------------------------------------------------
\subsection{STIRAP sequence followed by a $\pi$ pulse}
%----------------------------------------------------------------------------
We examine a STIRAP sequence followed by pi pulse. The first two pulses are a STIRAP sequence with $t_\beta=10$, $t_\alpha=t_\beta - \Delta_\alpha = 10 + 1.26997923288$ and $A=B=10.4616402919$. The last pulse is a $\pi$ pulse with $\sigma_\gamma=2$, $t_\gamma=25$, and $C=0.737832313319$. For each run the amplitudes are selected in a uniform random fashion on the interval $[-20\%,+20\%]$ using the ``runif'' random number generator in MathCAD. Then a fourth-order Runge-Kutta fixed-step method is used to find the solution at 5001 points in the interval, and hence the residue ($\Psi_{residue}$) for each run. See figure \ref{STIRAP plus pi pulse}.
%----------------------------------------------------------------------------
%----------------------------------------------------------------------------
%----------------------------------------------------------------------------
We examine a STIRAP sequence followed by pi pulse. The first two pulses are a STIRAP sequence with $t_\beta=10$, $t_\alpha=t_\beta - \Delta_\alpha = 10 + 1.26997923288$ and $A=B=10.4616402919$. The last pulse is a $\pi$ pulse with $\sigma_\gamma=2$, $t_\gamma=25$, and $C=0.737832313319$. For each run the amplitudes are selected in a uniform random fashion on the interval $[-20\%,+20\%]$ using the ``runif'' random number generator in MathCAD. Then a fourth-order Runge-Kutta fixed-step method is used to find the solution at 5001 points in the interval, and hence the residue ($\Psi_{residue}$) for each run. See figure \ref{STIRAP plus pi pulse}.
%----------------------------------------------------------------------------
%----------------------------------------------------------------------------
%----------------------------------------------------------------------------
We examine a STIRAP sequence followed by pi pulse. The first two pulses are a STIRAP sequence with $t_\beta=10$, $t_\alpha=t_\beta - \Delta_\alpha = 10 + 1.26997923288$ and $A=B=10.4616402919$. The last pulse is a $\pi$ pulse with $\sigma_\gamma=2$, $t_\gamma=25$, and $C=0.737832313319$. For each run the amplitudes are selected in a uniform random fashion on the interval $[-20\%,+20\%]$ using the ``runif'' random number generator in MathCAD. Then a fourth-order Runge-Kutta fixed-step method is used to find the solution at 5001 points in the interval, and hence the residue ($\Psi_{residue}$) for each run. See figure \ref{STIRAP plus pi pulse}.
%----------------------------------------------------------------------------
%----------------------------------------------------------------------------
\input{figures/transfer/STIRAP_pi/STIRAP_pi.tex}
%----------------------------------------------------------------------------
%----------------------------------------------------------------------------
%----------------------------------------------------------------------------
%----------------------------------------------------------------------------
%----------------------------------------------------------------------------
%----------------------------------------------------------------------------

%----------------------------------------------------------------------------
%----------------------------------------------------------------------------
%----------------------------------------------------------------------------
%----------------------------------------------------------------------------
%----------------------------------------------------------------------------
%----------------------------------------------------------------------------

%----------------------------------------------------------------------------
%----------------------------------------------------------------------------
%----------------------------------------------------------------------------
%----------------------------------------------------------------------------
%----------------------------------------------------------------------------
%----------------------------------------------------------------------------

%----------------------------------------------------------------------------
\subsection{Three color optimal sequence}
% three_STIRAP.tex
% by Troy Hix, April 2005
%----------------------------------------------------------------------------
\begin{figure}
\includegraphics[width=6.00in]
{three_STIRAP/three_STIRAP.png}\\
\caption[Residue for runs using three color STIRAP]{Residue for runs using three color STIRAP. A pulse sequence of the form show in figure \ref{solution three pulses} is used here. The pulse amplitudes $A$, $B$, and $C$ varied uniformly on the interval $[-20\%,+20\%]$ for 1000 runs.}
\label{three color optimum}
\end{figure} 
%----------------------------------------------------------------------------

%----------------------------------------------------------------------------
\subsection{Histogram comparison}
%----------------------------------------------------------------------------
We use a histogram to compare the three different three pulse sequences. The runs shown in figures \ref{three pi}, \ref{STIRAP plus pi pulse}, and \ref{three color optimum} were sorted in 0.01 wide bins. In Figure \ref{histogram} we see that the two coherent processes (STIRAP + $\pi$ and three color STIRAP) tend to result in nearly complete population transfer, while the sequential $\pi$ pulse scheme tends to leave about $10\%$ of the population in the lower levels. Thus, the STIRAP processes may not be negatively effected by the power fluctuations in the dye laser output.
%----------------------------------------------------------------------------
%----------------------------------------------------------------------------
%----------------------------------------------------------------------------
We use a histogram to compare the three different three pulse sequences. The runs shown in figures \ref{three pi}, \ref{STIRAP plus pi pulse}, and \ref{three color optimum} were sorted in 0.01 wide bins. In Figure \ref{histogram} we see that the two coherent processes (STIRAP + $\pi$ and three color STIRAP) tend to result in nearly complete population transfer, while the sequential $\pi$ pulse scheme tends to leave about $10\%$ of the population in the lower levels. Thus, the STIRAP processes may not be negatively effected by the power fluctuations in the dye laser output.
%----------------------------------------------------------------------------
%----------------------------------------------------------------------------
%----------------------------------------------------------------------------
We use a histogram to compare the three different three pulse sequences. The runs shown in figures \ref{three pi}, \ref{STIRAP plus pi pulse}, and \ref{three color optimum} were sorted in 0.01 wide bins. In Figure \ref{histogram} we see that the two coherent processes (STIRAP + $\pi$ and three color STIRAP) tend to result in nearly complete population transfer, while the sequential $\pi$ pulse scheme tends to leave about $10\%$ of the population in the lower levels. Thus, the STIRAP processes may not be negatively effected by the power fluctuations in the dye laser output.
%----------------------------------------------------------------------------
%----------------------------------------------------------------------------
\input{figures/transfer/histogram/histogram.tex}
%----------------------------------------------------------------------------
%----------------------------------------------------------------------------
%----------------------------------------------------------------------------
%----------------------------------------------------------------------------
%----------------------------------------------------------------------------
%----------------------------------------------------------------------------

%----------------------------------------------------------------------------
%----------------------------------------------------------------------------
%----------------------------------------------------------------------------
%----------------------------------------------------------------------------
%----------------------------------------------------------------------------
%----------------------------------------------------------------------------

%----------------------------------------------------------------------------
%----------------------------------------------------------------------------
%----------------------------------------------------------------------------
%----------------------------------------------------------------------------
%----------------------------------------------------------------------------
%----------------------------------------------------------------------------

%----------------------------------------------------------------------------
%----------------------------------------------------------------------------
\section{Stochastic collision model and density matrix methods}
First, we examine the effect of collision using a ``state vector'' approach assuming that the effect of collisions is to simply randomize the ``phase'' of the atom. Then we try to merge this idea with density matrix methods using a computer fit to the free parameters in a ``relaxation'' matrix. Modeling the effects of collisions in a stochastic manner (as in Section \ref{model} is computationally intensive and slow. The density matrix method is computational simple and would be preferred for further investigation into collisional effects.
\label{density section}
%----------------------------------------------------------------------------
\subsection{Collision model}
%----------------------------------------------------------------------------
\label{model}
%----------------------------------------------------------------------------
%----------------------------------------------------------------------------
%bb defines the bounding box for the pdf
%viewport defines the area of the pdf used
%in sidewaysfigure the last entry in bb moves the caption toward/away the pic
%in sidewaysfigure the second entry in bb moves the pic toward/away the caption
%----------------------------------------------------------------------------
\begin{figure}
\scalebox{0.8}[0.8]{
\includegraphics[bb=30 455 489 685]
{clean/clean.pdf}
}
\caption[Collisionless evolution of a three state system]{Collisionless evolution of a three state system. For this (and the following simulations) $\alpha=\beta=1$ and the time scale is the two state Rabi period associated with $\alpha$ (or $\beta$).}
\label{clean}
\end{figure}
%----------------------------------------------------------------------------

%----------------------------------------------------------------------------
%----------------------------------------------------------------------------
%bb defines the bounding box for the pdf
%viewport defines the area of the pdf used
%in sidewaysfigure the last entry in bb moves the caption toward/away the pic
%in sidewaysfigure the second entry in bb moves the pic toward/away the caption
%----------------------------------------------------------------------------
\begin{figure}
\scalebox{0.8}[0.8]{
\includegraphics[bb=30 455 489 685]
{coll_1/coll_1.pdf}
}
\caption[Evolution of a three state system with collisions - example 1]{Evolution of a three state system with collisions - example 1. The probability of a collision per Rabi period is 0.5/$\pi$.}
\label{coll_1}
\end{figure}
%----------------------------------------------------------------------------

%----------------------------------------------------------------------------
%----------------------------------------------------------------------------
Reference \cite{Siegman:1986a} describes a model for the collisional effects on the quantum mechanical evolution of an atomic system. The main effect of collisions is to randomize the phase of the expansion coefficients (i.e. the ``c's'' in Equations \ref{expansion} and \ref{se}) as they evolve. (Collisions may also transfer population with relatively low probability in inelastic collisions, but this is ignored in this analysis.) This model is easy to implement in computer code written to solve Equation \ref{se}. As the code solves the equation, it is interrupted randomly. The phase of the expansion coefficients (the expansion coefficients are in general complex) are uniformly randomized on the interval $[0,2\pi)$ while leaving the magnitude intact. Then the numerical integrator picks up where in left off at the interruption, except with the new ``randomized'' expansion coefficients and continues until another ``collision'' (i.e. interruption) takes place. See Figures \ref{clean} through \ref{coll_2} for examples.

The probability of a collision per unit time is arbitrarily selected as $0.5/\pi$ in the examples shown here (there were additional runs at $0.3/\pi$). In Figure \ref{average} we see the result when one million runs are averaged together: the behavior has the appearance of damped oscillations. This behavior seems like it may be described in a cleaner way; if not with an analytic form, then with a differential equation. In the next section we develop a formal candidate.
%----------------------------------------------------------------------------
%----------------------------------------------------------------------------
%bb defines the bounding box for the pdf
%viewport defines the area of the pdf used
%in sidewaysfigure the last entry in bb moves the caption toward/away the pic
%in sidewaysfigure the second entry in bb moves the pic toward/away the caption
%----------------------------------------------------------------------------
\begin{figure}
\scalebox{0.8}[0.8]{
\includegraphics[bb=30 455 489 685]
{coll_2/coll_2.pdf}
}
\caption[Evolution of a three state system with collisions - example 2]{Evolution of a three state system with collisions - example 2. The probability of a collision per Rabi period is 0.5/$\pi$.}
\label{coll_2}
\end{figure}
%----------------------------------------------------------------------------

%----------------------------------------------------------------------------
%----------------------------------------------------------------------------
%----------------------------------------------------------------------------
%----------------------------------------------------------------------------
%----------------------------------------------------------------------------

%----------------------------------------------------------------------------
\subsection{Density matrix formalism}
%----------------------------------------------------------------------------
\label{density formalism}
%----------------------------------------------------------------------------
The equation of motion for the density matrix is given by the Liouville-von Neumann equation:
\begin{equation}
i \hbar \frac{\partial \rho}{\partial t}
=
[H, \rho];
\label{Liouville equation}
\end{equation}
%----------------------------------------------------------------------------
where $H$ is the Hamiltonian and $\rho$ is the density matrix. If the system of interest is not totally isolated and the degrees of freedom due to its surroundings (the so called ``heat bath'') are unobserved, then equation (\ref{Liouville equation}) may not describe the time evolution of the system \cite{Blum:1981a}. In our application, we have a target molecule (system of interest) in a ``sea'' of non--target atmospheric molecules (the ``bath'').

We introduce a \emph{relaxation} operator $\hat{R}$ \cite{Band:1991a} to parameterize the effect of the surroundings:
%----------------------------------------------------------------------------
\begin{equation}
i \frac{\partial \rho}{\partial t}
=
[H, \rho]
+
i \hat{R} \vec{\rho}.
\label{open Liouville equation}
\end{equation}
%----------------------------------------------------------------------------
If we are working in $n$ dimensional Hilbert space, the left hand side and the first two terms on the right hand side are $n\cross n$ matrices. Since the last term is in Liouville space (i.e. $\vec{\rho} = (\rho_{00}, \cdots, \rho_{nn})^{T}$ and $\hat{R}$ is a $n^2\cross n^2$ matrix) there remains a one to one correspondence between the elements of each term in the equation of motion.

Let us consider the example of a many--level system (perhaps a target molecule with its many ro-vibrational levels) where we are only concerned with a small subset of these levels -- specifically two levels (perhaps the two levels resonantly coupled with a tuned laser field). The Hamiltonian from Equation \ref{two dynamics} is:
%----------------------------------------------------------------------------
\begin{equation}
H
=
i\left(
\begin{array}{cc}
0 & \alpha \\
-\alpha & 0
\end{array}
\right).
\label{Hilbert}
\end{equation}
%----------------------------------------------------------------------------
Next we give the relaxation matrix a specific form.
%----------------------------------------------------------------------------
\begin{equation}
\hat{R}
=
\hat{R}^{ext}
+
\hat{R}^{int}
+
\hat{R}^{phase};
\label{R}
\end{equation}
%----------------------------------------------------------------------------
where $\hat{R}^{ext}$ represents the tendency of the two level subset to relax into the other levels outside the subset (external relaxation), $\hat{R}^{int}$ represents the tendency of one level in the subset to relax into another level in the subset (internal relaxation), and $\hat{R}^{phase}$ represents de--phasing resulting from the relaxation process mentioned above and/or from other sources (perhaps collisions).
%----------------------------------------------------------------------------
\begin{equation}
\hat{R}^{ext}
=
\left(
\begin{array}{cccc}
-\Gamma_{00} & 0 & 0 & 0 \\
0 & 0 & 0 & 0 \\
0 & 0 & 0 & 0 \\
0 & 0 & 0 & -\Gamma_{11} 
\end{array}
\right),
\label{decay}
\end{equation}
%----------------------------------------------------------------------------
\begin{equation}
\hat{R}^{int}
=
\left(
\begin{array}{cccc}
-\Gamma^{0}_{1} & 0 & 0 & \Gamma^{0}_{1}  \\
0 & 0 & 0 & 0 \\
0 & 0 & 0 & 0 \\
\Gamma^{1}_{0} & 0 & 0 & -\Gamma^{1}_{0} 
\end{array}
\right), \quad \mbox{and}
\label{exchange}
\end{equation}
%----------------------------------------------------------------------------
\begin{equation}
\hat{R}^{phase}
=
\left(
\begin{array}{cccc}
0 & 0 & 0 & 0 \\
0 & \gamma_{01} & 0 & 0 \\
0 & 0 & \gamma_{01} & 0 \\
0 & 0 & 0 & 0 
\end{array}
\right);
\label{dephase}
\end{equation}
%----------------------------------------------------------------------------
where $\Gamma_{ii}$ is the relaxation rate of the state $i$ to some external state, $\Gamma^i_j$ is the relaxation rate of state $i$ into state $j$, and $\gamma_{ij}$ is the relaxation rate of $\rho_{ij}$ $(i \not= j)$. $\Gamma_{ii}$ is the rate at state $i$ relaxes to some other level \emph{outside} the subset under consideration; hence if $\Gamma_{ii}\not=0$ then probability will not be conserved in equation (\ref{open Liouville equation}). $\gamma_{ij}$ is applied to only the off diagonal terms in the density matrix; thus it is the rate at which the system loses coherence.
%----------------------------------------------------------------------------
%----------------------------------------------------------------------------
%----------------------------------------------------------------------------
%----------------------------------------------------------------------------
%----------------------------------------------------------------------------
%----------------------------------------------------------------------------
%----------------------------------------------------------------------------
%----------------------------------------------------------------------------
%----------------------------------------------------------------------------
%----------------------------------------------------------------------------

Now suppose our subsystem has three levels (see Figure \ref{2 color ladder}). The Hamiltonian from Equation \ref{three dynamics} is:
%----------------------------------------------------------------------------
\begin{equation}
H
=
i\left(
\begin{array}{ccc}
0 & \alpha & 0 \\
-\alpha & 0 & \beta \\
0 & -\beta & 0
\end{array}
\right).
\label{Hilbert3}
\end{equation}
%----------------------------------------------------------------------------
Using the same definitions as in the two level case, we can write the relaxation matrix terms as
%----------------------------------------------------------------------------
\begin{equation}
\hat{R}^{ext}
=
\left(
\begin{array}{ccccccccc}
-\Gamma_{00} & 0 & 0 & 0 & 0 & 0 & 0 & 0 & 0 \\
0 & 0 & 0 & 0 & 0 & 0 & 0 & 0 & 0 \\
0 & 0 & 0 & 0 & 0 & 0 & 0 & 0 & 0 \\
0 & 0 & 0 & 0 & 0 & 0 & 0 & 0 & 0 \\
0 & 0 & 0 & 0 & -\Gamma_{11} & 0 & 0 & 0 & 0 \\
0 & 0 & 0 & 0 & 0 & 0 & 0 & 0 & 0 \\
0 & 0 & 0 & 0 & 0 & 0 & 0 & 0 & 0 \\
0 & 0 & 0 & 0 & 0 & 0 & 0 & 0 & 0 \\
0 & 0 & 0 & 0 & 0 & 0 & 0 & 0 & -\Gamma_{22}
\end{array}
\right),
\label{decay3}
\end{equation}
%----------------------------------------------------------------------------
%----------------------------------------------------------------------------
\begin{equation}
\hat{R}^{int}
=
\left(
\begin{array}{ccccccccc}
-\Gamma^{0}_{1}-\Gamma^{0}_{2} & 0 & 0 & 0 & \Gamma^{0}_{1} & 0 & 0 & 0 & \Gamma^{0}_{2} \\
0 & 0 & 0 & 0 & 0 & 0 & 0 & 0 & 0 \\
0 & 0 & 0 & 0 & 0 & 0 & 0 & 0 & 0 \\
0 & 0 & 0 & 0 & 0 & 0 & 0 & 0 & 0 \\
\Gamma^{1}_{0} & 0 & 0 & 0 & -\Gamma^{1}_{0}-\Gamma^{1}_{2} & 0 & 0 & 0 & \Gamma^{1}_{2} \\
0 & 0 & 0 & 0 & 0 & 0 & 0 & 0 & 0 \\
0 & 0 & 0 & 0 & 0 & 0 & 0 & 0 & 0 \\
0 & 0 & 0 & 0 & 0 & 0 & 0 & 0 & 0 \\
\Gamma^{2}_{0} & 0 & 0 & 0 & \Gamma^{2}_{1} & 0 & 0 & 0 & -\Gamma^{2}_{0}-\Gamma^{2}_{1}
\end{array}
\right),
\label{exchange3}
\end{equation}
%----------------------------------------------------------------------------
and
%----------------------------------------------------------------------------
\begin{equation}
\hat{R}^{phase}
=
\left(
\begin{array}{ccccccccc}
0 & 0 & 0 & 0 & 0 & 0 & 0 & 0 & 0 \\
0 & \gamma_{01} & 0 & 0 & 0 & 0 & 0 & 0 & 0 \\
0 & 0 & \gamma_{02} & 0 & 0 & 0 & 0 & 0 & 0 \\
0 & 0 & 0 & \gamma_{01} & 0 & 0 & 0 & 0 & 0 \\
0 & 0 & 0 & 0 & 0 & 0 & 0 & 0 & 0 \\
0 & 0 & 0 & 0 & 0 & \gamma_{12} & 0 & 0 & 0 \\
0 & 0 & 0 & 0 & 0 & 0 & \gamma_{02} & 0 & 0 \\
0 & 0 & 0 & 0 & 0 & 0 & 0 & \gamma_{12} & 0 \\
0 & 0 & 0 & 0 & 0 & 0 & 0 & 0 & 0 
\end{array}
\right).
\label{dephase3}
\end{equation}
%----------------------------------------------------------------------------
%----------------------------------------------------------------------------
%----------------------------------------------------------------------------
%----------------------------------------------------------------------------

%----------------------------------------------------------------------------
\subsection{Collision model parametric fit}
%----------------------------------------------------------------------------
%bb defines the bounding box for the pdf
%viewport defines the area of the pdf used
%in sidewaysfigure the last entry in bb moves the caption toward/away the pic
%in sidewaysfigure the second entry in bb moves the pic toward/away the caption
%----------------------------------------------------------------------------
\begin{figure}
\scalebox{0.7}[0.7]{
\includegraphics*[bb=75 286 643 540]
{fit/fit.pdf}
}
\caption[Gaussinan fit for a single RF beat spectral feature]{The spectral feature at 1200 MHz fits a Gaussian with a FWHM of 83 MHz. This corresponds to a Gaussian in the time domain with a FWHM of 7.5 ns. This matches the observed pulse width - implying each mode is transform limited.}
\label{fit}
\end{figure}
%----------------------------------------------------------------------------

%----------------------------------------------------------------------------
%----------------------------------------------------------------------------
\section{Conclusion}
%%%%%%%%%%%%%%%%%%%%%%%%%%%%%% -*- Mode: Latex -*- %%%%%%%%%%%%%%%%%%%%%%%%%%%%
%% >>conclusion/conclusion.tex<<
%% Author          : R. Jeffrey Kowalski
%% Created On      : Fri Mar 27 13:13:03 2007
%% Last Modified On: Thu Aug  2 10:54:56 HST 2007
%%%%%%%%%%%%%%%%%%%%%%%%%%%%%%%%%%%%%%%%%%%%%%%%%%%%%%%%%%%%%%%%%%%%%%%%%%%%%%%
SLAC T486 provided the confirmation of the Askaryan effect in ice:  coherent radio Cherenkov emission from high-energy particle cascades in dense media is detectable.  Such a validation has already been performed for two other dielectrics (salt, silica sand) which have similar radio properties as ice for observing the Askaryan effect.

\par In this analysis, I have demonstrated the quadratic scaling (1.84533 $\pm$ 0.07013) of Cherenkov pulse power with shower energy which indicates that the radiation is coherent over 2.6-3.95 GHz.  Within rms uncertainties, the electric field strength of radio impulses received at the standard gain horn were in good agreement with simulations incorporating electrodynamics and shower properties in ice.  This result illustrates the roll-off of the frequency spectrum for coherent radio Cherenkov emission.

\par These results provide a promising outlook for existing ultra-high energy neutrino detectors using ice as their interaction medium.  The ANITA experiment has just completed its first flight observing the vast majority of the Antarctica ice for neutrino induced electromagnetic showers.  Having accelerator-based experimental data confirming the electric field frequency spectrum that follows our current theory will improve efforts and sensitivity in detecting UHE neutrinos.  Furthermore, SLAC T486 has illuminated what effect we can expect to see from radiation patterns penetrating the ice surface from electromagnetic cascades in the  $10^{18}\rightarrow10^{22}$ eV energy regime.
%----------------------------------------------------------------------------
%----------------------------------------------------------------------------
