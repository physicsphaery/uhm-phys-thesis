% Unicode symbols in LaTeX
%
% July 27th, 2006 -- Mike Hadmack
% hadmack@hawaii.edu
%
\documentclass[11pt]{article}
\usepackage[autogenerated,mathletters]{ucs}
\usepackage[utf8x]{inputenc}

\begin{document}
This is a demonstration of using unicode characters in a LaTeX file.  Notice that the the greek letters are not typeset in the standard LaTeX way but are instead given by their actual unicode characters.  This significantly cleans up the appearance of equations with many symbols in the source.

To input these characters I have used the Mac program Ukelele to create a custom keyboard map so that my most frequently used characters are accessible simply with an option modifier key.  Traditionally alpha is typeset with \\alpha, but I will enter it as opt-a resulting in α directly.  Be sure that you are viewing this file with UTF-8 encoding enabled or you will not see the characters properly.

This technique makes use of the 'ucs' package which should come with most latex distributions.
$$\frac{1}{2 π} \int_{∞}^{∞} h(ν)e^{i 2 π ν t}dν$$
$$y = \sin{ω t + φ}$$
\end{document}  