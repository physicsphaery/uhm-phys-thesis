%%%%%%%%%%%%%%%%%%%%%%%%%%%%%% -*- Mode: Latex -*- %%%%%%%%%%%%%%%%%%%%%%%%%%%%
%% >>introduction/motivation/motivation.tex<<
%% Author          : R. Jeffrey Kowalski
%% Created On      : Fri Mar 27 13:16:40 2007
%% Last Modified On: Thu Aug  2 10:54:56 HST 2007
%%%%%%%%%%%%%%%%%%%%%%%%%%%%%%%%%%%%%%%%%%%%%%%%%%%%%%%%%%%%%%%%%%%%%%%%%%%%%%%
Our understanding of the origin and nature of ultra-high energy cosmic rays (UHECR) and their accompanying neutrinos remains as one of the fundamental problems of experimental particle astrophysics.  Highest energy cosmic rays have long been suspected to be from extragalactic sources since the Milky Way Galaxy cannot magnetically contain and accelerate them to the energies we observe.  We still do not have a confirmed astrophysical source of these particles nor do we understand fully how they are accelerated to the Earth.  The observations of particles with energies above 10$^{20}$ eV with experiments such as HiRes and Auger have provided strong experimental support for the Greisen-Zatsepin-Kuzmin cutoff~\cite{Greisen.1966,Zatsepin.1966} and confidence in the existence of neutrinos reaching energies in the EeV scale\footnote{1 EeV = 10$^{18}$ eV}.  Detection of such neutrinos would provide new physical insight into our understanding of active galactic nuclei (AGN), gamma ray bursts, supernovae, etc. since neutrinos pass effortlessly through space to the Earth.  Ultra-high energy neutrinos, therefore, enable us to peer into the environment of the early universe and discover possible source evolutions since neutrinos will point back to their origin.
