%%%%%%%%%%%%%%%%%%%%%%%%%%%%%% -*- Mode: Latex -*- %%%%%%%%%%%%%%%%%%%%%%%%%%%%
%% >>acknowledgements.tex<<
%% Author          : R. Jeffrey Kowalski
%% Created On      : Fri Nov 3 22:06:26 2006
%% Last Modified On: Thu Aug  2 10:54:56 HST 2007
%%%%%%%%%%%%%%%%%%%%%%%%%%%%%%%%%%%%%%%%%%%%%%%%%%%%%%%%%%%%%%%%%%%%%%%%%%%%%%%
\begin{acknowledgements}
\par In the 3 years I have spent working on the ANITA project at the University of Hawai'i at M\~{a}noa, I have gained support from many individuals and institutions.  I'll begin by expressing my gratitude to the following:

\par Professor Peter W. Gorham has been my thesis advisor and the principal investigator for the ANITA experiment.  From his knowledge of RF system design, data analysis, laboratory and experimental conduct, and astrophysics, he has given me many of the essential tools needed to be a successful experimentalist.  In working with him both in UH labs and in the field (SLAC T461 \& T486 at Stanford University and A-142-M in Antarctica), I have gained the most valuable experience in my educational career.  Many thanks to Dr. Gorham for all of his teachings over the years.

\par Professor Gary S. Varner for his ongoing help with firmware development, hardware integration, and data analysis.  His leading examples shown while working with him in the field have improved my work tremendously and have given me the guidance to continue with this research.

\par ANITA project engineer, Marc Rosen, for his ongoing guidance during mechanical integration of ANITA subsystems and John G. Learned for his experience and motivation in particle astrophysics.

\par Members of the ANITA collaboration for playing a significant role during the course of this research and will continue to be colleagues in the future.  In particular, I would like to acknowledge: Kim Palladino (\textit{OSU}), David Saltzberg (\textit{UCLA}), James Beatty (\textit{OSU}), Paul Dowkowntt (\textit{WashU}), Dana Braun (\textit{WashU}), Ryan Nichol (\textit{OSU}), David Goldstein (\textit{UCI}), Chuck Naudet (\textit{JPL}), Kurt Liewer (\textit{JPL}), Bob Binns (\textit{WashU}), Amy Connolly (\textit{UCLA}), Michael DuVernois (\textit{UM}), and Jiwoo Nam (\textit{UCI}).

\par Our support staff during multiple instrument integrations has been tremendous.  Melvin Matsunaga and Roy Tom have made numerous contributions toward design and fabrication of ANITA hardware. I'd also like to thank Jan Bruce for help during all of the travel toward this research, and Jeff Griskevich for his cooperation at UCI integrations and for his excellent facility management.

\par I have also had the pleasure of working with various undergraduate, graduate students and post-doctoral researchers at the University of Hawai'i at M\~{a}noa.  Specifically, I'd like to thank: Larry Ruckman for his continuous help with electronics and firmware development, Jim Kennedy, Hannibal Starbuck, Christian Miki, Bryce Jacobson, Andres Romero-Wolf, Mike Hadmack for always having guidance toward developing Python code, Shige Matsuno, Predrag Mio\u{c}inovi\'{c}, Jason Link, and Nikolai Lehtinen.

\par During the weeks at SLAC, I had the opportunity to interact many talented individuals.  A few of them are: Dieter Walz for his knowledge of the accelerator facilities and physics; Clive Field for helping during the construction of the ice target and for getting a few last measurements for me; Richard Iverson for his knowledge of accelerator beam control; Carl Hudspeth for his cooperation while working in ESA and his foresight in producing the ice target.

\par The support crew at CSBF supercedes their expectations.  The crew chief for ANITA, Victor Davison, along with multiple individuals from their science support team aided in the success of the instrument and were essential during mechanical integration of the payload.

\par Having the opportunity to travel and work in Antarctica was a great experience and would not have been achievable without support from USAP, Raytheon Polar Services, and NSF.

\par Finally, funding and financial support is a vital element when conducting research at an academic institution.  This research would not have been made possible without the support from NASA under grant \# NAG5-5387 and the Department of Energy Office of Science High Energy Physics Division.  Lastly, special thanks to the SLAC Experiment Facilities Department.
\end{acknowledgements}

