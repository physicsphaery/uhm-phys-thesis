%%%%%%%%%%%%%%%%%%%%%%%%%%%%%% -*- Mode: Latex -*- %%%%%%%%%%%%%%%%%%%%%%%%%%%%
%% MScTHESIS.tex --
%% Author          : R. Jeffrey Kowalski
%% Created On      : Fri Nov 3 21:52:23 NZDT 2006
%% Last Modified On: Thu Aug  2 10:33:25 HST 2007
%%%%%%%%%%%%%%%%%%%%%%%%%%%%%%%%%%%%%%%%%%%%%%%%%%%%%%%%%%%%%%%%%%%%%%%%%%%%%%%
%!NOTE: This example file has been prepared according to the University of
%!      Hawaii Style & Policy Manual for Theses and Dissertations dated
%!      "Revised February 1998". If you have one with a later date, you may
%!      need to make revisions to this document as well. In any event, making
%!      sure your thesis complies with Graduate Division guidelines is
%!      ultimately your responsibility. Caveat LaTeXtor. :)
%!!!!!!!!!!!!!!!!!!!!!!!!!!!!!!!!!!!!!!!!!!!!!!!!!!!!!!!!!!!!!!!!!!!!!!!!!!!!!!

%% The options are (you can only choose one from each group):
%%
%% 10pt, 11pt, 12pt: chooses the point size for the document. "11pt" is the
%%                   default.
%%
%% oneside, twoside: whether you want your document onesided or twosided. Note
%%                   that twosided is not guaranteed to work, and style
%%                   guidelines prohibit double sided printouts on final
%%                   copy. "oneside" is the default.
%%
%% draft, final: when printing drafts you can save a lot of paper by using the
%%               "draft" option. It switches to single spacing, displays overful
%%               hboxes with a black box, prints a version number on title page 
%%               and omits signature page. Of course for the final copy make
%%               sure to use the "final" option! "final" is the default.
%%
%% cm, times, palatino, newcent, bookman: switches between different font
%%                                        sets. "cm" is the Computer Modern
%%                                        font (TeX's default), the rest are
%%                                        PostScript fonts. "times" is the
%%                                        default.
%%
%% thesis, dissertation: switches between the style for a master's thesis and a 
%%                       Ph.D. dissertation. The differences are fairly minor
%%                       and limited to the front matter. "thesis" is the
%%                       default.
%%
%% actual, proposal: switches between actual document and proposal mode. In
%%                   proposal mode: the title page is simplified, the
%%                   version number is always printed, and the signature page
%%                   is omitted.
%%
%%% Load the uhthesis2e document class
%\documentclass[11pt,final,cm,dissertation,actual]{uhthesis2e}
% \documentclass[11pt,draft,palatino,thesis,proposal]{uhthesis2e} %%use for single spacing
\documentclass[11pt,final,palatino,thesis,actual]{uhthesis2e} %%begins double spacing
% \documentclass[11pt,final,cm,thesis,proposal]{uhthesis2e}
% \documentclass[11pt,final,times,thesis,proposal]{uhthesis2e}
% \documentclass[11pt,final,newcent,thesis,proposal]{uhthesis2e}  %%This one is kind of cool
% \documentclass[11pt,final,bookman,thesis,proposal]{uhthesis2e}
%\documentclass[11pt,draft,cm,thesis,actual]{uhthesis2e}
%%% Load some useful packages:
%% Package to linebreak URLs in a sane manner.
\usepackage{url}
%----------------------------------------------------------------------------
%----------------------------LaTeX Packages----------------------------------
%----------------------------------------------------------------------------
% Added by Jeff Kowalski
\usepackage{acronym}
\usepackage{graphicx}
\usepackage{rotating}
\usepackage{sistyle}
\usepackage{lscape} %use for rotating tables
\usepackage{url}
\usepackage{color} %for adding color to text
% \usepackage{setspace}
% \usepackage{fullpage}
% \usepackage{doublespace}
\graphicspath{figures,figures/slacT486/}
%\usepackage{amsmath}
%\usepackage[pdftex]{graphicx}
% \usepackage{epsfig}
%\usepackage{fancyhdr}

%\numberwithin{equation}{section}
%----------------------------------------------------------------------------
%----------------------------LaTeX definitions-------------------------------
%----------------------------------------------------------------------------
% Added by Jeff Kowalski
\input epsf
\newlength\singlespacelength
\newlength\doublespacelength
% \setlength\singlespacelength{15\p@}
\setlength\singlespacelength{3.5mm}
\setlength\doublespacelength{22pt}
\newcommand\startsinglespace{\setlength\baselineskip{\singlespacelength}}
\newcommand\startdoublespace{\setlength\baselineskip{\doublespacelength}}
%
\newcommand{\per}[1]{\ifmmode{{#1}^{-1}}\else{{#1}$\vphantom{#1}^{-1}$}\fi}
%def\fourdots{\ifmmode{\ldotp\ldotp\mu\ldotp\ldotp}\else${\ldotp\ldotp\ldotp\ldotp}$\fi}

%Distance
%
\newcommand\Gpc{\ensuremath{\rm\,Gpc}}
\newcommand\Mpc{\ensuremath{\rm\,Mpc}}
\newcommand\kpc{\ensuremath{\rm\,kpc}}
\newcommand\pc{\ensuremath{\rm\,pc}}
\newcommand\km{\ensuremath{\rm\,km}}
\newcommand\m{\ensuremath{\rm\,m}}
\newcommand\cm{\ensuremath{\rm\,cm}}
\newcommand\mm{\ensuremath{\rm\,mm}}
\newcommand\um{\ensuremath{\rm\,\mu m}}
\newcommand\nm{\ensuremath{\rm\,nm}}
\newcommand\Ang{\ensuremath{\rm\,\mbox{\AA}}}

%Time
%
\newcommand\ms{\ensuremath{\rm\,ms}}
\newcommand\s{\ensuremath{\rm\,s}}
%\newcommand\min{\ensuremath{\rm\,min}}
%\newcommand\mins{\ensuremath{\rm\,mins}}
\newcommand\mins[1][0]{\ensuremath{\rm\,min%
  \ifcase#1%
  s\else\fi}}
%\newcommand\hr{\ensuremath{\rm\,hr}}
%\newcommand\hrs{\ensuremath{\rm\,hrs}}
\newcommand\hrs[1][0]{\ensuremath{\rm\,hr%
  \ifcase#1%
  s\else\fi}}
%\newcommand\day{\ensuremath{\rm\,day}}
%\newcommand\days{\ensuremath{\rm\,days}}
\newcommand\days[1][0]{\ensuremath{\rm\,day%
  \ifcase#1%
  s\else\fi}}
\newcommand\yrs[1][0]{\ensuremath{\rm\,yr%
  \ifcase#1%
  s\else\fi}}
\newcommand\Hz{\ensuremath{\rm\,s^{-1}}}
\newcommand\MJD{\ensuremath{\rm MJD\,}}

\newcommand\mags{\ensuremath{\rm\,mags}}
\newcommand\J{\ensuremath{\rm\,J}}
\newcommand\W{\ensuremath{\rm\,J\,s^{-1}}}
\newcommand\C{\ensuremath{\rm\,C}}
\newcommand\kg{\ensuremath{\rm\,kg}}
\newcommand\eV{\ensuremath{\rm\,eV}}
\newcommand\keV{\ensuremath{\rm\,keV}}
\newcommand\MeV{\ensuremath{\rm\,MeV}}
\newcommand\K{\ensuremath{\rm\,K}}
\newcommand\Ohm{\ensuremath{\rm\,\Omega}}
\newcommand\Crab{\ensuremath{\rm\,Crab}}

%Solar
%
\newcommand\Ls{\ensuremath{L_\odot}}%{\ifmmode{L_\odot}\else{${L_\odot}$}\fi}
\newcommand\Ms{\ensuremath{M_\odot}}%{\ifmmode{M_\odot}\else{${M_\odot}$}\fi}
\newcommand\Rs{\ensuremath{R_\odot}}%{\ifmmode{R_\odot}\else{${R_\odot}$}\fi}

\expandafter\ifx\csname deg\endcsname\relax%
 \newcommand\deg{O}\else\fi
\expandafter\ifx\csname arcmin\endcsname\relax%
 \newcommand\arcmin{m}\else\fi
\expandafter\ifx\csname arcsec\endcsname\relax%
 \newcommand\arcsec{s}\else\fi
\renewcommand\deg{\ensuremath{\vphantom{0}^\circ}}
\renewcommand\arcmin{\ensuremath{\vphantom{0}^\prime}}
\renewcommand\arcsec{\ensuremath{\vphantom{0}^{\prime\prime}}}
\newcommand\arch{\ensuremath{\mathrel{\mathchoice
 {\vcenter{\offinterlineskip\halign{\hfil\!$\displaystyle##$\hfil\cr\vphantom{hms}h\cr\cdot\cr}}}
 {\vcenter{\offinterlineskip\halign{\hfil\!$\textstyle##$\hfil\cr\vphantom{hms}h\cr\cdot\cr}}}
 {\vcenter{\offinterlineskip\halign{\hfil\!$\scriptstyle##$\hfil\cr\vphantom{hms}h\cr\cdot\cr}}}
 {\vcenter{\offinterlineskip\halign{\hfil\!$\scriptscriptstyle##$\hfil\cr\vphantom{hms}h\cr\cdot\cr}}}}}}
\newcommand\arcm{\ensuremath{\mathrel{\mathchoice
 {\vcenter{\offinterlineskip\halign{\hfil\!$\displaystyle##$\hfil\cr\vphantom{hms}m\cr\cdot\cr}}}
 {\vcenter{\offinterlineskip\halign{\hfil\!$\textstyle##$\hfil\cr\vphantom{hms}m\cr\cdot\cr}}}
 {\vcenter{\offinterlineskip\halign{\hfil\!$\scriptstyle##$\hfil\cr\vphantom{hms}m\cr\cdot\cr}}}
 {\vcenter{\offinterlineskip\halign{\hfil\!$\scriptscriptstyle##$\hfil\cr\vphantom{hms}m\cr\cdot\cr}}}}}}
\newcommand\arcs{\ensuremath{\mathrel{\mathchoice
 {\vcenter{\offinterlineskip\halign{\hfil\!$\displaystyle##$\hfil\cr\vphantom{hms}s\cr\cdot\cr}}}
 {\vcenter{\offinterlineskip\halign{\hfil\!$\textstyle##$\hfil\cr\vphantom{hms}s\cr\cdot\cr}}}
 {\vcenter{\offinterlineskip\halign{\hfil\!$\scriptstyle##$\hfil\cr\vphantom{hms}s\cr\cdot\cr}}}
 {\vcenter{\offinterlineskip\halign{\hfil\!$\scriptscriptstyle##$\hfil\cr\vphantom{hms}s\cr\cdot\cr}}}}}}

\newcommand{\sinc}{{\rm sinc}}
\newcommand{\cross}{\times}
\newcommand{\Vector}{\mathbf}
\newcommand{\del}{\nabla}
\newcommand{\bra}[1]{\left\langle \hspace{0.10em}#1 \hspace{0.10em}\right|}
\newcommand{\ket}[1]{\left| \hspace{0.15em} #1 \hspace{0.15em}\right\rangle}
\newcommand{\braket}[2]
{\left\langle \hspace{0.10em} #1 \hspace{0.10em}
\right|
#2 \left\rangle \right.}
\newcommand{\expectation}[1]{\left\langle #1 \right\rangle}
% Force hyphenation...
\hyphenation{Ask-ar-yan}
\hyphenation{Sci-ent-ific-Atlanta}
\hyphenation{Gor-ham}
%  \hyphenation{exp-eri-ment}
% \hyphenation{/man-ual/html\_no-de/GNU-Free-Documentation-License.html}
%----------------------------------------------------------------------------
%----------------------------------------------------------------------------

%%% End of preamble
\begin{document}

%%% Declarations for Front Matter. Capitalize all of these values
%%% "normally". This allows the document class to format them properly.
%% Full title of thesis or dissertation, capitalized like a title should be.
% \title{Observation of the Askaryan Effect in Ice and Calibration of the ANITA Experiment}
\title{Observation of the Askaryan Effect in Ice with the ANITA Experiment}
%% Your name, capitalized normally. Do not include any titles like Dr.
\author{R. Jeffrey Kowalski}
%% Month in which you intend to receive your degree (i.e. graduation).
%% Presumably this will be one of: May, August, or December.
\degreemonth{August}
%% Year of expected graduation.
\degreeyear{2007}
%% Type of degree to be conferred.
\degree{Master of Science}
%% This is the chairperson of your committee. Do not use titles like Dr.
\chair{Peter W. Gorham}
%% The other members of your committee, seperated by "\\". Again, no titles,
%% and it is customary to list the outside committee member (if you have one)
%% last.
\othermembers{
Gary S. Varner\\
John G. Learned}
%% This is the total size of your committee, including the chairperson. The
%% signature page routine will only handle up to 6 members; if you have more
%% than that you will need to modify the document class.
\numberofmembers{3}
%% The field in which you are obtaining your degree, capitalized normally.
\field{Physics}
%% The version number of your document. Consistent use of this will enable you
%% to tell old drafts from new ones. Final actual documents omit this
%% automatically so you can use it without fear of submission problems at the
%% end. If you do not define this parameter, it defaults to "1.0.0".
\versionnum{2.6.3}

%%% Create the title page from all the information above. Note that the
%%% titlepage is outside the front matter.
\maketitle

\begin{frontmatter}

\signaturepage %%% Create the signature page (when indicated by the options)
\copyrightpage %%% Create the copyright page
%%%%%%%%%%%%%%%%%%%%%%%%%%%%%% -*- Mode: Latex -*- %%%%%%%%%%%%%%%%%%%%%%%%%%%%
%% >>dedication.tex<<
%% Author          : R. Jeffrey Kowalski
%% Created On      : Fri Nov 3 22:08:51 2006
%% Last Modified On: Tue Apr 10 22:09:53 HST 2007
%%%%%%%%%%%%%%%%%%%%%%%%%%%%%%%%%%%%%%%%%%%%%%%%%%%%%%%%%%%%%%%%%%%%%%%%%%%%%%%
\begin{dedication}
\null\vfil
{\large
\begin{center}
To Mom and Dad,\\\vspace{12pt}
Debbie and Rick Kowalski\\\vspace{12pt}
\end{center}}
\vfil\null
\end{dedication}
 %%% Bring in the dedication page from external file
%%%%%%%%%%%%%%%%%%%%%%%%%%%%%% -*- Mode: Latex -*- %%%%%%%%%%%%%%%%%%%%%%%%%%%%
%% >>acknowledgements.tex<<
%% Author          : [Your Name]
%% Created On      : 
%% Last Modified On: 
%%%%%%%%%%%%%%%%%%%%%%%%%%%%%%%%%%%%%%%%%%%%%%%%%%%%%%%%%%%%%%%%%%%%%%%%%%%%%%%
\begin{acknowledgements}
Acknowledge some folks here.
\end{acknowledgements}

 %%% Bring in the acknowledgements section from external file
%%%%%%%%%%%%%%%%%%%%%%%%%%%%%% -*- Mode: Latex -*- %%%%%%%%%%%%%%%%%%%%%%%%%%%%
%% >>abstract.tex<<
%% Author          : [Your Name]
%% Created On      : 
%% Last Modified On: 
%%%%%%%%%%%%%%%%%%%%%%%%%%%%%%%%%%%%%%%%%%%%%%%%%%%%%%%%%%%%%%%%%%%%%%%%%%%%%%%
\begin{abstract}
\par Type your abstract here.
\end{abstract}
 %%% Bring in the abstract section from external file
\tableofcontents %%% Generate table of contents
\listoftables %%% Generate list of tables
\listoffigures %%% Generate list of figures
\chapter*{List of Acronyms} 
\addcontentsline{toc}{chapter}{List of Acronyms} 
\begin{acronym}
\acro{BDP}{Boogie Down Productions}
\acro{KRSONE}{Knowledge Reigns Supreme Over Nearly Everyone}
\acro{SOD}{Stormtroopers of Death}
\acro{MOD}{Method of Destruction}
\acro{COC}{Corrosion Of Conformity}
\acro{CBGB}{Country, Blue Grass \& Blues}
\acro{IROC}{International Race Of Champions}
\acro{NWOBHM}{New Wave Of British Heavy Metal}
\acro{NAMBLA}{Multi--Color Coherent Quantum Control Laser Induced Fluroescence Light Imaging Detection and Ranging}
\end{acronym} %%% Generate list of acronyms

\end{frontmatter}

%%%%%%%%%%%%%%%%%%%%%%%%%%%%%% -*- Mode: Latex -*- %%%%%%%%%%%%%%%%%%%%%%%%%%%%
%% >>body.tex<<
%% Author          : [Your Name]
%% Created On      : 
%% Last Modified On: 
%%%%%%%%%%%%%%%%%%%%%%%%%%%%%%%%%%%%%%%%%%%%%%%%%%%%%%%%%%%%%%%%%%%%%%%%%%%%%%%
\chapter{Introduction}
\label{c:intro}
%%%%%%%%%%%%%%%%%%%%%%%%%%%%%% -*- Mode: Latex -*- %%%%%%%%%%%%%%%%%%%%%%%%%%%%
%% >>slacT486/intro.tex<<
%% Author          : R. Jeffrey Kowalski
%% Created On      : Tue Apr 10 19:50:51 HST 2007
%% Last Modified On: Thu Aug  2 10:54:56 HST 2007
%%%%%%%%%%%%%%%%%%%%%%%%%%%%%%%%%%%%%%%%%%%%%%%%%%%%%%%%%%%%%%%%%%%%%%%%%%%%%%%
From June 19-24, 2006, the experiment, SLAC T486, was performed in the End Station A facility at the Stanford Linear Accelerator Center to measure the Askaryan effect in ice.  28.5 GeV electrons were accelerated with typically 10$^9$ particles in 10 picosecond bunches and delivered into a 7.5 metric tonne target of carving-grade ice to produce electromagnetic showers.  In a dense media like ice, coherent microwave Cherenkov radiation emerges from the particle shower and propagates to the surface of the target where radio antennas can detect the radiation.  This chapter outlines the T486 experiment and the analysis of the Askaryan effect in ice.
%
\chapter{Dogs}
\label{c:nuAstro}
%%%%%%%%%%%%%%%%%%%%%%%%%%%%%% -*- Mode: Latex -*- %%%%%%%%%%%%%%%%%%%%%%%%%%%%
%% >>science/science.tex<<
%% Author          : [Your Name]
%% Created On      : 
%% Last Modified On: 
%%%%%%%%%%%%%%%%%%%%%%%%%%%%%%%%%%%%%%%%%%%%%%%%%%%%%%%%%%%%%%%%%%%%%%%%%%%%%%%
%%%%%%%%%%%%%%%%%%%%%%%%%%%%%% -*- Mode: Latex -*- %%%%%%%%%%%%%%%%%%%%%%%%%%%%
%% >>slacT486/intro.tex<<
%% Author          : R. Jeffrey Kowalski
%% Created On      : Tue Apr 10 19:50:51 HST 2007
%% Last Modified On: Thu Aug  2 10:54:56 HST 2007
%%%%%%%%%%%%%%%%%%%%%%%%%%%%%%%%%%%%%%%%%%%%%%%%%%%%%%%%%%%%%%%%%%%%%%%%%%%%%%%
From June 19-24, 2006, the experiment, SLAC T486, was performed in the End Station A facility at the Stanford Linear Accelerator Center to measure the Askaryan effect in ice.  28.5 GeV electrons were accelerated with typically 10$^9$ particles in 10 picosecond bunches and delivered into a 7.5 metric tonne target of carving-grade ice to produce electromagnetic showers.  In a dense media like ice, coherent microwave Cherenkov radiation emerges from the particle shower and propagates to the surface of the target where radio antennas can detect the radiation.  This chapter outlines the T486 experiment and the analysis of the Askaryan effect in ice.
%
\chapter{Cats}
\label{c:anita}
%%%%%%%%%%%%%%%%%%%%%%%%%%%%%% -*- Mode: Latex -*- %%%%%%%%%%%%%%%%%%%%%%%%%%%%
%% >>instrument/instrument.tex<<
%% Author          : [Your Name]
%% Created On      : 
%% Last Modified On: 
%%%%%%%%%%%%%%%%%%%%%%%%%%%%%%%%%%%%%%%%%%%%%%%%%%%%%%%%%%%%%%%%%%%%%%%%%%%%%%%
%%%%%%%%%%%%%%%%%%%%%%%%%%%%%% -*- Mode: Latex -*- %%%%%%%%%%%%%%%%%%%%%%%%%%%%
%% >>slacT486/intro.tex<<
%% Author          : R. Jeffrey Kowalski
%% Created On      : Tue Apr 10 19:50:51 HST 2007
%% Last Modified On: Thu Aug  2 10:54:56 HST 2007
%%%%%%%%%%%%%%%%%%%%%%%%%%%%%%%%%%%%%%%%%%%%%%%%%%%%%%%%%%%%%%%%%%%%%%%%%%%%%%%
From June 19-24, 2006, the experiment, SLAC T486, was performed in the End Station A facility at the Stanford Linear Accelerator Center to measure the Askaryan effect in ice.  28.5 GeV electrons were accelerated with typically 10$^9$ particles in 10 picosecond bunches and delivered into a 7.5 metric tonne target of carving-grade ice to produce electromagnetic showers.  In a dense media like ice, coherent microwave Cherenkov radiation emerges from the particle shower and propagates to the surface of the target where radio antennas can detect the radiation.  This chapter outlines the T486 experiment and the analysis of the Askaryan effect in ice.

%
\chapter{Conclusion}
\label{c:concl}
%----------------------------------------------------------------------------
%------------------------------Broad objectives------------------------------
%----------------------------------------------------------------------------
This chapter chronicles the stages of laboratory development undergone over the past few years. The main measurements at each stage were used as a guide to develop the equipment and techniques required for demonstration of molecular control in LIDAR systems.
%----------------------------------------------------------------------------
%----------------------------------So what?----------------------------------
%----------------------------------------------------------------------------

As each stage was completed various components of the apparatus were either designed and assembled or evolved to the next generation. After the installation of the PMT at its output the monochromator served each experiment well until the recent aromatic compound measurements. The Hg pulser and Pockles cell system went through various stages of development starting with the initial tests of the Hg pulser on LED's to the integration of the system with the YAG pumped dye laser system during the fluorescence line decay measurements. The software model was tested at each stage from the familiar non-resonant HeNe LIF to pulsed resonant dye LIF. The data acquisition system was built for the first dye laser experiments and has remained relatively unchanged since then. Recently, a calibration issue with the monochromator self scan feature has prompted the need of a second generation of the data acquisition software.
%----------------------------------------------------------------------------
%---------------------------------Synthesize---------------------------------
%----------------------------------------------------------------------------
%----------------------------------------------------------------------------
%----------------------------------------------------------------------------
%----------------------------------------------------------------------------


 %%% Bring in the body of the thesis from external file
%%%%%%%%%%%%%%%%%%%%%%%%%%%%%% -*- Mode: Latex -*- %%%%%%%%%%%%%%%%%%%%%%%%%%%%
%% >>appendix/appendix.tex<<
%% Author          : [Your Name]
%% Created On      : 
%% Last Modified On: 
%%%%%%%%%%%%%%%%%%%%%%%%%%%%%%%%%%%%%%%%%%%%%%%%%%%%%%%%%%%%%%%%%%%%%%%%%%%%%%%
\appendix
\chapter{Ye Old Appendix}
\label{ap:iceTarget}
 %% Bring in any appendices from external file
\startdoublespace %% Remove this if the bibliography needs to be single spaced
\bibliography{bib/thesis} %%% Input file for bibliography
\bibliographystyle{h-physrev} %% Use this for an alphabetically organized bibliography phys-rev style
% \bibliographystyle{unsrt} %% Use this for a reference order organized bibliography

\end{document}
