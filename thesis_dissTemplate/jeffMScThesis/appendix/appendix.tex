%%%%%%%%%%%%%%%%%%%%%%%%%%%%%% -*- Mode: Latex -*- %%%%%%%%%%%%%%%%%%%%%%%%%%%%
%% >>appendix/appendix.tex<<
%% Author          : R. Jeffrey Kowalski
%% Created On      : Thu Apr 12 23:37:21 HST 2007
%% Last Modified On: Thu Aug  2 10:54:56 HST 2007
%%%%%%%%%%%%%%%%%%%%%%%%%%%%%%%%%%%%%%%%%%%%%%%%%%%%%%%%%%%%%%%%%%%%%%%%%%%%%%%
\appendix
\chapter{SLAC T486 Ice Target}
\label{ap:iceTarget}
As discussed in section~\ref{ss:iceTarget}, many independent measurements were made to assess the effect of surface roughness and to calibrate the slope of the carved ice surface.  Since the target was carved using various hand tools, a careful examination of the surface is necessary to further understand the Cherenkov emission leaving the target.

\par The target was assembled using $\sim$55 136 kg blocks of carving grade ice.  With the target enclosure empty, each block was carefully stacked inside forming a large rectangular prism.  The surface of the unified block of ice was then carved to create a $\sim$8$^{\circ}$ slope to prevent total-internal-reflection (TIR) of the emerging radiation from the electromagnetic shower.

\par After carving, a 13 $\times$ 6 element grid was produced that contained the length, width, height, slope along the x-axis, slope along the y-axis, and the position of the ice relative to the two walls of the target enclosure along the y-axis.  All angles were measured with a digital protractor while positional data was taken standard measuring tape.  I would estimate errors in length, width, height, etc. to be within $\pm$2 cm and angle uncertainties to be $\pm$0.5$^{\circ}$.  The surface data from the ice target is summarized in the following pages.

%%%%%%%%%%%%%%%%%%%%%%%%%%%%%% -*- Mode: Latex -*- %%%%%%%%%%%%%%%%%%%%%%%%%%%%
%% >>appendix/ice/ice.tex<<
%% Author          : R. Jeffrey Kowalski
%% Created On      : Thu Apr 12 23:37:21 HST 2007
%% Last Modified On: Thu May 24 16:48:26 HST 2007
%%%%%%%%%%%%%%%%%%%%%%%%%%%%%%%%%%%%%%%%%%%%%%%%%%%%%%%%%%%%%%%%%%%%%%%%%%%%%%%
\begin{landscape}
\begin{center}
\begin{tabular}{| c | c | c | c | c | c | c | c | c |} \hline
Length (\textit{cm}) & Width (\textit{cm}) & Height (\textit{cm}) & $\theta_{x} (^{\circ})$ & $\theta_{y} (^{\circ})$ & Target Width (\textit{cm}) & +y - gap (\textit{cm}) & -y - gap (\textit{cm}) \\
\hline \hline
0 & 15.24 & 13.2 & -5.5 & -10.4 & 179.705 & 8.89 & 10.16 \\
0 & 45.72 & 13.5 & -11.8 & -1.3 & 179.705 & 8.89 & 10.16 \\
0 & 76.2 & 13.7 & -9.4 & 0 & 179.705 & 8.89 & 10.16 \\
0 & 106.68 & 13.2 & -9.4 & -3.9 & 179.705 & 8.89 & 10.16 \\
0 & 137.16 & 11.8 & -15.5 & -1 & 179.705 & 8.89 & 10.16 \\
0 & 167.64 & 13 & -15.5 & -2.3 & 179.705 & 8.89 & 10.16 \\
30.48 & 15.24 & 16.5 & -6.9 & -4.4 & 179.705 & 7.62 & 7.62 \\
30.48 & 45.72 & 16.3 & 1.1 & -4.6 & 179.705 & 7.62 & 7.62 \\
30.48 & 76.2 & 16.5 & 0.3 & 0.5 & 179.705 & 7.62 & 7.62 \\
30.48 & 106.68 & 16.2 & -4.4 & -1.8 & 179.705 & 7.62 & 7.62 \\
30.48 & 137.16 & 17 & 0.4 & 0.5 & 179.705 & 7.62 & 7.62 \\
30.48 & 167.64 & 16.4 & -3.1 & -2.7 & 179.705 & 7.62 & 7.62 \\
60.96 & 15.24 & 18.5 & -5.4 & -7.4 & 179.705 & 8.255 & 8.255 \\
60.96 & 45.72 & 17.2 & -7 & -0.58 & 179.705 & 8.255 & 8.255 \\
60.96 & 76.2 & 18.2 & -10.9 & -0.5 & 179.705 & 8.255 & 8.255 \\
60.96 & 106.68 & 18.2 & -1.4 & 1.9 & 179.705 & 8.255 & 8.255 \\
60.96 & 137.16 & 17.5 & -7 & -2 & 179.705 & 8.255 & 8.255 \\
60.96 & 167.64 & 18 & -12.2 & -11.3 & 179.705 & 8.255 & 8.255 \\
91.44 & 15.24 & 17.6 & -9.3 & -4.3 & 179.705 & 10.16 & 8.89 \\
91.44 & 45.72 & 18.8 & -6.9 & -2.1 & 179.705 & 10.16 & 8.89 \\
91.44 & 76.2 & 18.2 & -8.8 & 0.2 & 179.705 & 10.16 & 8.89 \\
91.44 & 106.68 & 18.2 & -8.7 & -1 & 179.705 & 10.16 & 8.89 \\
91.44 & 137.16 & 17.5 & -6.6 & 1.5 & 179.705 & 10.16 & 8.89 \\
91.44 & 167.64 & 19.2 & -5.5 & -5.9 & 179.705 & 10.16 & 8.89 \\
121.92 & 15.24 & 20.5 & -10.1 & -9.5 & 179.705 & 9.525 & 10.16 \\
121.92 & 45.72 & 19.9 & -5.3 & -2.5 & 179.705 & 9.525 & 10.16 \\
121.92 & 76.2 & 20 & -8.2 & -3.5 & 179.705 & 9.525 & 10.16 \\
121.92 & 106.68 & 19 & -9.3 & -1.4 & 179.705 & 9.525 & 10.16 \\
121.92 & 137.16 & 20.5 & -10 & 3.3 & 179.705 & 9.525 & 10.16 \\
121.92 & 167.64 & 21.2 & -12.3 & -2 & 179.705 & 9.525 & 10.16 \\
\hline
\end{tabular}
\begin{tabular}{| c | c | c | c | c | c | c | c | c |} \hline
Length (\textit{cm}) & Width (\textit{cm}) & Height (\textit{cm}) & $\theta_{x} (^{\circ})$ & $\theta_{y} (^{\circ})$ & Target Width (\textit{cm}) & +y - gap (\textit{cm}) & -y - gap (\textit{cm}) \\
\hline \hline
152.4 & 15.24 & 22.8 & -4.9 & -4.6 & 180.0352 & 11.43 & 6.35 \\
152.4 & 45.72 & 21.8 & -6 & 0 & 180.0352 & 11.43 & 6.35 \\
152.4 & 76.2 & 21.5 & 4 & -0.7 & 180.0352 & 11.43 & 6.35 \\
152.4 & 106.68 & 21.8 & -4.3 & 3.8 & 180.0352 & 11.43 & 6.35 \\
152.4 & 137.16 & 22.6 & -4.1 & -1.8 & 180.0352 & 11.43 & 6.35 \\
152.4 & 167.64 & 22.1 & -10.2 & 0 & 180.0352 & 11.43 & 6.35 \\
182.88 & 15.24 & 23 & -6.6 & 1.5 & 179.705 & 12.065 & 6.35 \\
182.88 & 45.72 & 22.4 & -4.4 & 3.7 & 179.705 & 12.065 & 6.35 \\
182.88 & 76.2 & 22.8 & -4.6 & -0.5 & 179.705 & 12.065 & 6.35 \\
182.88 & 106.68 & 23.6 & -4.8 & 3.1 & 179.705 & 12.065 & 6.35 \\
182.88 & 137.16 & 24 & 0.6 & 2 & 179.705 & 12.065 & 6.35 \\
182.88 & 167.64 & 24.6 & -8.8 & 4.7 & 179.705 & 12.065 & 6.35 \\
213.36 & 15.24 & 23 & -8.4 & 10.7 & 180.0352 & 13.97 & 6.35 \\
213.36 & 45.72 & 23.2 & -1.9 & 8.5 & 180.0352 & 13.97 & 6.35 \\
213.36 & 76.2 & 23.6 & -5.6 & 7.5 & 180.0352 & 13.97 & 6.35 \\
213.36 & 106.68 & 24 & -11.8 & 0.5 & 180.0352 & 13.97 & 6.35 \\
213.36 & 137.16 & 23.8 & -7.3 & 1 & 180.0352 & 13.97 & 6.35 \\
213.36 & 167.64 & 24 & -4.7 & 0.7 & 180.0352 & 13.97 & 6.35 \\
243.84 & 15.24 & 28.5 & -2 & 2 & 180.0352 & 13.6652 & 6.35 \\
243.84 & 45.72 & 27.8 & -3.9 & 11.5 & 180.0352 & 13.6652 & 6.35 \\
243.84 & 76.2 & 26.5 & -1.8 & 11.4 & 180.0352 & 13.6652 & 6.35 \\
243.84 & 106.68 & 27.2 & -2.8 & 15.1 & 180.0352 & 13.6652 & 6.35 \\
243.84 & 137.16 & 26.5 & -7.2 & 11.5 & 180.0352 & 13.6652 & 6.35 \\
243.84 & 167.64 & 26.2 & -0.7 & 8.2 & 180.0352 & 13.6652 & 6.35 \\
274.32 & 15.24 & 28 & 10 & 9 & 180.34 & 7.62 & 5.715 \\
274.32 & 45.72 & 25.5 & -7.9 & 11.6 & 180.34 & 7.62 & 5.715 \\
274.32 & 76.2 & 25 & -9.2 & 5.5 & 180.34 & 7.62 & 5.715 \\
274.32 & 106.68 & 24.2 & 7.5 & 11 & 180.34 & 7.62 & 5.715 \\
274.32 & 137.16 & 24 & 11.4 & 6 & 180.34 & 7.62 & 5.715 \\
274.32 & 167.64 & 24 & 8.8 & 13.4 & 180.34 & 7.62 & 5.715 \\
\hline
\end{tabular}
\begin{tabular}{| c | c | c | c | c | c | c | c | c |} \hline
Length (\textit{cm}) & Width (\textit{cm}) & Height (\textit{cm}) & $\theta_{x} (^{\circ})$ & $\theta_{y} (^{\circ})$ & Target Width (\textit{cm}) & +y - gap (\textit{cm}) & -y - gap (\textit{cm}) \\
\hline \hline
304.8 & 15.24 & 25 & -8.9 & 8.5 & 180.34 & 9.525 & 7.62 \\
304.8 & 45.72 & 25 & -8.2 & 8 & 180.34 & 9.525 & 7.62 \\
304.8 & 76.2 & 24 & 6.7 & 9.9 & 180.34 & 9.525 & 7.62 \\
304.8 & 106.68 & 23.4 & 8.7 & -8.9 & 180.34 & 9.525 & 7.62 \\
304.8 & 137.16 & 23.5 & -12.4 & 8.1 & 180.34 & 9.525 & 7.62 \\
304.8 & 167.64 & 23 & 2.9 & 12.6 & 180.34 & 9.525 & 7.62 \\
335.28 & 15.24 & 23.5 & -10.5 & 10.5 & 180.34 & 10.16 & 8.89 \\
335.28 & 45.72 & 21.5 & -7.2 & -2 & 180.34 & 10.16 & 8.89 \\
335.28 & 76.2 & 21.8 & -9.9 & 10.6 & 180.34 & 10.16 & 8.89 \\
335.28 & 106.68 & 21 & -8.1 & 10.7 & 180.34 & 10.16 & 8.89 \\
335.28 & 137.16 & 20.8 & -8.3 & -8 & 180.34 & 10.16 & 8.89 \\
335.28 & 167.64 & 20 & -4.2 & 14.9 & 180.34 & 10.16 & 8.89 \\
365.76 & 15.24 & 23.5 & -12.8 & 15 & 180.34 & 13.335 & 9.525 \\
365.76 & 45.72 & 21.5 & -0.3 & 5 & 180.34 & 13.335 & 9.525 \\
365.76 & 76.2 & 20 & -4.3 & 0 & 180.34 & 13.335 & 9.525 \\
365.76 & 106.68 & 30.5 & -23.5 & 0.3 & 180.34 & 13.335 & 9.525 \\
365.76 & 137.16 & 20 & -31.4 & 25 & 180.34 & 13.335 & 9.525 \\
365.76 & 167.64 & 31 & "No data" & "No data" & 180.34 & 13.335 & 9.525 \\
\hline
\end{tabular}
\label{tab:iceRoughness}
\end{center}
\end{landscape}

\chapter{Bases, Camps, \& Research Stations on the Antarctic Continent}
\label{ap:antMap}
During preliminary analysis of the ANITA 2006-2007 Antarctic flight, a comprehensive map of the continent was developed to assess possible man-made sources of EMI.  The first set of bases came from \textcolor{blue}{http://members.eunet.at/castaway/stations/aa-bases.html} which gave a detailed listing of all research stations on Antarctica from the late 1940's to the present.  In addition to bases and research stations, there were many expedition paths that could contribute to EMI.  The main expeditions that could provide GPS data of their trek were:

\begin{enumerate}
\item Correne, Erasmus, \& Coetzer
\item John Wilton Davies
\item Team N2i (Novo to Inaccessibility)
\item ITASE (International Trans-Antarctic Scientific Expedition).
\end{enumerate}

\noindent Another possible source of EMI could come from automated-weather-stations (AWS) that exist all over the continent to collect climate data.  A detailed list of the existing AWS stations was taken from \textcolor{blue}{http://amrc.ssec.wisc.edu/aws.html}.

\par The map was generated with a combination of \textsf{python}$^{TM}$, \textsc{matplotlib}, and the Basemap toolkit extension for \textsc{matplotlib}.  The topographic data was provided by the National Oceanic \& Atmospheric Administration (NOAA) with 20 m resolution\\ (\textcolor{blue}{http://www.cdc.noaa.gov/Datasets/ferret/data/}).  The generated map, figure~\ref{fig:antMap} uses a stereographic projection centered on the South Pole and displays all of the elevation data below 60$^{\circ}$ S in meters.  The 35-day flight path of the ANITA payload is plotted in solid \textcolor{red}{red}, bases and research stations are point-like squares, and the AWS stations are black x's.

%%%%%%%%%%%%%%%%%%%%%%%%%%%%%% -*- Mode: Latex -*- %%%%%%%%%%%%%%%%%%%%%%%%%%%%
%% >>appendix/antarctica/prettyCode.tex<<
%% Author          : R. Jeffrey Kowalski
%% Created On      : Wed Jul 25 21:51:12 HST 2007
%% Last Modified On: Wed Aug  1 16:48:23 HST 2007
%%%%%%%%%%%%%%%%%%%%%%%%%%%%%%%%%%%%%%%%%%%%%%%%%%%%%%%%%%%%%%%%%%%%%%%%%%%%%%%
% Contact: Jeff Whitaker <jeffrey.s.whitaker@noaa.gov>
% The colors available with this are: red, green, and blue (for screen display) and cyan, magenta, and yellow (to go with black for the CMYK color model for printing)
% \textcolor{color}{words to be in color}
\newpage
% \begin{spacing}{1}
% \begin{boxedverbatim}
% \fbox{
% \begin{minipage}[t]{6in}
% \begin{singlespace}
\startsinglespace
{\scriptsize
\noindent \#!/usr/bin/python \\
\\
\textcolor{magenta}{from} pylab \textcolor{magenta}{import *} \\
\textcolor{magenta}{from} matplotlib.toolkits.basemap \textcolor{magenta}{import} Basemap \\
\textcolor{magenta}{import} gpsCoordinateDatabase \textcolor{magenta}{as} gps \\
\\
\textbf{def} makeMap\textcolor{magenta}{(}\textcolor{magenta}{)}\textcolor{magenta}{\textcolor{magenta}{:}} \\
\indent m \textcolor{magenta}{$=$} Basemap\textcolor{magenta}{(}projection\textcolor{magenta}{$=$}\textcolor{red}{'spstere'},boundinglat\textcolor{magenta}{$=$}-60.,lat\_0\textcolor{magenta}{$=$}-90.,lon\_0\textcolor{magenta}{$=$}180.,resolution\textcolor{magenta}{$=$}\textcolor{red}{'i'}, \\ \indent area\_thresh\textcolor{magenta}{$=$}10000.\textcolor{magenta}{)} \\
\indent m.drawmapboundary\textcolor{magenta}{(}color\textcolor{magenta}{$=$}\textcolor{red}{'b'}\textcolor{magenta}{)} \\
\indent m.drawmeridians\textcolor{magenta}{(}arange\textcolor{magenta}{(}0,360,10\textcolor{magenta}{)}, linewidth\textcolor{magenta}{$=$}0.7, linestyle\textcolor{magenta}{$=$}\textcolor{red}{'--'}, labels\textcolor{magenta}{$=$}[1, 1, 1, 1]\textcolor{magenta}{)} \\
\indent m.drawparallels\textcolor{magenta}{(}arange\textcolor{magenta}{(}-90,-62,2\textcolor{magenta}{)}, linewidth\textcolor{magenta}{$=$}0.5, linestyle\textcolor{magenta}{$=$}\textcolor{red}{'--'}, labels\textcolor{magenta}{$=$}[1, 0, 0, 1]\textcolor{magenta}{)} \\
\\
\indent \textcolor{green}{\#this gives elevation contours} \\
\indent etopo\textcolor{magenta}{$=$}load\textcolor{magenta}{(}\textcolor{red}{'etopo20data.gz'}\textcolor{magenta}{)} \\
\indent lons\textcolor{magenta}{$=$}load\textcolor{magenta}{(}\textcolor{red}{'etopo20lons.gz'}\textcolor{magenta}{)} \\
\indent lats\textcolor{magenta}{$=$}load\textcolor{magenta}{(}\textcolor{red}{'etopo20lats.gz'}\textcolor{magenta}{)} \\
\indent x, y \textcolor{magenta}{$=$} m\textcolor{magenta}{(}*meshgrid\textcolor{magenta}{(}lons,lats\textcolor{magenta}{)}\textcolor{magenta}{)} \\
\indent \# make filled contour plot. \\
\indent cs \textcolor{magenta}{$=$} m.contourf\textcolor{magenta}{(}x,y,etopo,30,cmap\textcolor{magenta}{$=$}cm.bone\textcolor{magenta}{)} \\
\indent colorbar\textcolor{magenta}{(}\textcolor{magenta}{)} \\
\\
\indent \textcolor{green}{\#plot some locations of Common bases in Antarctica with Names} \\
\indent commonSourceNames \textcolor{magenta}{$=$}  [\textcolor{red}{'McMurdo'},\textcolor{red}{'Amundsen-Scott'},\textcolor{red}{'Vostok'},\textcolor{red}{'Palmer Station'}, \textcolor{red}{'Taylor Dome'}, \textcolor{red}{'WAIS'}, \\ \indent \textcolor{red}{'Scott Base'}, \textcolor{red}{'Novolazarevskaya'}, \textcolor{red}{'Blue Fields Camp'}, \textcolor{red}{'Neumayer'}, \textcolor{red}{'Beardmore South Camp'}, \textcolor{red}{'Byrd Station'}, \\ \indent \textcolor{red}{'Central Western Camp'}, \textcolor{red}{'Crary Ice Rise'}, \textcolor{red}{'Dome C'}, \textcolor{red}{'Siple Dome'}, \textcolor{red}{'South Ice'}] \\
\indent xcommonSourceLats, xcommonSourceLons \textcolor{magenta}{$=$} m\textcolor{magenta}{(}gps.commonSourceLons, gps.commonSourceLats\textcolor{magenta}{)} \\
\indent commonSource \textcolor{magenta}{$=$} m.plot\textcolor{magenta}{(}[xcommonSourceLats],[xcommonSourceLons], \textcolor{red}{'o'}, markersize\textcolor{magenta}{$=$}3, \\ \indent markerfacecolor\textcolor{magenta}{$=$}\textcolor{red}{'r'}, markeredgecolor\textcolor{magenta}{$=$}\textcolor{red}{'k'}\textcolor{magenta}{)} \\
\indent \textbf{for} name, xpt, ypt \textbf{in} \textcolor{blue}{zip}\textcolor{magenta}{(}commonSourceNames, xcommonSourceLats, xcommonSourceLons\textcolor{magenta}{)}\textcolor{magenta}{:} \\
\indent \indent text\textcolor{magenta}{(}xpt,ypt,name, fontsize\textcolor{magenta}{$=$}4.5\textcolor{magenta}{)} \\
\\
\indent \textcolor{green}{\#Plot expedition paths for 2006-07 season from ANITA note 341} \\
\indent xCECpathLat, xCECpathLon \textcolor{magenta}{$=$} m\textcolor{magenta}{(}gps.CorreneErasmusLongitude, gps.CorreneErasmusLatitude\textcolor{magenta}{)} \\
\indent CEC \textcolor{magenta}{$=$}  m.plot\textcolor{magenta}{(}[xCECpathLat],[xCECpathLon],\textcolor{red}{'g-.'},markersize\textcolor{magenta}{$=$}4\textcolor{magenta}{)} \\
\indent xWDpathLat, xWDpathLon \textcolor{magenta}{$=$} m\textcolor{magenta}{(}gps.WiltonDaviesLon, gps.WiltonDaviesLat\textcolor{magenta}{)} \\
\indent WD \textcolor{magenta}{$=$} m.plot\textcolor{magenta}{(}[xWDpathLat],[xWDpathLon],\textcolor{red}{'m:'},markersize\textcolor{magenta}{$=$}4\textcolor{magenta}{)} \\
\indent xn2iLat, xn2iLon \textcolor{magenta}{$=$} m\textcolor{magenta}{(}gps.n2iLon, gps.n2iLat\textcolor{magenta}{)} \\
\indent NI \textcolor{magenta}{$=$}  m.plot\textcolor{magenta}{(}[xn2iLat],[xn2iLon],\textcolor{red}{'b:'},markersize\textcolor{magenta}{$=$}4\textcolor{magenta}{)} \\
\indent \textcolor{green}{\#Plot ITASE data from Jiwoo's elog \#360} \\
\indent xITASElat, xITASElon = m\textcolor{magenta}{(}gps.ITASElon, gps.ITASElat\textcolor{magenta}{)} \\
\indent ITASE = m.plot\textcolor{magenta}{(}[xITASElat],[xITASElon],\textcolor{red}{':'},color \textcolor{magenta}{$=$} colors[2], markersize\textcolor{magenta}{$=$}3\textcolor{magenta}{)}
\\
\indent \textcolor{green}{\#Plot locations of Automatic-Weather Stations in Antarctica from Peter's list on elog 341} \\
\indent xAWSlat, xAWSlon \textcolor{magenta}{$=$} m\textcolor{magenta}{(}gps.AWSlon, gps.AWSlat\textcolor{magenta}{)} \\
\indent AWS \textcolor{magenta}{$=$} m.plot\textcolor{magenta}{(}[xAWSlat],[xAWSlon],\textcolor{red}{'kx'},markersize\textcolor{magenta}{$=$}3\textcolor{magenta}{)} \\
\\
\indent \textcolor{green}{\#input ANITA flight trajectory} \\
\indent anitaLat, anitaLon \textcolor{magenta}{$=$} readAnitaGPS\textcolor{magenta}{(}\textcolor{magenta}{)} \\
\indent xanitaLat, xanitaLon \textcolor{magenta}{$=$} m\textcolor{magenta}{(}anitaLon, anitaLat\textcolor{magenta}{)} \\
\indent FlightPath \textcolor{magenta}{$=$} m.plot\textcolor{magenta}{(}[xanitaLat],[xanitaLon],\textcolor{red}{'r-'}, markersize\textcolor{magenta}{$=$}2, linewidth\textcolor{magenta}{$=$}0.7\textcolor{magenta}{)} \\
\\
\indent legend\textcolor{magenta}{(} \textcolor{magenta}{(}FlightPath, CEC, WD, NI, AWS\textcolor{magenta}{)}, \textcolor{magenta}{(}\textcolor{red}{'ANITA 35-day Flight Path'}, \textcolor{red}{'Correne, Erasmus, \& Coetzer'}, \\ \indent \textcolor{red}{'John Wilton Davies'}, \textcolor{red}{'Team N2i (Novo to Inaccessibility)'}, \textcolor{red}{'AWS Stations'}\textcolor{magenta}{)}, loc\textcolor{magenta}{$=$}3, shadow\textcolor{magenta}{$=$}\textcolor{cyan}{True}\textcolor{magenta}{)} \\
\noindent \textbf{if} \_\_name\_\_ \textcolor{magenta}{\textcolor{magenta}{$=$}\textcolor{magenta}{$=$}} \textcolor{red}{''\_\_main\_\_"}\textcolor{magenta}{:} \\
\indent figure\textcolor{magenta}{(}1\textcolor{magenta}{)} \\
\indent makeMap\textcolor{magenta}{(}\textcolor{magenta}{)} \\
\indent show\textcolor{magenta}{(}\textcolor{magenta}{)} \\
}
% }
% \end{singlespace}
% \end{minipage}
% \end{spacing}
% \end{boxedverbatim}
% \verbatimtabinput[]{appendix/antarctica/prettyCode.tex}
%%%%%%%%%%%%%%%%%%%%%%%%%%%%%% -*- Mode: Latex -*- %%%%%%%%%%%%%%%%%%%%%%%%%%%%
%% >>appendix/antarctica/antarctica.tex<<
%% Author          : R. Jeffrey Kowalski
%% Created On      : Wed Jul 25 21:51:12 HST 2007
%% Last Modified On: Wed Jul 25 21:51:12 HST 2007
%%%%%%%%%%%%%%%%%%%%%%%%%%%%%%%%%%%%%%%%%%%%%%%%%%%%%%%%%%%%%%%%%%%%%%%%%%%%%%%
\begin{landscape}
\begin{figure}[htbp]
\centering
% \epsfxsize=8.3in\epsfbox{figures/appendix/ANTtopoFlightMap_anita2007.eps}
\epsfxsize=8.3in\epsfbox{figures/appendix/antMap_legendOutside.eps}
% \caption{}
\label{fig:antMap}
\end{figure}
\end{landscape}