%%%%%%%%%%%%%%%%%%%%%%%%%%%%%% -*- Mode: Latex -*- %%%%%%%%%%%%%%%%%%%%%%%%%%%%
%% >>conclusion/conclusion.tex<<
%% Author          : R. Jeffrey Kowalski
%% Created On      : Fri Mar 27 13:13:03 2007
%% Last Modified On: Thu Aug  2 10:54:56 HST 2007
%%%%%%%%%%%%%%%%%%%%%%%%%%%%%%%%%%%%%%%%%%%%%%%%%%%%%%%%%%%%%%%%%%%%%%%%%%%%%%%
SLAC T486 provided the confirmation of the Askaryan effect in ice:  coherent radio Cherenkov emission from high-energy particle cascades in dense media is detectable.  Such a validation has already been performed for two other dielectrics (salt, silica sand) which have similar radio properties as ice for observing the Askaryan effect.

\par In this analysis, I have demonstrated the quadratic scaling (1.84533 $\pm$ 0.07013) of Cherenkov pulse power with shower energy which indicates that the radiation is coherent over 2.6-3.95 GHz.  Within rms uncertainties, the electric field strength of radio impulses received at the standard gain horn were in good agreement with simulations incorporating electrodynamics and shower properties in ice.  This result illustrates the roll-off of the frequency spectrum for coherent radio Cherenkov emission.

\par These results provide a promising outlook for existing ultra-high energy neutrino detectors using ice as their interaction medium.  The ANITA experiment has just completed its first flight observing the vast majority of the Antarctica ice for neutrino induced electromagnetic showers.  Having accelerator-based experimental data confirming the electric field frequency spectrum that follows our current theory will improve efforts and sensitivity in detecting UHE neutrinos.  Furthermore, SLAC T486 has illuminated what effect we can expect to see from radiation patterns penetrating the ice surface from electromagnetic cascades in the  $10^{18}\rightarrow10^{22}$ eV energy regime.