%%%%%%%%%%%%%%%%%%%%%%%%%%%%%% -*- Mode: Latex -*- %%%%%%%%%%%%%%%%%%%%%%%%%%%%
%% >>abstract.tex<<
%% Author          : R. Jeffrey Kowalski
%% Created On      : Fri Nov 3 22:10:37 2006
%% Last Modified On: Thu Aug  2 10:54:56 HST 2007
%%%%%%%%%%%%%%%%%%%%%%%%%%%%%%%%%%%%%%%%%%%%%%%%%%%%%%%%%%%%%%%%%%%%%%%%%%%%%%%
\begin{abstract}
\par First hypothesized by Gurgen Askaryan in the 1960's and later confirmed in 2001 at SLAC (\textbf{S}tanford \textbf{L}inear \textbf{A}ccelerator \textbf{C}enter), radio Cherenkov detection techniques are possible in the ultra-high energy regime ($10^{18}\rightarrow10^{22}$ eV) while observing electromagnetic cascades in dielectric media.  This method of detection has now moved into the field of neutrino astrophysics.  Recently, the interest in using ice as a dielectric medium to observe coherent microwave Cherenkov pulses from ultra-high energy neutrino induced particle showers has grown considerably with advances from experiments such as RICE, FORTE, and ANITA-lite.  ANITA (\textbf{AN}tarctic \textbf{I}mpulsive \textbf{T}ransient \textbf{A}ntenna), is a radio telescope designed to exploit this effect while looking for UHE neutrino interactions in Antarctic ice.  In June 2006, ANITA observed these highly coherent radio impulses in SLAC's ESA (\textbf{E}nd \textbf{S}tation \textbf{A}) with 28.5 GeV electrons interacting with a 7.5 tonne ice target to produce the EM shower.  These first measurements of the Askaryan effect in ice were consistent with shower and electrodynamics simulations for ice and provided a clear indication that the radiation is coherent over the 200-1200 MHz frequency window.  In addition to the ANITA payload in SLAC's ESA, four log-period dipole array antennas, two monocone antennas, and one high frequency gain horn (nominally 2.6-3.95 GHz) recorded impulsive events emanating from the ice target.  I report on further analysis of coherent radio Cherenkov impulses using the standard gain horn and demonstrate that the results are fully consistent with theoretical expectations.
\end{abstract}
