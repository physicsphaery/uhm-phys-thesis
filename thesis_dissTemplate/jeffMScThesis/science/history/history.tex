%%%%%%%%%%%%%%%%%%%%%%%%%%%%%% -*- Mode: Latex -*- %%%%%%%%%%%%%%%%%%%%%%%%%%%%
%% >>science/histroy/history.tex<<
%% Author          : R. Jeffrey Kowalski
%% Created On      : Thu Apr 12 23:37:21 HST 2007
%% Last Modified On: Thu Aug  2 10:54:56 HST 2007
%%%%%%%%%%%%%%%%%%%%%%%%%%%%%%%%%%%%%%%%%%%%%%%%%%%%%%%%%%%%%%%%%%%%%%%%%%%%%%%
\subsection{Experimental Support for the GZK Cutoff}
\label{s:uhecrDetection}
When a high energy cosmic ray hits the upper atmosphere ($\sim$20 km), a jet of particles is created that continues to travel in the same direction as the cosmic ray.  As particles in the jet collide with nuclei of oxygen and nitrogen in the air, secondary cosmic rays are produced which can yield over a million particles which can be detected by a large telescope array on Earth.  The detection of Extensive Air Showers (EAS) has proven to be a successful means of measuring the the energy spectra of ultra-high energy cosmic rays which extends beyond 10$^{20}$ eV.  Where does GZK suppression begin?

\par Currently, the best sources for the highest energy cosmic rays has come from five experimental collaborations:  AGASA~\cite{AGASA.1998}, Fly's Eye~\cite{FLYSEYE.1993}, Haverah Park~\cite{haverah.2002}, HiRes~\cite{HiRES.2003}, and Yakutsk~\cite{Yakutsk.1991}.  The results from the five different experiments has a two-fold conclusion suggesting that all data with energies below 10$^{20}$ eV is in good agreement while all data consistent with a GZK suppression points above 10$^{20}$ eV with the exception of the AGASA collaboration~\cite{Bahcall.2003}.  The results from HiRes and AGASA have created an inconsistency in the literature for experimental evidence that supports a GZK cutoff that is still under debate.

\par The next major experiment to probe the energy and spectrum of UHECR's is the Pierre Auger Cosmic Ray Observatory (AUGER).  Built in western Argentina, this large-scale surface array utilizes two methods for detection of UHECR's (both air fluorescence telescopes and water Cherenkov methods) making it a ''hybrid detector."  Although the Auger Observatory is only $\sim$85 \% complete, results have already been announced confirming the presence of the ankle and steepening of the CR spectrum near $\sim$10$^{19}$ eV~\cite{icrc0318}.

\par Neutrino astronomy has recently become a major player in the ability to confirm GZK suppression.  Existing experiments such as ANITA, IceCube, RICE, etc. and next generation experiments being planned (EUSO, OWL, SalSA, X-RICE) may confirm the existence of GZK neutrinos which will compliment and expand AUGER observations.  The study of UHE $\nu$'s has an advantage of allowing for observation into the interior of objects whereas protons can only show the surface since interactions during propagation obscure the early Universe from direct observation~\cite{Seckel.2005}.