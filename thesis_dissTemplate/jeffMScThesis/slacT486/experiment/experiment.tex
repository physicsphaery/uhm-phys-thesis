%%%%%%%%%%%%%%%%%%%%%%%%%%%%%% -*- Mode: Latex -*- %%%%%%%%%%%%%%%%%%%%%%%%%%%%
%% >>slacT486/experiment/experiment.tex<<
%% Author          : R. Jeffrey Kowalski
%% Created On      : Tue Apr 10 19:46:26 HST 2007
%% Last Modified On: Thu Aug  2 10:54:56 HST 2007
%%%%%%%%%%%%%%%%%%%%%%%%%%%%%%%%%%%%%%%%%%%%%%%%%%%%%%%%%%%%%%%%%%%%%%%%%%%%%%%
\section{Experimental Setup}
\label{s:slacSetup}
Figure~\ref{fig:slacExpSetup} shows the overall setup of T486 at ESA.  28.5 GeV electrons were accelerated with the SLAC Linac and deposited into a $\sim$7.2 metric tonne ice target ($n_{ice} = 1.8$).  From the Askaryan process, Cherenkov radiation propagates out of the target at an angle of $\sim$56$^\circ$ where it refracts at the surface carved to a mean slope of $\sim$8$^\circ$.  The resulting emission was then detected with the ANITA receiver system at an angle of $\sim$25$^\circ$ from the electron beam axis.  In addition to the ANITA payload, several exterior antennas (LPDA, Standard Gain Horn, Dipole, etc.) were placed around the ice target and are not shown in figure~\ref{fig:slacExpSetup} but will be discussed in section~\ref{ss:extAnt} and will be the emphasis of this analysis.
\begin{figure}[htbp]
\centering
\epsfxsize=4.0in\epsfbox{figures/slacT486/slacExperimentSetup.eps}
\caption{SLAC T486 Experimental Setup:  Rendered view of the ANITA payload with the 28.5 GeV electron beam at SLAC's ESA.}
\label{fig:slacExpSetup}
\end{figure}
%
\subsection{Beam Current}
\label{ss:beamCurrent}
The electromagnetic showers were produced with 28.5 GeV electrons accelerated by the SLAC Linac.  In most cases, 10$^9$ particles were delivered in 10 picosecond bunches.  Two toroids within the Linac were used to monitor the beam current.  Toroid 34 is located upstream in the beam switchyard and seems to be somewhat unreliable at low currents due to the granularity of it's ADC.  Further downstream, toroid 4140 was set up after the experiment started as seen in figure~\ref{fig:beamCurrent} and has an overall gain correction factor of 10 in order to correlate the 2 sets of data.  For the purpose of matching beam current with antenna peak voltages, only toroid 34 will be used in the analysis.

\begin{figure}[htbp]
\centering
\epsfxsize=4.0in\epsfbox{figures/slacT486/t486BeamCurrent.eps}
\caption{SLAC T486 Beam Current:  Beam Current vs. time showing the number of electron bunches per unit energy per unit time.  The data sets came from 2 individual monitoring toroids which span the duration of the experiment.}
\label{fig:beamCurrent}
\end{figure}
%
\subsection{RF Monitor and External Antennas}
\label{ss:extAnt}

\begin{figure}[htbp]
\centering
\epsfxsize=4.0in\epsfbox{figures/slacT486/dipole.eps}
\caption{Diagram showing the relative position of the dipole antenna to the beam pipe and the external RF absorber.}
\label{fig:dipole}
\end{figure}

Several exterior antennas were setup as additional monitors from the ANITA payload for SLAC T486.  A dipole antenna, depicted in figure~\ref{fig:dipole}, served as an absolute beam current monitor and as a calibration for the remainder of the antennas receiving Askaryan pulses emanating from the ice.  Two LPDA model EM-6952 antennas manufactured by \textit{Electro-Metrics} with 1-18 GHz of bandwidth were mounted on a wooden structure positioned on a concrete wall that stood $\sim$5 m from the center of the ice target.  One LPDA was directed in the horizontal plane while the other in the vertical plane.  Two custom-made Monocone antennas were positioned on a similar wooden structure on a concrete wall adjacent to the LPDA's on the other surface.  Each monocone was directed in the horizontal and vertical plane $\sim$5 m from the center of the ice target which stood between the two concrete walls.  Below the two monocone's, a \textit{Scientific-Atlanta} Standard Gain Horn model 12-2.6 with 2.6-3.95 GHz of bandwidth was positioned $\sim$4.5 m from the center of the ice and aligned with boresight.  In addition to the RF antennas mentioned, two PCB LPDA's were placed below the LPDA's mentioned previously but will not mentioned in the data analysis.  The description of the overall antenna setup is shown in figure~\ref{fig:externalAnts}

\begin{figure}[htbp]
\centering
\epsfxsize=5.0in\epsfbox{figures/slacT486/extAntSetup.eps}
\caption{Antenna positions for T486:  Top left and right show the positions of the gain horn, LPDA's, Monocones, PCB LPDA's, and the ANITA instrument relative to the ice target.}
\label{fig:externalAnts}
\end{figure}

\subsection{Ice Target}
\label{ss:iceTarget}
The ice target for T486 was constructed using $\sim$55 136 kg blocks of carving grade ice.  Each block, with dimensions 25.4 cm $\times$ 53.3 cm $\times$ 114.3 cm, was lowered into the target housing made predominately of 1.9 cm plywood to form a monolithic block 2 m $\times$ 1.5 m $\times$ 5 m.  The housing was lined with 10.2 cm foam thermal insulation with a thin layer of hypalon liner to contain any melting water.  Beneath the ice block and hypalon liner was a layer of 10 cm ferrite tile to suppress reflections of Cherenkov radiation that was generated downwards.  The entire volume was regulated between -5$^\circ$C to -20$^\circ$C throughout the duration of the experiment.

\par The top surface of the ice target was carved to a slope of $\sim$8$^\circ$ to prevent total internal reflection (TIR) from Cherenkov emission near the surface.  The roughness of the surface after carving was measured to have a 2.3 cm root-mean-square (rms).  Figure~\ref{fig:icePhotos} shows some photographs of the ice surface and the target housing with the ice surface grid-map superimposed in the bottom-left photograph.

\begin{figure}[htbp]
\centering
\epsfxsize=4.5in\epsfbox{figures/slacT486/IceTargetPhotos.eps}
\caption{Photographs of SLAC T486 Ice Target:  (Top left) Front view of ice target with $\sim$8$^\circ$ slope.  (Top right) Front view of ice target with foam insulation cover and cooling ducts shown.  This was the configuration for the majority of runs during the experiment.  (Bottom left) Close-up showing the surface roughness of the ice target.  (Bottom right) Top view of ice target with measured surface roughness grid from appendix~\ref{tab:iceRoughness} (not to scale).}
\label{fig:icePhotos}
\end{figure}

% \subsubsection{Surface Roughness}
\paragraph{Surface Roughness}
% Surface Roughness Table
% \input{slacT486/experiment/iceSurfaceDataTable.tex}
%
Several measurements were made on the completed surface to quantify effects of surface roughness that will have more importance with data analysis from the December 2006 ANITA flight.  A 13 $\times$ 6 element grid was developed to get a precise value of effective surface height and slope.  At each grid point, an effective surface height and slope were measured and summarized in appendix~\ref{ap:iceTarget}.  Figure~\ref{fig:IceSurfaceGrid} shows the design of the ice target and the layout of the constructed surface grid and the measured surface is plotted in figure~\ref{fig:IceSurfacePlot}.

\begin{figure}[htbp]
\centering
\epsfxsize=4.5in\epsfbox{figures/slacT486/IceSurfaceGridMap.eps}
\caption{Schematic view of the Ice Target's Surface Roughness:  The above describes the physical layout of the ice target for T486 (not to scale).}
\label{fig:IceSurfaceGrid}
\end{figure}

\begin{figure}[htbp]
\centering
\epsfxsize=4.5in\epsfbox{figures/slacT486/iceTargetSurface.eps}
\caption{Plot of T486 ice surface.  All values are in inches.  Note the pronounced change in slope at $\approx$ 140" $\approx$ 3.5 m.}
\label{fig:IceSurfacePlot}
\end{figure}

\par Due to the apparent change in slope of the ice target surface, it becomes necessary to divide the ice surface into two distinct regions - A \& B.  Figure~\ref{fig:iceSlopes} shows the division of the ice surface into these regions.  By fitting a plane to each surface ($h_{fit} = A \times l + B \times w + C$), which is shown in red, to the measured grid points (blue), two slopes can be derived.  For Region A, the fit reveals a slope of 8.4$^\circ$ while Region B has a slope of 2.6$^\circ$.  Since shower max occurs in Region A from $\sim$2.4 m $\rightarrow$ $\sim$2.8 m, a slope of 8.4$^\circ$ will be used in this analysis for the radiated Cherenkov emission.

\begin{figure}[htbp]
\centerline{
\mbox{
\epsfxsize=3.0in\epsfbox{figures/slacT486/iceTarget_p1.eps}
\epsfxsize=3.0in\epsfbox{figures/slacT486/iceTarget_p2.eps}
}}
\caption{Left:  Region A and Right: Region B of the T486 ice target.  The fitted plane (red) to the measured grid points (blue) is shown revealing two distinct inclination angles - 8.4$^\circ$ and 2.6$^\circ$ respectively.  All dimensions are in inches here.}
\label{fig:iceSlopes}
\end{figure}