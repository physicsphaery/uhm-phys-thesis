%%%%%%%%%%%%%%%%%%%%%%%%%%%%%% -*- Mode: Latex -*- %%%%%%%%%%%%%%%%%%%%%%%%%%%%
%% >>slacT486/intro.tex<<
%% Author          : R. Jeffrey Kowalski
%% Created On      : Tue Apr 10 19:50:51 HST 2007
%% Last Modified On: Thu Aug  2 10:54:56 HST 2007
%%%%%%%%%%%%%%%%%%%%%%%%%%%%%%%%%%%%%%%%%%%%%%%%%%%%%%%%%%%%%%%%%%%%%%%%%%%%%%%
From June 19-24, 2006, the experiment, SLAC T486, was performed in the End Station A facility at the Stanford Linear Accelerator Center to measure the Askaryan effect in ice.  28.5 GeV electrons were accelerated with typically 10$^9$ particles in 10 picosecond bunches and delivered into a 7.5 metric tonne target of carving-grade ice to produce electromagnetic showers.  In a dense media like ice, coherent microwave Cherenkov radiation emerges from the particle shower and propagates to the surface of the target where radio antennas can detect the radiation.  This chapter outlines the T486 experiment and the analysis of the Askaryan effect in ice.