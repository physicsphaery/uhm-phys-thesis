%%%%%%%%%%%%%%%%%%%%%%%%%%%%%% -*- Mode: Latex -*- %%%%%%%%%%%%%%%%%%%%%%%%%%%%
%% >>slacT486/analysis/analysis.tex<<
%% Author          : R. Jeffrey Kowalski
%% Created On      : Tue Apr 10 19:46:26 HST 2007
%% Last Modified On: Wed Jun  6 20:26:31 HST 2007
%%%%%%%%%%%%%%%%%%%%%%%%%%%%%%%%%%%%%%%%%%%%%%%%%%%%%%%%%%%%%%%%%%%%%%%%%%%%%%%
\section{Data Acquisition}
\label{s:acq}
All measurements of pulsed RF emission were made in the time domain with the ANITA instrument (described in section~\ref{s:DACsystem}; a Tektronix 694C real-time digital sampling oscilloscope with 3 GHz and 10 GSa/s, eight-bit linear digitization of four channels; and with a Hewlett-Packard 54121A with 20 GHz of bandwidth.  The signals received at the ANITA antennas were attenuated by 60 dB to bring signals into an acceptable range for 50 $\ohm$ inputs while 20 dB of attenuation was used on received signals from the dipole antenna.  Table~\ref{tab:extAntenna} summarizes the mapping of antennas to the TDS694C and HP54121A with a switch occuring on the morning of June 20, 2006.

\begin{center}
\begin{table}
\caption{A summary of the additional antenna used in T486 with their mapping to oscilloscope.  The top half of the table shows the initial setup for data acquisition and the bottom half denotes the setup after 06/20/06 $\sim$2am PDT.}
\begin{tabular}{| l | c | c | c | c | c |} \hline
Antenna & Polarization & Channel & Device & Distance from Ice Target \\
\hline \hline
PCB LPDA & H & 1 & TDS694C & 5.2 m \\
PCB LPDA & V & 2 & TDS694C & 5.2 m \\
Gain Horn & V & 3 & TDS694C & 4.5 m\\
Dipole & & 4 & TDS694C & N/A \\
LPDA & H & 1 & HP54121A & 5.2 m \\
LPDA & V & 2 & HP54121A & 5.2 m \\
Monocone & H & 3 & HP54121A & 5.2 m \\
Monocone & V & 4 & HP54121A & 5.2 m \\
\hline
Monocone & H & 1 & TDS694C & 5.2 m \\
Monocone & V & 2 & TDS694C & 5.2 m \\
Gain Horn & V & 3 & TDS694C & 4.5 m\\
Dipole & & 4 & TDS694C & N/A \\
LPDA & H & 1 & HP54121A & 5.2 m \\
LPDA & V & 2 & HP54121A & 5.2 m \\
PCB LPDA & H & 3 & HP54121A & 5.2 m \\
PCB LPDA & V & 4 & HP54121A & 5.2 m \\
\hline
\end{tabular}
\label{tab:extAntenna}
\end{table}
\end{center}

\begin{figure}[htbp]
\centering
\epsfxsize=4.0in\epsfbox{figures/slacT486/wvfmsRUN126evt888_newPSD.eps}
\caption{Waveforms with the respective power spectrum from run 126, event 888.}
\label{fig:anitaEvtWfm}
\end{figure}

\begin{figure}[htbp]
\centering
% \epsfxsize=4.0in\epsfbox{figures/slacT486/wvfmsRUN126evt888_newPSD.eps}
\caption{Sample ANITA event waveform from average beam current.}
\label{fig:EvtWfm}
\end{figure}

\subsection{Dipole Calibration}
\label{ss:dipoleCal}
To calibrate the transition radiation entering the ice target, 104 runs with typically 1000-2000 events were used to build a correlation between the peak voltages received by the dipole antenna and the generated beam current.  Due to the poor quality of the beam current data set, only 8 runs passed a selection criteria which was motivated by well-behaved dipole peak distributions (figure~\ref{fig:dipHrnDist}, and the beam current tracking the dipole peaks consistently).

\begin{figure}[htbp]
\centering
\epsfxsize=4.0in\epsfbox{figures/slacT486/run92DipoleHornPeakDist.eps}
\caption{Peak voltage distributions for the dipole antenna and gain horn which is normalized to the mean of the dipole distribution.}
\label{fig:dipHrnDist}
\end{figure}

\par Figure~\ref{fig:dipoleCal} shows the coherence of the detected power as a function of beam current (\textbf{\textsc{try using electron number in the beam pulse}}).

\begin{figure}[htbp]
\centering
\epsfxsize=4.0in\epsfbox{figures/slacT486/coherence.eps}
\caption{Dipole-Beam Current Calibration}
\label{fig:dipoleCal}
\end{figure}

\noindent Apply calibration to all waveforms.

\subsection{Gain Horn Field Simulation}
\label{ss:gainHorn}
The response of the gain horn was generated using a MATLAB routine that solves for the electric and magnetic fields from the physical dimensions of the horn antenna.  Figure~\ref{fig:fieldPattern} shows the \textbf{E}- and \textbf{H}-plane beam patterns for the for the low, mid, and high-bands of the horn antenna.

\begin{figure}[htbp]
\centering
\epsfxsize=4.0in\epsfbox{figures/slacT486/gainHornField.eps}
\caption{\textbf{E} \& \textbf{H} - Fields of low, mid, and high-bands for Standard Gain Horn (\textit{Scientific-Atlanta, Inc.}).}
\label{fig:fieldPattern}
\end{figure}

\par In order to calculate the effective height of the horn antenna, several parameters needed to be numerically simulated.  From Krauss~\cite{}, we define the aperture efficiency as

\begin{equation}
\varepsilon_{ap} = \frac{A_{e}}{A_{p}},
\end{equation}

\noindent where A$_{e}$ and A$_{p}$ are the effective area and physical area of the antenna's aperture.  Since the physical area of the antenna aperture is a known quantity, we can determine the effective area with the use of another MATLAB function which determines the aperture efficiency.  The effective area of an antenna is related to the boresight gain, G, and the free-space wavelength, $\lambda$, of the radiation by the formula:

\begin{equation}
G = \frac{4\pi}{\lambda^{2}}A_{e}.
\end{equation}

\par The directivity of the gain horn is related to the HPBW (Half-Power-Beam-Width) by

\begin{equation}
D \approx \frac{4\pi}{\Omega_{A}} \approx \frac{41253^{\circ}}{D_{H}D_{E}},
\end{equation}

\noindent where $\Omega_{A}$ is the beam solid angle, and D$_{H}$ and D$_{E}$ are the HPBW in the \textbf{H}- and \textbf{E}-plane respectively.  Another parameter related to the antenna aperture is the \emph{effective height}, h$_{eff}$, which is expressed in meters.  This quantity is simply the ratio of the induced voltage to the incident electric field:

\begin{equation}
h_{eff} = \frac{V}{E}.
\end{equation}

\noindent Krauss~\cite{} expresses the effective height in terms of the effective aperture, A$_{e}$, radiation resistance, Z$_r$, and the intrinsic inpedence of free space, Z$_0 = 377 \Omega$ which is given by

\begin{equation}
h_{eff} = 2\sqrt{\frac{Z_{r}A_{e}}{Z_0}}.
\end{equation}

\begin{figure}[htbp]
\centerline{
\mbox{
\epsfxsize=3.5in\epsfbox{figures/slacT486/GEallfreq.eps}
\epsfxsize=3.5in\epsfbox{figures/slacT486/GHallfreq.eps}
}}
\caption{Left: \textbf{E}-plane gain and Right: \textbf{H}-plane gain for low- (green), mid- (blue), and high-bands (orange) of the horn antenna.  The beam pattern narrows significantly with higher frequencies suggesting the gain should increase with frequency.}
\label{fig:gainPattern}
\end{figure}

\begin{figure}[htbp]
\centering
\epsfxsize=4.0in\epsfbox{figures/slacT486/hornHPBWvsFreq.eps}
\caption{Half-power-beam-width as a function of frequency for the Standard Gain Horn showing the main lobe of the beam pattern narrowing at higher frequencies.}
\label{fig:gainDirFreq}
\end{figure}

\begin{figure}[htbp]
\centering
\epsfxsize=4.0in\epsfbox{figures/slacT486/directivityGainVsFreq.eps}
\caption{Directivity (top) and gain (bottom) as a function of frequency.}
\label{fig:gainDirFreq}
\end{figure}

\begin{figure}[htbp]
\centering
\epsfxsize=4.0in\epsfbox{figures/slacT486/effectiveAreaAndHeightVsFreq.eps}
\caption{Effective area (top) and effective height (bottom) for the Standard Gain Horn as a function of frequency.}
\label{fig:gainDirFreq}
\end{figure}

\section{Askaryan Pulse Reconstruction}
\label{ss:askPulseRecon}

\begin{figure}[htbp]
\centering
\epsfxsize=4.0in\epsfbox{figures/slacT486/evt888Run126_horn_PSD.eps}
\caption{Top:  Waveform received by the horn antenna for run 126, event 888 with corresponding power spectrum.}
\label{fig:hornWvfmPsd}
\end{figure}

% \subsection{Discussion of Analysis Code (\textit{Optional})}
% If I have time to talk about Python.