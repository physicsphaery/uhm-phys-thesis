%------------------------------------------------------------------------------
%------------------------------------------------------------------------------
%This is a modified version of the sample UH thesis LaTeX from Robert Brewer
%Modified by Troy Hix APR-12-06
%Robert Brewer's original remarks are included below
%Troy Hix's remarks are in these hyphens: ------------------
%------------------------------------------------------------------------------
%------------------------------------------------------------------------------
%%%%%%%%%%%%%%%%%%%%%%%%%%%%%% -*- Mode: Latex -*- %%%%%%%%%%%%%%%%%%%%%%%%%%%%
%% uhtest.tex -- 
%% Author          : Robert Brewer
%% Created On      : Wed Sep 30 16:08:49 1998
%% Last Modified By: Robert Brewer 
%% Last Modified On: Mon Oct  5 16:17:16 1998
%% RCS: $Id: uhtest.tex,v 1.2 1998/10/06 02:04:56 rbrewer Exp rbrewer $
%%%%%%%%%%%%%%%%%%%%%%%%%%%%%%%%%%%%%%%%%%%%%%%%%%%%%%%%%%%%%%%%%%%%%%%%%%%%%%%
%%   Copyright (C) 1998 Robert Brewer
%%%%%%%%%%%%%%%%%%%%%%%%%%%%%%%%%%%%%%%%%%%%%%%%%%%%%%%%%%%%%%%%%%%%%%%%%%%%%%%

%!!!!!!!!!!!!!!!!!!!!!!!!!!!!!!!!!!!!!!!!!!!!!!!!!!!!!!!!!!!!!!!!!!!!!!!!!!!!!!
%!NOTE: This example file has been prepared according to the University of
%!      Hawaii Style & Policy Manual for Theses and Dissertations dated
%!      "Revised February 1998". If you have one with a later date, you may
%!      need to make revisions to this document as well. In any event, making
%!      sure your thesis complies with Graduate Division guidelines is
%!      ultimately your responsibility. Caveat LaTeXtor. :)
%!!!!!!!!!!!!!!!!!!!!!!!!!!!!!!!!!!!!!!!!!!!!!!!!!!!!!!!!!!!!!!!!!!!!!!!!!!!!!!

%% The options are (you can only choose one from each group):
%%
%% 10pt, 11pt, 12pt: chooses the point size for the document. "11ot" is the
%%                   default.
%%
%% oneside, twoside: whether you want your document onesided or twosided. Note
%%                   that twosided is not guaranteed to work, and style
%%                   guidelines prohibit double sided printouts on final
%%                   copy. "oneside" is the default.
%%
%% draft, final: when printing drafts you can save a lot of paper by using the
%%               "draft" option. It switches to single spacing, displays overful
%%               hboxes with a black box, prints a version number on title page 
%%               and omits signature page. Of course for the final copy make
%%               sure to use the "final" option! "final" is the default.
%%
%% cm, times, palatino, newcent, bookman: switches between different font
%%                                        sets. "cm" is the Computer Modern
%%                                        font (TeX's default), the rest are
%%                                        PostScript fonts. "times" is the
%%                                        default.
%%
%% thesis, dissertation: switches between the style for a master's thesis and a 
%%                       Ph.D. dissertation. The differences are fairly minor
%%                       and limited to the front matter. "thesis" is the
%%                       default.
%%
%% actual, proposal: switches between actual document and proposal mode. In
%%                   proposal mode: the title page is simplified, the
%%                   version number is always printed, and the signature page
%%                   is omitted.
%%
%%% Load the uhthesis2e document class
\documentclass[11pt,final,cm,dissertation,actual]{uhthesis2e}

%%% Load some useful packages:
%% Package to linebreak URLs in a sane manner.
\usepackage{url}
%----------------------------------------------------------------------------
%----------------------------LaTeX Packages----------------------------------
%----------------------------------------------------------------------------
\usepackage{amsmath}
\usepackage[pdftex]{graphicx}
\usepackage{epsfig}
\usepackage{rotating}
\usepackage{fancyhdr}
\usepackage{acronym}
\graphicspath{
{figures/introduction/}
{figures/theory/}
{figures/experiment/}
{figures/conclusion/}
{figures/appendix_mst/}
}
%\numberwithin{equation}{section}
%----------------------------------------------------------------------------
%----------------------------LaTeX definitions-------------------------------
%----------------------------------------------------------------------------
\newcommand{\sinc}{{\rm sinc}}
\newcommand{\cross}{\times}
\newcommand{\Vector}{\mathbf}
\newcommand{\del}{\nabla}
\newcommand{\bra}[1]{\left\langle \hspace{0.10em}#1 \hspace{0.10em}\right|}
\newcommand{\ket}[1]{\left| \hspace{0.15em} #1 \hspace{0.15em}\right\rangle}
\newcommand{\braket}[2]
{\left\langle \hspace{0.10em} #1 \hspace{0.10em}
\right|
#2 \left\rangle \right.}
\newcommand{\expectation}[1]{\left\langle #1 \right\rangle}
%----------------------------------------------------------------------------
%----------------------------------------------------------------------------

%%% End of preamble
\begin{document}

%%% Declarations for Front Matter. Capitalize all of these values
%%% "normally". This allows the document class to format them properly.
%% Full title of thesis or dissertation, capitalized like a title should be.
\title{UltraMega OK study of matter and energy}
%% Your name, capitalized normally. Do not include any titles like Dr.
\author{Paige Turner}
%% Month in which you intend to receive your degree (i.e. graduation).
%% Presumably this will be one of: May, August, or December.
\degreemonth{August}
%% Year of expected graduation.
\degreeyear{2006}
%% Type of degree to be conferred.
\degree{Doctor of Philosophy}
%% This is the chairperson of your committee. Do not use titles like Dr.
\chair{Huang Annsaw}
%% The other members of your committee, seperated by "\\". Again, no titles,
%% and it is customary to list the outside committee member (if you have one)
%% last.
\othermembers{
Sherman Wadd Evver\\
Walter Melon\\
Ariel Hassle\\
Gene E. Yuss }
%% This is the total size of your committee, including the chairperson. The
%% signature page routine will only handle up to 6 members; if you have more
%% than that you will need to modify the document class.
\numberofmembers{5}
%% The field in which you are obtaining your degree, capitalized normally.
\field{Physics}
%% The version number of your document. Consistent use of this will enable you
%% to tell old drafts from new ones. Final actual documents omit this
%% automatically so you can use it without fear of submission problems at the
%% end. If you do not define this parameter, it defaults to "1.0.0".
\versionnum{1.5.0}

%%% Create the title page from all the information above. Note that the
%%% titlepage is outside the front matter.
\maketitle

%\begin{frontmatter}

%%% Create the signature page (when indicated by the options)
%\signaturepage

%%% Create the copyright page
%\copyrightpage

%%% Bring in the dedication page from external file
%%%%%%%%%%%%%%%%%%%%%%%%%%%%%%% -*- Mode: Latex -*- %%%%%%%%%%%%%%%%%%%%%%%%%%%%
%% >>dedication.tex<<
%% Author          : R. Jeffrey Kowalski
%% Created On      : Fri Nov 3 22:08:51 2006
%% Last Modified On: Tue Apr 10 22:09:53 HST 2007
%%%%%%%%%%%%%%%%%%%%%%%%%%%%%%%%%%%%%%%%%%%%%%%%%%%%%%%%%%%%%%%%%%%%%%%%%%%%%%%
\begin{dedication}
\null\vfil
{\large
\begin{center}
To Mom and Dad,\\\vspace{12pt}
Debbie and Rick Kowalski\\\vspace{12pt}
\end{center}}
\vfil\null
\end{dedication}


%%% Bring in the acknowledgements section from external file
%%%%%%%%%%%%%%%%%%%%%%%%%%%%%%% -*- Mode: Latex -*- %%%%%%%%%%%%%%%%%%%%%%%%%%%%
%% >>acknowledgements.tex<<
%% Author          : [Your Name]
%% Created On      : 
%% Last Modified On: 
%%%%%%%%%%%%%%%%%%%%%%%%%%%%%%%%%%%%%%%%%%%%%%%%%%%%%%%%%%%%%%%%%%%%%%%%%%%%%%%
\begin{acknowledgements}
Acknowledge some folks here.
\end{acknowledgements}



%%% Bring in the abstract section from external file
%%%%%%%%%%%%%%%%%%%%%%%%%%%%%%% -*- Mode: Latex -*- %%%%%%%%%%%%%%%%%%%%%%%%%%%%
%% >>abstract.tex<<
%% Author          : [Your Name]
%% Created On      : 
%% Last Modified On: 
%%%%%%%%%%%%%%%%%%%%%%%%%%%%%%%%%%%%%%%%%%%%%%%%%%%%%%%%%%%%%%%%%%%%%%%%%%%%%%%
\begin{abstract}
\par Type your abstract here.
\end{abstract}


%%% Generate table of contents
%\tableofcontents

%%% Generate list of tables
%\listoftables

%%% Generate list of figures
%\listoffigures

%%% Generate list of acronyms
%\chapter*{List of Acronyms} 
\addcontentsline{toc}{chapter}{List of Acronyms} 
\begin{acronym}
\acro{BDP}{Boogie Down Productions}
\acro{KRSONE}{Knowledge Reigns Supreme Over Nearly Everyone}
\acro{SOD}{Stormtroopers of Death}
\acro{MOD}{Method of Destruction}
\acro{COC}{Corrosion Of Conformity}
\acro{CBGB}{Country, Blue Grass \& Blues}
\acro{IROC}{International Race Of Champions}
\acro{NWOBHM}{New Wave Of British Heavy Metal}
\acro{NAMBLA}{Multi--Color Coherent Quantum Control Laser Induced Fluroescence Light Imaging Detection and Ranging}
\end{acronym}

%\end{frontmatter}

%%% Bring in the body of the thesis from external file
%----------------------------------------------------------------------------
%----------------------------------------------------------------------------
The YAG--pumped dye laser system produces 8 ns pulses at 20 Hz - in the transform limit, these pulses should have a spectral width of 55 MHz. In fact, the manufacturer of the dye lasers claims the spectral width of each laser pulse to be 1 GHz with axial modes separated by 600 MHz. These specifications are assumed to not be base on measurement (they could not produce hard data), so we assume they arrived at these numbers through some calculation. For example, with a quick glance at the dye laser cavity, one can see that its approximate length is 30 cm. This corresponds to an axial mode spacing of 500 MHz which is consistent with the manufacture's claim.

If the dye laser output contains multiple longitudinal modes, this should show up as ``beating'' in the intensity profile. This can be seen by direct observation of the laser output on a square law detector using an oscilloscope; however, this method is usually limited by the bandwidth of the scope. For example, if the manufacture's claim is true and the dye laser output has 2 or 3 axial modes separated by 600 MHz, then we should see intensity beats at 600 MHz and 1200 MHz - this is beyond the bandwidth of typical oscilloscopes. However, when analyzed with a narrow band RF receiver, there should be 1 or 2 spectral features (in addition to the ``DC'' feature) at 600 MHz and 1200 MHz - well within the bandwidth of the RF receivers we have in the lab. See the discussion in Chapter 19 in reference \cite{Siegman:1986a}.
%----------------------------------------------------------------------------
%----------------------------------------------------------------------------


%%% Bring in any appendices from external file
%%%%%%%%%%%%%%%%%%%%%%%%%%%%%%% -*- Mode: Latex -*- %%%%%%%%%%%%%%%%%%%%%%%%%%%%
%% >>appendix/appendix.tex<<
%% Author          : [Your Name]
%% Created On      : 
%% Last Modified On: 
%%%%%%%%%%%%%%%%%%%%%%%%%%%%%%%%%%%%%%%%%%%%%%%%%%%%%%%%%%%%%%%%%%%%%%%%%%%%%%%
\appendix
\chapter{Ye Old Appendix}
\label{ap:iceTarget}


%%% Input file for bibliography
%\bibliography{bib/physics,bib/molecules,bib/optics,bib/iodine,bib/LIDAR,bib/nuclear_mag,bib/QuantumControl,bib/program}
%% Use this for an alphabetically organized bibliography
%\bibliographystyle{plain}
%% Use this for a reference order organized bibliography
%\bibliographystyle{unsrt}

\end{document}
